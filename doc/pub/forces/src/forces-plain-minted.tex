%%
%% Automatically generated file from DocOnce source
%% (https://github.com/hplgit/doconce/)
%%


%-------------------- begin preamble ----------------------

\documentclass[%
twoside,                 % oneside: electronic viewing, twoside: printing
final,                   % or draft (marks overfull hboxes, figures with paths)
10pt]{article}

\listfiles               % print all files needed to compile this document

\usepackage{relsize,makeidx,color,setspace,amsmath,amsfonts}
\usepackage[table]{xcolor}
\usepackage{bm,microtype}

\usepackage{fancyvrb} % packages needed for verbatim environments

\usepackage[T1]{fontenc}
%\usepackage[latin1]{inputenc}
\usepackage[utf8]{inputenc}

\usepackage{lmodern}         % Latin Modern fonts derived from Computer Modern

% Hyperlinks in PDF:
\definecolor{linkcolor}{rgb}{0,0,0.4}
\usepackage[%
    colorlinks=true,
    linkcolor=linkcolor,
    urlcolor=linkcolor,
    citecolor=black,
    filecolor=black,
    %filecolor=blue,
    pdfmenubar=true,
    pdftoolbar=true,
    bookmarksdepth=3   % Uncomment (and tweak) for PDF bookmarks with more levels than the TOC
            ]{hyperref}
%\hyperbaseurl{}   % hyperlinks are relative to this root

\setcounter{tocdepth}{2}  % number chapter, section, subsection

\usepackage[framemethod=TikZ]{mdframed}

% --- begin definitions of admonition environments ---

% --- end of definitions of admonition environments ---

% prevent orhpans and widows
\clubpenalty = 10000
\widowpenalty = 10000

% --- end of standard preamble for documents ---


% insert custom LaTeX commands...

\raggedbottom
\makeindex

%-------------------- end preamble ----------------------

\begin{document}



% ------------------- main content ----------------------

% Slides for PHY981


% ----------------- title -------------------------

\thispagestyle{empty}

\begin{center}
{\LARGE\bf
\begin{spacing}{1.25}
Quantum numbers and nuclear forces
\end{spacing}
}
\end{center}

% ----------------- author(s) -------------------------

\begin{center}
{\bf Morten Hjorth-Jensen, National Superconducting Cyclotron Laboratory and Department of Physics and Astronomy, Michigan State University, East Lansing, MI 48824, USA {\&} Department of Physics, University of Oslo, Oslo, Norway${}^{}$} \\ [0mm]
\end{center}

    \begin{center}
% List of all institutions:
\end{center}
    
% ----------------- end author(s) -------------------------

\begin{center} % date
Spring 2015
\end{center}

\vspace{1cm}


% !split
\subsection*{Quantum numbers and nuclear forces}

% --- begin paragraph admon ---
\paragraph{Outline.}
\begin{itemize}
\item Discussion of single-particle and two-particle quantum numbers, uncoupled and coupled schemes

\item Discussion of nuclear forces and models thereof (this material will not be covered in depth this spring)
\end{itemize}

\noindent
For quantum numbers, chapter 1 on angular momentum and chapter 5 of Suhonen.
See also chapters 5, 12 and 13 of Alex Brown. For models of the nuclear forces, see present lectures and pdf files.
For a good discussion of isospin, see Alex Brown's lecture notes chapter 12, 13 and 19.
% --- end paragraph admon ---





% !split
\subsection*{Single-particle and two-particle quantum numbers}

% --- begin paragraph admon ---
\paragraph{}
In order to understand the basics of the nucleon-nucleon interaction, we need to define the relevant quantum numbers and how we build up a single-particle state and a two-body state. 

\begin{itemize}
\item For the single-particle states, due to the fact that we have the spin-orbit force, the quantum numbers for the projection of orbital momentum $l$, that is $m_l$, and for spin $s$, that is $m_s$, are no longer so-called good quantum numbers. The total angular momentum $j$ and its projection $m_j$ are then  so-called \emph{good quantum numbers}.

\item This means that the operator $\hat{J}^2$ does not commute with $\hat{L}_z$  or $\hat{S}_z$.  

\item We also start normally with single-particle state functions defined using say the harmonic oscillator. For these functions, we have no explicit dependence on $j$. How can we introduce single-particle wave functions which have $j$ and its projection $m_j$ as quantum numbers? 
\end{itemize}

\noindent
% --- end paragraph admon ---




% !split
\subsection*{Single-particle and two-particle quantum numbers, brief review on angular momenta etc}


% --- begin paragraph admon ---
\paragraph{}
We have that the operators for the orbital momentum are given by
\[
L_x=-i\hbar(y\frac{\partial }{\partial z}-z\frac{\partial }{\partial y})=
yp_z-zp_y,
\]
\[
L_y=-i\hbar(z\frac{\partial }{\partial x}-x\frac{\partial }{\partial z})= zp_x-xp_z,
\]
\[
L_z=-i\hbar(x\frac{\partial }{\partial y}-y\frac{\partial }{\partial x})=xp_y-yp_x.
\]
% --- end paragraph admon ---



% !split
\subsection*{Single-particle and two-particle quantum numbers, brief review on angular momenta etc}

% --- begin paragraph admon ---
\paragraph{}
Since we have a spin orbit force which is strong, it is easy to show that 
the total angular momentum operator
\[
   \hat{J}=\hat{L}+\hat{S}
\]
does not commute with $\hat{L}_z$ and $\hat{S}_z$. To see this, we calculate for example
\begin{eqnarray} 
   [\hat{L}_z,\hat{J}^2]&=&[\hat{L}_z,(\hat{L}+\hat{S})^2] \\ \nonumber
   &=&[\hat{L}_z,\hat{L}^2+\hat{S}^2+2\hat{L}\hat{S}]\\ \nonumber 
   &=& [\hat{L}_z,\hat{L}\hat{S}]=[\hat{L}_z,\hat{L}_x\hat{S}_x+\hat{L}_y\hat{S}_y+\hat{L}_z\hat{S}_z]\ne 0, 
\end{eqnarray}
since we have that $[\hat{L}_z,\hat{L}_x]=i\hbar\hat{L}_y$ and $[\hat{L}_z,\hat{L}_y]=i\hbar\hat{L}_x$.
% --- end paragraph admon ---



% !split
\subsection*{Single-particle and two-particle quantum numbers, brief review on angular momenta etc}

% --- begin paragraph admon ---
\paragraph{}
We have also
\[
   |\hat{J}|=\hbar\sqrt{J(J+1)},
\]
with the the following degeneracy
\[
   M_J=-J, -J+1, \dots, J-1, J.
\]
With a given value of  $L$ and $S$ we can then determine the possible values of 
 $J$ by studying the $z$ component of  $\hat{J}$. 
It is given by
\[
\hat{J}_z=\hat{L}_z+\hat{S}_z.
\]
The operators $\hat{L}_z$ and $\hat{S}_z$ have the quantum numbers
$L_z=M_L\hbar$ and $S_z=M_S\hbar$, respectively, meaning that
\[
   M_J\hbar=M_L\hbar +M_S\hbar,
\]
or
\[
   M_J=M_L +M_S.
\]
Since the max value of  $M_L$ is $L$ and for  $M_S$ is $S$
we obtain
\[
   (M_J)_{\mathrm{maks}}=L+S.
\]
% --- end paragraph admon ---



% !split
\subsection*{Single-particle and two-particle quantum numbers, brief review on angular momenta etc}

% --- begin paragraph admon ---
\paragraph{}
For nucleons we have that the maximum value of $M_S=m_s=1/2$, yielding
\[
   (m_j)_{\mathrm{max}}=l+\frac{1}{2}.
\]
Using this and the fact that the maximum value of  $M_J=m_j$ is $j$ we have
\[
   j=l+\frac{1}{2}, l-\frac{1}{2}, l-\frac{3}{2}, l-\frac{5}{2}, \dots 
\]
To decide where this series terminates, we use the vector inequality
\[
   |\hat{L}+\hat{S}| \ge \left| |\hat{L}|-|\hat{S}|\right|.
\]
% --- end paragraph admon ---




% !split
\subsection*{Single-particle and two-particle quantum numbers, brief review on angular momenta etc}

% --- begin paragraph admon ---
\paragraph{}
Using $\hat{J}=\hat{L}+\hat{S}$ we get 
\[
   |\hat{J}| \ge |\hat{L}|-|\hat{S}|,
\]
or
\[
   |\hat{J}|=\hbar\sqrt{J(J+1)}\ge |\hbar\sqrt{L(L+1)}-
   \hbar\sqrt{S(S+1)}|.
\]
% --- end paragraph admon ---





% !split
\subsection*{Single-particle and two-particle quantum numbers, brief review on angular momenta etc}

% --- begin paragraph admon ---
\paragraph{}
If we limit ourselves to nucleons only with $s=1/2$ we find that
\[
   |\hat{J}|=\hbar\sqrt{j(j+1)}\ge |\hbar\sqrt{l(l+1)}-
   \hbar\sqrt{\frac{1}{2}(\frac{1}{2}+1)}|.
\]
It is then easy to show that for nucleons there are only two possible values of
$j$ which satisfy the inequality, namely
\[
   j=l+\frac{1}{2}\hspace{0.1cm} \mathrm{or} \hspace{0.1cm}j=l-\frac{1}{2},
\]
and with $l=0$ we get 
\[
   j=\frac{1}{2}.
\]
% --- end paragraph admon ---




% !split
\subsection*{Single-particle and two-particle quantum numbers, brief review on angular momenta etc}

% --- begin paragraph admon ---
\paragraph{}
Let us study some selected examples. We need also to keep in mind that parity is conserved.
The strong and electromagnetic Hamiltonians conserve parity. Thus the eigenstates can be
broken down into two classes of states labeled by their parity $\pi= +1$ or $\pi=-1$.
The nuclear interactions do not mix states with different parity.

For nuclear structure the total parity originates
from the intrinsic parity of the nucleon which is $\pi_{\mathrm{intrinsic}}=+1$ 
and the parities associated with
the orbital angular momenta $\pi_l=(-1)^l$ . The total parity is the product over all nucleons
$\pi = \prod_i \pi_{\mathrm{intrinsic}}(i)\pi_l(i) = \prod_i (-1)^{l_i}$

The basis states we deal with are constructed so that they conserve parity and have thus a definite parity. 

Note that we do have parity violating processes, more on this later although our focus will be mainly on non-parity viloating processes
% --- end paragraph admon ---




% !split
\subsection*{Single-particle and two-particle quantum numbers}

% --- begin paragraph admon ---
\paragraph{}

Consider now the single-particle orbits of the $1s0d$ shell. 
For a $0d$ state we have the quantum numbers $l=2$, $m_l=-2,-1,0,1,2$, $s+1/2$, $m_s=\pm 1/2$,
$n=0$ (the number of nodes of the wave function).   This means that we have positive parity and
\[
j=\frac{3}{2}=l-s\hspace{1cm} m_j=-\frac{3}{2},-\frac{1}{2},\frac{1}{2},\frac{3}{2}.
\]
and
\[
j=\frac{5}{2}=l+s\hspace{1cm} m_j=-\frac{5}{2},-\frac{3}{2},-\frac{1}{2},\frac{1}{2},\frac{3}{2},\frac{5}{2}.
\]
% --- end paragraph admon ---




% !split
\subsection*{Single-particle and two-particle quantum numbers}

% --- begin paragraph admon ---
\paragraph{}
Our single-particle wave functions, if we use the harmonic oscillator, do however not contain the quantum numbers $j$ and $m_j$.
Normally what we have is an eigenfunction for the one-body problem defined as
\[
\phi_{nlm_lsm_s}(r,\theta,\phi)=R_{nl}(r)Y_{lm_l}(\theta,\phi)\xi_{sm_s},
\]
where we have used spherical coordinates (with a spherically symmetric potential) and the spherical harmonics
\[
    Y_{lm_l}(\theta,\phi)=P(\theta)F(\phi)=\sqrt{\frac{(2l+1)(l-m_l)!}{4\pi (l+m_l)!}}
                      P_l^{m_l}(cos(\theta))\exp{(im_l\phi)},
\]
with $P_l^{m_l}$ being the so-called associated Legendre polynomials.
% --- end paragraph admon ---





% !split
\subsection*{Single-particle and two-particle quantum numbers}

% --- begin paragraph admon ---
\paragraph{}
Examples are
\[
   Y_{00}=\sqrt{\frac{1}{4\pi}},
\]
for $l=m_l=0$, 
\[
   Y_{10}=\sqrt{\frac{3}{4\pi}}cos(\theta),
\]
for $l=1$ and $m_l=0$, 
\[
   Y_{1\pm 1}=\sqrt{\frac{3}{8\pi}}sin(\theta)exp(\pm i\phi),
\]
for  $l=1$ and $m_l=\pm 1$, 
\[
   Y_{20}=\sqrt{\frac{5}{16\pi}}(3cos^2(\theta)-1)
\]
for $l=2$ and $m_l=0$ etc.
% --- end paragraph admon ---






% !split
\subsection*{Single-particle and two-particle quantum numbers}

% --- begin paragraph admon ---
\paragraph{}
How can we get a function in terms of $j$ and $m_j$?
Define now
\[
\phi_{nlm_lsm_s}(r,\theta,\phi)=R_{nl}(r)Y_{lm_l}(\theta,\phi)\xi_{sm_s},
\]
and
\[
\psi_{njm_j;lm_lsm_s}(r,\theta,\phi),
\]
as the state with quantum numbers $jm_j$.
Operating with 
\[
   \hat{j}^2=(\hat{l}+\hat{s})^2=\hat{l}^2+\hat{s}^2+2\hat{l}_z\hat{s}_z+\hat{l}_+\hat{s}_{-}+\hat{l}_{-}\hat{s}_{+},
\]
on the latter state we will obtain admixtures from possible $\phi_{nlm_lsm_s}(r,\theta,\phi)$ states.
% --- end paragraph admon ---




% !split
\subsection*{Single-particle and two-particle quantum numbers}

% --- begin paragraph admon ---
\paragraph{}
To see this, we consider the following example and fix
\[
j=\frac{3}{2}=l-s\hspace{1cm} m_j=\frac{3}{2}.
\]
and
\[
j=\frac{5}{2}=l+s\hspace{1cm} m_j=\frac{3}{2}.
\]
It means we can have, with $l=2$ and $s=1/2$ being fixed, in order to have $m_j=3/2$ either $m_l=1$ and $m_s=1/2$ or
$m_l=2$ and $m_s=-1/2$. The two states    
\[
\psi_{n=0j=5/2m_j=3/2;l=2s=1/2}
\]
and
\[
\psi_{n=0j=3/2m_j=3/2;l=2s=1/2}
\]
will have admixtures from $\phi_{n=0l=2m_l=2s=1/2m_s=-1/2}$ and $\phi_{n=0l=2m_l=1s=1/2m_s=1/2}$. 
How do we find these admixtures? Note that we don't specify the values of $m_l$ and $m_s$ 
in the functions $\psi$ since    
$\hat{j}^2$ does not commute with $\hat{L}_z$ and $\hat{S}_z$.
% --- end paragraph admon ---



% !split
\subsection*{Single-particle and two-particle quantum numbers}

% --- begin paragraph admon ---
\paragraph{}
We operate with 
\[
   \hat{j}^2=(\hat{l}+\hat{s})^2=\hat{l}^2+\hat{s}^2+2\hat{l}_z\hat{s}_z+\hat{l}_+\hat{s}_{-}+\hat{l}_{-}\hat{s}_{+}
\]
on the two $jm_j$ states, that is
\[
\hat{j}^2\psi_{n=0j=5/2m_j=3/2;l=2s=1/2}= \alpha\hbar^2[l(l+1)+\frac{3}{4}+2m_lm_s]\phi_{n=0l=2m_l=2s=1/2m_s=-1/2}+
\]
\[
\beta\hbar^2\sqrt{l(l+1)-m_l(m_l-1)}\phi_{n=0l=2m_l=1s=1/2m_s=1/2},
\]
and
\[
\hat{j}^2\psi_{n=0j=3/2m_j=3/2;l=2s=1/2}= \alpha\hbar^2[l(l+1)+\frac{3}{4}+2m_lm_s]+ \phi_{n=0l=2m_l=1s=1/2m_s=1/2}+
\]
\[
\beta\hbar^2\sqrt{l(l+1)-m_l(m_l+1)}\phi_{n=0l=2m_l=2s=1/2m_s=-1/2}.
\]
% --- end paragraph admon ---



% !split
\subsection*{Single-particle and two-particle quantum numbers}

% --- begin paragraph admon ---
\paragraph{}
This means that the eigenvectors $\phi_{n=0l=2m_l=2s=1/2m_s=-1/2}$ etc are not eigenvectors of $\hat{j}^2$. The above problems gives a $2\times2$ matrix that mixes the vectors $\psi_{n=0j=5/2m_j3/2;l=2m_ls=1/2m_s}$ and $\psi_{n=0j=3/2m_j3/2;l=2m_ls=1/2m_s}$ with the states  $\phi_{n=0l=2m_l=2s=1/2m_s=-1/2}$ and
$\phi_{n=0l=2m_l=1s=1/2m_s=1/2}$. The unknown coefficients $\alpha$ and $\beta$ results from eigenvectors of this matrix. That is, inserting all values $m_l,l,m_s,s$ we obtain the matrix 
\[
\left[ \begin{array} {cc} 19/4 & 2 \\ 2 & 31/4 \end{array} \right]\]
whose eigenvectors are the columns of
\[
\left[ \begin{array} {cc} 2/\sqrt{5} &1/\sqrt{5}  \\ 1/\sqrt{5} & -2/\sqrt{5} \end{array}\right]\]  
These numbers define the so-called Clebsch-Gordan coupling coefficients  (the overlaps between the two basis sets). We can thus write
\[
\psi_{njm_j;ls}=\sum_{m_lm_s}\langle lm_lsm_s|jm_j\rangle\phi_{nlm_lsm_s},
\]
where the coefficients $\langle lm_lsm_s|jm_j\rangle$ are the so-called Clebsch-Gordan coeffficients.
% --- end paragraph admon ---



% !split
\subsection*{Clebsch-Gordan coefficients}

% --- begin paragraph admon ---
\paragraph{}
The Clebsch-Gordan coeffficients $\langle lm_lsm_s|jm_j\rangle$ have some interesting properties for us, like the following  orthogonality relations
\[
\sum_{m_1m_2}\langle j_1m_1j_2m_2|JM\rangle\langle j_1m_1j_2m_2|J'M'\rangle=\delta_{J,J'}\delta_{M,M'},
\]
\[
\sum_{JM}\langle j_1m_1j_2m_2|JM\rangle\langle j_1m_1'j_2m_2'|JM\rangle=\delta_{m_1,m_1'}\delta_{m_2,m_2'},
\]
\[
\langle j_1m_1j_2m_2|JM\rangle=(-1)^{j_1+j_2-J}\langle j_2m_2j_1m_1|JM\rangle,
\]
and many others. The latter will turn extremely useful when we are going to define two-body states and interactions in a coupled basis.
% --- end paragraph admon ---



% !split
\subsection*{Quantum numbers and the Schroeodinger equation in relative and CM coordinates}

% --- begin paragraph admon ---
\paragraph{}
Summing up, for 
for the single-particle case, we have the following eigenfunctions 
\[
\psi_{njm_j;ls}=\sum_{m_lm_s}\langle lm_lsm_s|jm_j\rangle\phi_{nlm_lsm_s},
\]
where the coefficients $\langle lm_lsm_s|jm_j\rangle$ are the so-called Clebsch-Gordan coeffficients.
The relevant quantum numbers are $n$ (related to the principal quantum number and the number of nodes of the wave function) and 
\[
   \hat{j}^2\psi_{njm_j;ls}=\hbar^2j(j+1)\psi_{njm_j;ls},
\]
\[
   \hat{j}_z\psi_{njm_j;ls}=\hbar m_j\psi_{njm_j;ls},
\]
\[
   \hat{l}^2\psi_{njm_j;ls}=\hbar^2l(l+1)\psi_{njm_j;ls},
\]
\[
   \hat{s}^2\psi_{njm_j;ls}=\hbar^2s(s+1)\psi_{njm_j;ls},
\]
but $s_z$ and $l_z$ do not result in good quantum numbers in a basis where we
use the angular momentum $j$.
% --- end paragraph admon ---



% !split
\subsection*{Quantum numbers and the Schroedinger equation in relative and CM coordinates}

% --- begin paragraph admon ---
\paragraph{}
For a two-body state where we couple two angular momenta $j_1$ and $j_2$ to a final
angular momentum $J$ with projection $M_J$, we can define a similar transformation in terms
of the Clebsch-Gordan coeffficients
\[
\psi_{(j_1j_2)JM_J}=\sum_{m_{j_1}m_{j_2}}\langle j_1m_{j_1}j_2m_{j_2}|JM_J\rangle\psi_{n_1j_1m_{j_1};l_1s_1}\psi_{n_2j_2m_{j_2};l_2s_2}.
\]
We will write these functions in a more compact form hereafter, namely,
\[
|(j_1j_2)JM_J\rangle=\psi_{(j_1j_2)JM_J},
\]
and
\[
|j_im_{j_i}\rangle=\psi_{n_ij_im_{j_i};l_is_i},
\]
where we have skipped the explicit reference to $l$, $s$ and $n$. The spin of a nucleon is always $1/2$ while the value of $l$ can be deduced from the parity of the state.
It is thus normal to label a state with a given total angular momentum as 
$j^{\pi}$, where $\pi=\pm 1$.
% --- end paragraph admon ---



% !split
\subsection*{Quantum numbers and the Schroedinger equation in relative and CM coordinates}

% --- begin paragraph admon ---
\paragraph{}
Our two-body state can thus be written as 
\[
|(j_1j_2)JM_J\rangle=\sum_{m_{j_1}m_{j_2}}\langle j_1m_{j_1}j_2m_{j_2}|JM_J\rangle|j_1m_{j_1}\rangle|j_2m_{j_2}\rangle.
\]
Due to the coupling order of the Clebsch-Gordan coefficient it reads as 
$j_1$ coupled to $j_2$ to yield a final angular momentum $J$. If we invert the order of coupling we would have
\[
|(j_2j_1)JM_J\rangle=\sum_{m_{j_1}m_{j_2}}\langle j_2m_{j_2}j_1m_{j_1}|JM_J\rangle|j_1m_{j_1}\rangle|j_2m_{j_2}\rangle,
\]
and due to the symmetry properties of the Clebsch-Gordan coefficient we have
\[
|(j_2j_1)JM_J\rangle=(-1)^{j_1+j_2-J}\sum_{m_{j_1}m_{j_2}}\langle j_1m_{j_1}j_2m_{j_2}|JM_J\rangle|j_1m_{j_1}\rangle|j_2m_{j_2}\rangle=(-1)^{j_1+j_2-J}|(j_1j_2)JM_J\rangle.
\]
We call the basis $|(j_1j_2)JM_J\rangle$ for the \textbf{coupled basis}, or just $j$-coupled basis/scheme. The basis formed by the simple product of single-particle eigenstates 
$|j_1m_{j_1}\rangle|j_2m_{j_2}\rangle$ is called the \textbf{uncoupled-basis}, or just the $m$-scheme basis.
% --- end paragraph admon ---



% !split
\subsection*{Quantum numbers}

% --- begin paragraph admon ---
\paragraph{}
We have thus the coupled basis 
\[
|(j_1j_2)JM_J\rangle=\sum_{m_{j_1}m_{j_2}}\langle j_1m_{j_1}j_2m_{j_2}|JM_J\rangle|j_1m_{j_1}\rangle|j_2m_{j_2}\rangle.
\]
and the uncoupled basis 
\[
|j_1m_{j_1}\rangle|j_2m_{j_2}\rangle.
\]
The latter can easily be generalized to many single-particle states whereas the first 
needs specific coupling coefficients and definitions of coupling orders. 
The $m$-scheme basis is easy to implement numerically and is used in most standard shell-model codes. 
Our coupled basis obeys also the following relations
\[
   \hat{J}^2|(j_1j_2)JM_J\rangle=\hbar^2J(J+1)|(j_1j_2)JM_J\rangle
\]
\[
   \hat{J}_z|(j_1j_2)JM_J\rangle=\hbar M_J|(j_1j_2)JM_J\rangle,
\]
% --- end paragraph admon ---




% !split
\subsection*{Components of the force and isospin}

% --- begin paragraph admon ---
\paragraph{}
 The nuclear forces are almost charge independent. If we assume they are, 
we can introduce a new quantum number which is conserved. For nucleons only, that is a proton and neutron, we can limit ourselves
to two possible values which allow us to distinguish between the two particles. If we assign an isospin value of $\tau=1/2$ for protons
and neutrons (they belong to an isospin doublet, in the same way as we discussed the spin $1/2$ multiplet), we can define 
the neutron to have isospin projection $\tau_z=+1/2$ and a proton to have $\tau_z=-1/2$. These assignements are the standard choices in low-energy nuclear physics.
% --- end paragraph admon ---



% !split
\subsection*{Isospin}

% --- begin paragraph admon ---
\paragraph{}
This leads to the introduction of an additional quantum number called isospin.
We can define a single-nucleon
state function in terms of the quantum numbers $n$, $j$, $m_j$, $l$, $s$, $\tau$ and $\tau_z$. Using our definitions in terms of an uncoupled basis, we had 
\[
\psi_{njm_j;ls}=\sum_{m_lm_s}\langle lm_lsm_s|jm_j\rangle\phi_{nlm_lsm_s},
\]
which we can now extend to
\[
\psi_{njm_j;ls}\xi_{\tau\tau_z}=\sum_{m_lm_s}\langle lm_lsm_s|jm_j\rangle\phi_{nlm_lsm_s}\xi_{\tau\tau_z},
\]
with the isospin spinors defined as 
\[
\xi_{\tau=1/2\tau_z=+1/2}=\left(\begin{array}{c} 1  \\ 0\end{array}\right),
\]
and
\[
\xi_{\tau=1/2\tau_z=-1/2}=\left(\begin{array}{c} 0  \\ 1\end{array}\right).
\]
We can then define the proton state function as 
\[
\psi^p(\mathbf{r})  =\psi_{njm_j;ls}(\mathbf{r})\left(\begin{array}{c} 0  \\ 1\end{array}\right), 
\]
and similarly for neutrons as
\[
\psi^n(\mathbf{r})  =\psi_{njm_j;ls}(\mathbf{r})\left(\begin{array}{c} 1  \\ 0\end{array}\right). 
\]
% --- end paragraph admon ---




% !split
\subsection*{Isospin}

% --- begin paragraph admon ---
\paragraph{}
We can in turn define the isospin Pauli matrices (in the same as we define the spin matrices) as
\[
\hat{\tau}_x =\left(\begin{array}{cc} 0 & 1 \\ 1 & 0 \end{array}\right),
\]
\[
\hat{\tau}_y =\left(\begin{array}{cc} 0 & -\imath \\ \imath & 0 \end{array}\right),
\]
and
\[
\hat{\tau}_z =\left(\begin{array}{cc} 1 & 0 \\ 0 & -1 \end{array}\right),
\]
and operating with $\hat{\tau}_z$ on the proton state function we have
\[
\hat{\tau}_z\psi^p(\mathbf{r})=-\frac{1}{2}\psi^p(\mathbf{r}),
\]
and for neutrons we have
\[
\hat{\tau}\psi^n(\mathbf{r})=\frac{1}{2}\psi^n(\mathbf{r}).
\]
% --- end paragraph admon ---



% !split
\subsection*{Isospin}

% --- begin paragraph admon ---
\paragraph{}
We can now define the so-called charge operator as 
\[
\frac{\hat{Q}}{e} = \frac{1}{2}\left(1-\hat{\tau}_z\right)=\begin{Bmatrix} 0 & 0 \\ 0 & 1 \end{Bmatrix},
\]
which results in 
\[
\frac{\hat{Q}}{e}\psi^p(\mathbf{r})=\psi^p(\mathbf{r}),
\]
and
\[
\frac{\hat{Q}}{e}\psi^n(\mathbf{r})=0,
\]
as it should be.
% --- end paragraph admon ---



% !split
\subsection*{Isospin}

% --- begin paragraph admon ---
\paragraph{}
The total isospin is defined as
\[
\hat{T}=\sum_{i=1}^A\hat{\tau}_i,
\]
and its corresponding isospin projection as
\[
\hat{T}_z=\sum_{i=1}^A\hat{\tau}_{z_i},
\]
with eigenvalues $T(T+1)$ for $\hat{T}$ and $1/2(N-Z)$ for $\hat{T}_z$, where $N$ is the number of neutrons and $Z$ the number of protons. 

If charge is conserved, the Hamiltonian $\hat{H}$ commutes with $\hat{T}_z$ and all members of a given isospin multiplet
(that is the same value of $T$) have the same energy and there is no $T_z$ dependence and we say that $\hat{H}$ is a scalar in isospin space.
% --- end paragraph admon ---




% !split
\subsection*{Two-body matrix elements}

% --- begin paragraph admon ---
\paragraph{}
Till now we have not said anything about the explicit calculation of two-body matrix elements. It is time to amend this deficiency.
We have till now seen the following definitions of a two-body matrix elements. In $m$-scheme
with quantum numbers $p=j_pm_p$ etc we have a two-body state defined as
\[
|(pq)M\rangle  = a^{\dagger}_pa^{\dagger}_q|\Phi_0\rangle,
\]
where $|\Phi_0\rangle$ is a chosen reference state, say for example the Slater determinant which approximates ${}^{16}\mbox{O}$ with the $0s$ and the $0p$ shells being filled, and $M=m_p+m_q$. Recall that we label single-particle states above the Fermi level as $abcd\dots$ and states below the Fermi level for $ijkl\dots$.  
In case of two-particles in the single-particle states $a$ and $b$ outside ${}^{16}\mbox{O}$ as a closed shell core, say ${}^{18}\mbox{O}$, 
we would write the representation of the Slater determinant as
\[
|^{18}\mathrm{O}\rangle =|(ab)M\rangle  = a^{\dagger}_aa^{\dagger}_b|^{16}\mathrm{O}\rangle=|\Phi^{ab}\rangle.
\]
In case of two-particles removed from say ${}^{16}\mbox{O}$, for example two neutrons in the single-particle states $i$ and $j$, we would write this as
\[
|^{14}\mathrm{O}\rangle =|(ij)M\rangle  = a_ja_i|^{16}\mathrm{O}\rangle=|\Phi_{ij}\rangle.
\]
% --- end paragraph admon ---



% !split
\subsection*{Two-body matrix elementsin $m$-scheme}

% --- begin paragraph admon ---
\paragraph{}
For a one-hole-one-particle state we have
\[
|^{16}\mathrm{O}\rangle_{1p1h} =|(ai)M\rangle  = a_a^{\dagger}a_i|^{16}\mathrm{O}\rangle=|\Phi_{i}^a\rangle,
\]
and finally for a two-particle-two-hole state we 
\[
|^{16}\mathrm{O}\rangle_{2p2h} =|(abij)M\rangle  = a_a^{\dagger}a_b^{\dagger}a_ja_i|^{16}\mathrm{O}\rangle=|\Phi_{ij}^{ab}\rangle.
\]
% --- end paragraph admon ---



% !split
\subsection*{Two-body matrix elements in $m$-scheme}

% --- begin paragraph admon ---
\paragraph{}
Let us go back to the case of two-particles in the single-particle states $a$ and $b$ outside ${}^{16}\mbox{O}$ as a closed shell core, say ${}^{18}\mbox{O}$.
The representation of the Slater determinant is 
\[
|^{18}\mathrm{O}\rangle =|(ab)M\rangle  = a^{\dagger}_aa^{\dagger}_b|^{16}\mathrm{O}\rangle=|\Phi^{ab}\rangle.
\]
The anti-symmetrized matrix element is detailed as 
\[
\langle (ab) M | \hat{V} | (cd) M \rangle = \langle (j_am_aj_bm_b)M=m_a+m_b |  \hat{V} | (j_cm_cj_dm_d)M=m_a+m_b \rangle,
\]
and note that anti-symmetrization means 
\[
\langle (ab) M | \hat{V} | (cd) M \rangle =-\langle (ba) M | \hat{V} | (cd) M \rangle =\langle (ba) M | \hat{V} | (dc) M \rangle,
\]
\[
\langle (ab) M | \hat{V} | (cd) M \rangle =-\langle (ab) M | \hat{V} | (dc) M \rangle. 
\]
This matrix element is the expectation value of 
\[
\langle ^{16}\mathrm{O}|a_ba_a\frac{1}{4}\sum_{pqrs}\langle (pq) M | \hat{V} | (rs) M' \rangle a^{\dagger}_pa^{\dagger}_qa_sa_r a^{\dagger}_ca^{\dagger}_c|^{16}\mathrm{O}\rangle.
\]
% --- end paragraph admon ---




% !split
\subsection*{Two-body matrix elements in $J$-scheme}

% --- begin paragraph admon ---
\paragraph{}
We have also defined matrix elements in the coupled basis, the so-called $J$-coupled scheme.
In this case the two-body wave function for two neutrons outside ${}^{16}\mbox{O}$ is written as 
\[
|^{18}\mathrm{O}\rangle_J =|(ab)JM\rangle  = \left\{a^{\dagger}_aa^{\dagger}_b\right\}^J_M|^{16}\mathrm{O}\rangle=N_{ab}\sum_{m_am_b}\langle j_am_aj_bm_b|JM\rangle|\Phi^{ab}\rangle, 
\]
with
\[
|\Phi^{ab}\rangle=a^{\dagger}_aa^{\dagger}_b|^{16}\mathrm{O}\rangle.
\]
We have now an explicit coupling order, where the angular momentum $j_a$ is coupled to the angular momentum $j_b$ to yield a final two-body angular momentum $J$. 
The normalization factor (to be derived below) is
\[
N_{ab}=\frac{\sqrt{1+\delta_{ab}\times (-1)^J}}{1+\delta_{ab}}.
\]
% --- end paragraph admon ---



% !split
\subsection*{Two-body matrix elements in $J$-scheme}

% --- begin paragraph admon ---
\paragraph{}
The implementation of the Pauli principle looks different in the $J$-scheme compared with the $m$-scheme. In the latter, no two fermions or more can have the same set of quantum numbers. In the $J$-scheme, when we write a state with the shorthand 
\[
|^{18}\mathrm{O}\rangle_J =|(ab)JM\rangle,
\]
we do refer to the angular momenta only. This means that another way of writing the last state is
\[
|^{18}\mathrm{O}\rangle_J =|(j_aj_b)JM\rangle.
\]
We will use this notation throughout when we refer to a two-body state in $J$-scheme. The Kronecker $\delta$ function in the normalization factor 
refers thus to the values of $j_a$ and $j_b$. If two identical particles are in a state with the same $j$-value, then only even values of the total angular momentum apply.
% --- end paragraph admon ---




% !split
\subsection*{Two-body matrix elements in $J$-scheme}

% --- begin paragraph admon ---
\paragraph{}
Note also that, using the anti-commuting properties of the creation operators, we obtain
\[
N_{ab}\sum_{m_am_b}\langle j_am_aj_bm_b|JM>|\Phi^{ab}\rangle=-N_{ab}\sum_{m_am_b}\langle j_am_aj_bm_b|JM\rangle|\Phi^{ba}\rangle.
\]
Furthermore, using the property of the Clebsch-Gordan coefficient
\[
\langle j_am_aj_bm_b|JM>=(-1)^{j_a+j_b-J}\langle j_bm_bj_am_a|JM\rangle,
\]
which can be used to show that
\[
|(j_bj_a)JM\rangle  = \left\{a^{\dagger}_ba^{\dagger}_a\right\}^J_M|^{16}\mathrm{O}\rangle=N_{ab}\sum_{m_am_b}\langle j_bm_bj_am_a|JM\rangle|\Phi^{ba}\rangle, 
\]
is equal to 
\[
|(j_bj_a)JM\rangle=(-1)^{j_a+j_b-J+1}|(j_aj_b)JM\rangle.
\]
This relation is important since we will need it when using anti-symmetrized matrix elements in $J$-scheme.
% --- end paragraph admon ---



% !split
\subsection*{Two-body matrix elements in $J$-scheme}

% --- begin paragraph admon ---
\paragraph{}
The two-body matrix element is a scalar and since it obeys rotational symmetry, it is diagonal in $J$, 
meaning that the corresponding matrix element in $J$-scheme is 
\[
\langle (j_aj_b) JM | \hat{V} | (j_cj_d) JM \rangle = N_{ab}N_{cd}\sum_{m_am_b m_cm_d}\langle j_am_aj_bm_b|JM\rangle
\]
\[\times \langle j_cm_cj_dm_d|JM\rangle\langle (j_am_aj_bm_b)M |  \hat{V} | (j_cm_cj_dm_d)M \rangle,
\]
and note that of the four $m$-values in the above sum, only three are independent due to the constraint $m_a+m_b=M=m_c+m_d$.
Since
\[
|(j_bj_a)JM\rangle=(-1)^{j_a+j_b-J+1}|(j_aj_b)JM\rangle,
\]
the anti-symmetrized matrix elements need now to obey the following relations
\[
\langle (j_aj_b) JM | \hat{V} | (j_cj_d) JM \rangle = (-1)^{j_a+j_b-J+1}\langle (j_bj_a) JM | \hat{V} | (j_cj_d) JM \rangle,
\]
\[
\langle (j_aj_b) JM | \hat{V} | (j_cj_d) JM \rangle = (-1)^{j_c+j_d-J+1}\langle (j_aj_b) JM | \hat{V} | (j_dj_c) JM \rangle,
\]
\[
\langle (j_aj_b) JM | \hat{V} | (j_cj_d) JM \rangle = (-1)^{j_a+j_b+j_c+j_d}\langle (j_bj_a) JM | \hat{V} | (j_dj_c) JM \rangle=\langle (j_bj_a) JM | \hat{V} | (j_dj_c) JM \rangle,
\]
where the last relations follows from the fact that $J$ is an integer and $2J$ is always an even number.
% --- end paragraph admon ---



% !split
\subsection*{Two-body matrix elements in $J$-scheme}

% --- begin paragraph admon ---
\paragraph{}
Using the orthogonality properties of the Clebsch-Gordan coefficients,
\[
\sum_{m_am_b}\langle j_am_aj_bm_b|JM\rangle\langle j_am_aj_bm_b|J'M'\rangle=\delta_{JJ'}\delta_{MM'},
\]
and
\[
\sum_{JM}\langle j_am_aj_bm_b|JM\rangle\langle j_am_a'j_bm_b'|JM\rangle=\delta_{m_am_a'}\delta_{m_bm_b'},
\]
we can also express the two-body matrix element in $m$-scheme in terms of that in $J$-scheme, that is, if we multiply with 
\[
\sum_{JMJ'M'}\langle j_am_a'j_bm_b'|JM\rangle\langle j_cm_c'j_dm_d'|J'M'\rangle
\]
from left in
\[
\langle (j_a j_b) JM | \hat{V} | (j_c j_d) JM \rangle = N_{ab}N_{cd}\sum_{m_a m_b m_c m_d}\langle j_am_aj_bm_b|JM\rangle\langle j_cm_cj_dm_d|JM\rangle
\]
\[
\times \langle (j_am_aj_bm_b)M|  \hat{V} | (j_cm_cj_dm_d)M\rangle,
\]
we obtain
% --- end paragraph admon ---



% !split
\subsection*{Two-body matrix elements in $J$-scheme}

% --- begin paragraph admon ---
\paragraph{}
we obtain
\[
\langle (j_am_aj_bm_b)M |  \hat{V} | (j_cm_cj_dm_d)M\rangle=\frac{1}{N_{ab}N_{cd}}\sum_{JM}\langle j_am_aj_bm_b|JM\rangle\langle j_cm_cj_dm_d|JM\rangle
\]
\[
\times \langle (j_aj_b) JM | \hat{V} | (j_cj_d) JM \rangle.
\]
% --- end paragraph admon ---




% !split
\subsection*{Phenomenology of nuclear forces}

% --- begin paragraph admon ---
\paragraph{From Yukawa to Lattice QCD and Effective Field Theory.}

\begin{itemize}
\item Chadwick (1932) discovers the neutron and Heisenberg (1932) proposes the first Phenomenology (Isospin).  

\item Yukawa (1935) and his Meson Hypothesis       

\item Discovery of the pion in cosmic ray (1947) and in the Berkeley Cyclotron Lab (1948).

\item Nobelprize awarded to Yukawa (1949).  Rabi (1948) measures quadrupole moment of the deuteron.

\item Taketani, Nakamura, Sasaki (1951): 3 ranges.      One-Pion-Exchange (OPE): o.k.

\item Multi-pion exchanges: Problems!   Taketani, Machida, Onuma (1952);

\item \emph{Pion Theories} Brueckner, Watson (1953).
\end{itemize}

\noindent
% --- end paragraph admon ---



% !split
\subsection*{Phenomenology of nuclear forces}

% --- begin paragraph admon ---
\paragraph{From Yukawa to Lattice QCD and Effective Field Theory.}

\begin{itemize}
\item Many pions = multi-pion resonances: $\sigma(600)$,  $\rho(770)$,  $\omega(782)$ etc. One-Boson-Exchange Model.

\item Refined Meson Theories

\item Sophisticated models for two-pion exchange:
\begin{itemize}

      \item Paris Potential (Lacombe et al., Phys.~Rev.~C \textbf{21}, 861 (1980))

      \item Bonn potential (Machleidt et al., Phys.~Rep. \textbf{149}, 1 (1987))
\end{itemize}

\noindent
\end{itemize}

\noindent
*Quark cluster models. Begin of effective field theory studies.
% --- end paragraph admon ---



% !split
\subsection*{Phenomenology of nuclear forces}

% --- begin paragraph admon ---
\paragraph{From Yukawa to Lattice QCD and Effective Field Theory.}

\begin{itemize}
\item 1990's
\begin{itemize}

  \item 1993-2001: High-precision NN potentials: Nijmegen I, II, '93, Reid93 (Stoks et al.~1994), 

  \item Argonne V18 (Wiringa et al, 1995), CD-Bonn (Machleidt et al.~1996 and 2001. 

  \item Advances in effective field theory: Weinberg (1990); Ordonez, Ray, van Kolck and many more.

\end{itemize}

\noindent
\item 3rd Millenium
\begin{itemize}

  \item Another "pion theory"; but now right: constrained by chiral symmetry. Three-body and higher-body forces appear naturally at a given order of the chiral expansion. 
\end{itemize}

\noindent
\end{itemize}

\noindent
Nucleon-nucleon interaction from Lattice QCD, final confirmation of meson hypothesis of Yukawa?  See for example Ishii \emph{et al}, PRL 2007
% --- end paragraph admon ---




% !split
\subsection*{Phenomenology of nuclear forces}

% --- begin paragraph admon ---
\paragraph{Features of the Nucleon-Nucleon (NN) Force.}

The aim is to give you an overview over central features of the nucleon-nucleon interaction and how it is constructed, with both technical and theoretical approaches. 

\begin{itemize}
\item The existence of the deuteron with $J^{\pi}=1^+$ indicates that the force between protons and neutrons is attractive at least for the $^3S_1$ partial wave. Interference between Coulomb and nuclear scattering for the proton-proton partial wave $^1S_0$ shows that  the NN force is attractive at least for the $^1S_0$ partial wave. 

\item It has a short range and strong intermediate attraction.

\item Spin dependent, scattering lengths for triplet and singlet states are different,

\item Spin-orbit force. Observation of large polarizations of scattered nucleons perpendicular to the plane of scattering.
\end{itemize}

\noindent
% --- end paragraph admon ---



% !split
\subsection*{Phenomenology of nuclear forces}

% --- begin paragraph admon ---
\paragraph{}

\begin{itemize}
\item Strongly repulsive core. The $s$-wave phase shift becomes negative at $\approx 250$ MeV implying that the singlet $S$ has a hard core with range $0.4-0.5$ fm. 

\item Charge independence (almost). Two nucleons in a given two-body state always (almost) experience the same force. Modern interactions break charge and isospin symmetry lightly. That means that the pp, neutron-neutron and pn parts of the interaction will be different for the same quantum numbers. 

\item Non-central. There is a tensor force. First indications from the quadrupole moment of the deuteron pointing to an admixture in the ground state of both $l=2$ ($^3D_1$) and $l=0$ ($^3S_1$) orbital momenta.
\end{itemize}

\noindent
% --- end paragraph admon ---



% !split
\subsection*{Phenomenology of nuclear forces}

% --- begin paragraph admon ---
\paragraph{Short Range Evidence.}
Comparison of the binding energies of
${}^2\mbox{H}$ (deuteron), ${}^3\mbox{H}$ (triton), ${}^4\mbox{He}$ (alpha - particle) show that the nuclear force is of finite range ($1-2$ fm) and very strong within that range.

For nuclei with $A>4$, the energy saturates: Volume and binding energies of nuclei are proportional to the mass number $A$ (as we saw from exercise 1).

Nuclei are also bound. The average distance
between nucleons in nuclei is about $2$ fm which
must roughly correspond to the range of the
attractive part.
% --- end paragraph admon ---



% !split
\subsection*{Phenomenology of nuclear forces}

% --- begin paragraph admon ---
\paragraph{Charge Dependence.}

\begin{itemize}
 \item After correcting for the electromagnetic interaction, the forces between nucleons (pp, nn, or np) in the same state are almost the same.

 \item \emph{Almost the same}: Charge-independence is slightly broken.

 \item Equality between the pp and nn forces: Charge symmetry.

 \item Equality between pp/nn force and np force: Charge independence.

 \item Better notation: Isospin symmetry, invariance under rotations in isospin
\end{itemize}

\noindent
% --- end paragraph admon ---



% !split
\subsection*{Phenomenology of nuclear forces}

% --- begin paragraph admon ---
\paragraph{Charge Dependence, $^1S_0$ Scattering Lengths.}

Charge-symmetry breaking (CSB), after electromagnetic effects
have been removed:
\begin{itemize}
\item $a_{pp}=  -17.3 \pm 0.4 \hspace{0.cm} \mathrm{fm}$

\item $a_{nn}=-18.8 \pm 0.5 \hspace{0.cm} \mathrm{fm}$. Note however discrepancy from $nd$ breakup reactions resulting in  $a_{nn}=-18.72 \pm 0.13 \pm 0.65 \hspace{0.cm} \mathrm{fm}$ and $\pi^- + d \rightarrow \gamma + 2n$ reactions giving  $a_{nn}=-18.93 \pm 0.27 \pm 0.3 \hspace{0.cm} \mathrm{fm}$.
\end{itemize}

\noindent
Charge-independence breaking (CIB)
\begin{itemize}
\item $a_{pn}=  -23.74 \pm 0.02 \hspace{0.cm} \mathrm{fm}$ 
\end{itemize}

\noindent
% --- end paragraph admon ---





% !split
\subsection*{Symmetries of the Nucleon-Nucleon (NN) Force}

% --- begin paragraph admon ---
\paragraph{}
\begin{itemize}
\item Translation invariance

\item Galilean invariance

\item Rotation invariance in space

\item Space reflection invariance

\item Time reversal invariance

\item Invariance under the interchange of particle $1$ and $2$

\item Almost isospin symmetry
\end{itemize}

\noindent
% --- end paragraph admon ---





% !split
\subsection*{A typical form of the nuclear force}

% --- begin paragraph admon ---
\paragraph{}
Here we display a typical way to parametrize (non-relativistic expression) the nuclear two-body force
in terms of some operators, the central part, the spin-spin part and the central force.
\[
V(\mathbf{r})= \left\{ C_c + C_\mathbf{\sigma} \mathbf{\sigma}_1\cdot\mathbf{\sigma}_2
 + C_T \left( 1 + {3\over m_\alpha r} + {3\over\left(m_\alpha r\right)^2}\right) S_{12} (\hat r)\right. 
\]
\[
\left. + C_{SL} \left( {1\over m_\alpha r} + {1\over \left( m_\alpha r\right)^2}
\right) \mathbf{L}\cdot \mathbf{S}
\right\} \frac{e^{-m_\alpha r}}{m_\alpha r}
\]
How do we derive such terms?  (Note: no isospin dependence and that the above is an approximation)
% --- end paragraph admon ---




% !split
\subsection*{Nuclear forces}

% --- begin paragraph admon ---
\paragraph{}
To derive the above famous form of the nuclear force using field theoretical concepts, we will need some 
elements from relativistic quantum mechanics. These derivations will be given below. 
The material here gives some background to this.
I know that many of you have not taken a course in quantum field theory. I hope however that you can see the basic ideas leading to the famous non-relativistic expressions for the nuclear force. 

\textbf{Furthermore, when we analyze nuclear data, we will actually try to explain properties like spectra, single-particle energies etc in terms of the various terms of the nuclear force. Moreover, many of you will hear about these terms at various talks, workshops, seminars etc. Then, it is good to have an idea of what people actually mean!!}
% --- end paragraph admon ---

  

% !split
\subsection*{Dramatis Personae}

% --- begin paragraph admon ---
\paragraph{}


\begin{quote}
\begin{tabular}{cccc}
\hline
\multicolumn{1}{c}{ Baryons } & \multicolumn{1}{c}{ Mass (MeV) } & \multicolumn{1}{c}{ Mesons } & \multicolumn{1}{c}{ Mass (MeV) } \\
\hline
$p,n$     & 938.926    & $\pi$       & 138.03          \\
$\Lambda$ & 1116.0     & $\eta$      & 548.8           \\
$\Sigma$  & 1197.3     & $\sigma$    & $\approx 550.0$ \\
$\Delta$  & 1232.0     & $\rho$      & 770             \\
          &            & $\omega$    & 782.6           \\
          &            & $\delta$    & 983.0           \\
          &            & $K$         & 495.8           \\
          &            & $K^{\star}$ & 895.0           \\
\hline
\end{tabular}
\end{quote}

\noindent
% --- end paragraph admon ---




% !split
\subsection*{Components of the force and quantum numbers}

% --- begin paragraph admon ---
\paragraph{}
But before we proceed, we will look into specific quantum numbers of the relative system and study 
expectation vaues of the various terms of
\[
V(\mathbf{r})= \left\{ C_c + C_\mathbf{\sigma} \mathbf{\sigma}_1\cdot\mathbf{\sigma}_2
 + C_T \left( 1 + {3\over m_\alpha r} + {3\over
\left(m_\alpha r\right)^2}\right) S_{12} (\hat r)\right. 
\]
\[
\left. + C_{SL} \left( {1\over m_\alpha r} + {1\over \left( m_\alpha r\right)^2}
\right) \mathbf{L}\cdot \mathbf{S}
\right\} \frac{e^{-m_\alpha r}}{m_\alpha r}
\]
% --- end paragraph admon ---



% !split
\subsection*{Relative and CoM system, quantum numbers}

% --- begin paragraph admon ---
\paragraph{}

When solving the scattering equation or solving the two-nucleon problem, it is convenient to rewrite the Schroedinger equation, due to
the spherical symmetry of the Hamiltonian, in relative and center-of-mass coordinates. This will also define the quantum numbers of the relative and center-of-mass system and will aid us later in solving
the so-called Lippman-Schwinger equation for the scattering problem. 

We define the center-of-mass (CoM)  momentum as
 \[
    \mathbf{K}=\sum_{i=1}^A\mathbf{k}_i,
 \]
with $\hbar=c=1$ the wave number $k_i=p_i$, with $p_i$ the pertinent momentum of a single-particle state. 
We have also the relative momentum
\[
    \mathbf{k}_{ij}=\frac{1}{2}(\mathbf{k}_i-\mathbf{k}_j).
 \]
We will below skip the indices $ij$ and simply write $\mathbf{k}$
% --- end paragraph admon ---



% !split
\subsection*{Relative and CoM system, quantum numbers}

% --- begin paragraph admon ---
\paragraph{}

In a similar fashion we can define the CoM coordinate
 \[
     \mathbf{R}=\frac{1}{A}\sum_{i=1}^{A}\mathbf{r}_i,
 \]
 and the relative distance 
\[
    \mathbf{r}_{ij}=(\mathbf{r}_i-\mathbf{r}_j).
 \]
% --- end paragraph admon ---



% !split
\subsection*{Relative and CoM system, quantum numbers}

% --- begin paragraph admon ---
\paragraph{}
With the definitions
 \[
    \mathbf{K}=\sum_{i=1}^A\mathbf{k}_i,
 \]
and
\[
    \mathbf{k}_{ij}=\frac{1}{2}(\mathbf{k}_i-\mathbf{k}_j).
 \]
we can rewrite the two-particle kinetic energy (note that we use $\hbar=c=1$ as 
\[
\frac{\mathbf{k}_1^2}{2m_n}+\frac{\mathbf{k}_2^2}{2m_n}=\frac{\mathbf{k}^2}{m_n}+\frac{\mathbf{K}^2}{4m_n},
\]
where $m_n$ is the average of the proton and the neutron masses.
% --- end paragraph admon ---




% !split
\subsection*{Relative and CoM system, quantum numbers}

% --- begin paragraph admon ---
\paragraph{}

Since the two-nucleon interaction depends only on the relative distance, this means that we can separate Schroedinger's equation in an equation for the center-of-mass motion and one for the relative motion.

With an equation for the relative motion only and a separate one for the center-of-mass motion we need to redefine the two-body quantum numbers.

Previously we had a two-body state vector defined as $|(j_1j_2)JM_J\rangle$ in a coupled basis. 
We will now define the quantum numbers for the relative motion. Here we need to define new orbital momenta (since these are the quantum numbers which change). 
We define 
\[
\hat{l}_1+\hat{l}_2=\hat{\lambda}=\hat{l}+\hat{L},
\]
where $\hat{l}$ is the orbital momentum associated with the relative motion and
$\hat{L}$ the corresponding one linked with the CoM. The total spin $S$ is unchanged since it acts in a different space. We have thus that
\[
\hat{J}=\hat{l}+\hat{L}+\hat{S},
\]
which allows us to define the angular momentum of the relative motion
\[
{ \cal J} =  \hat{l}+\hat{S},
\]
where ${ \cal J}$ is the total angular momentum of the relative motion.
% --- end paragraph admon ---



% !split
\subsection*{Phenomenology of nuclear forces}

% --- begin paragraph admon ---
\paragraph{}

The total two-nucleon state function has to be anti-symmetric. The total function contains a spatial part, a spin part and an isospin part. If isospin is conserved, this leads to in case we have an $s$-wave with spin $S=0$ to an isospin 
two-body state with $T=1$ since the spatial part is symmetric and the spin part is anti-symmetric. 

Since the projections for $T$ are $T_z=-1,0,1$, we can have a $pp$, an $nn$ and a $pn$ state.

For $l=0$ and $S=1$, a so-called triplet state, $^3S_1$, we must have $T=0$, meaning that we have only one state, a $pn$ state. For other partial waves, the following table lists states up to $f$ waves.
We can systemize this in a table as follows, recalling that $|\mathbf{l}-\mathbf{S}| \le |\mathbf{J}| \le |\mathbf{l}+\mathbf{S}|$,  


\begin{quote}
\begin{tabular}{cccccccc}
\hline
\multicolumn{1}{c}{ $^{2S+1}l_J$ } & \multicolumn{1}{c}{ $J$ } & \multicolumn{1}{c}{ $l$ } & \multicolumn{1}{c}{ $S$ } & \multicolumn{1}{c}{ $T$ } & \multicolumn{1}{c}{ $\vert pp\rangle$ } & \multicolumn{1}{c}{ $\vert pn\rangle$ } & \multicolumn{1}{c}{ $\vert nn\rangle$ } \\
\hline
$^{1}S_0$    & 0   & 0   & 0   & 1   & yes               & yes               & yes               \\
$^{3}S_1$    & 1   & 0   & 1   & 0   & no                & yes               & no                \\
$^{3}P_0$    & 0   & 1   & 1   & 1   & yes               & yes               & yes               \\
$^{1}P_1$    & 1   & 1   & 0   & 0   & no                & yes               & no                \\
$^{3}P_1$    & 1   & 1   & 1   & 1   & yes               & yes               & yes               \\
$^{3}P_2$    & 2   & 1   & 1   & 1   & yes               & yes               & yes               \\
$^{3}D_1$    & 1   & 2   & 1   & 0   & no                & yes               & no                \\
$^{3}F_2$    & 2   & 3   & 1   & 1   & yes               & yes               & yes               \\
\hline
\end{tabular}
\end{quote}

\noindent
% --- end paragraph admon ---





% !split
\subsection*{Components of the force and quantum numbers}

% --- begin paragraph admon ---
\paragraph{}
The tensor force is given by
\[
S_{12} (\hat r) = \frac{3}{r^2}\left(\mathbf{\sigma}_1\cdot \mathbf{r}\right) \left(\mathbf{\sigma}_2\cdot \mathbf{r}\right) -\mathbf{\sigma}_1\cdot\mathbf{\sigma}_2\]
where the Pauli matrices are defined as
\[
\sigma_x =\begin{Bmatrix} 0 & 1 \\ 1 & 0 \end{Bmatrix},
\]
\[
\sigma_y =\begin{Bmatrix} 0 & -\imath \\ \imath & 0 \end{Bmatrix},
\]
and
\[
\sigma_z =\begin{Bmatrix} 1 & 0 \\ 0 & -1 \end{Bmatrix},
\]
with the properties $\sigma = 2\mathbf{S}$ (the spin of the system, being $1/2$ for nucleons), 
$\sigma^2_x=\sigma^2_y=\sigma_z=\mathbf{1}$ and
obeying the commutation and anti-commutation relations $\{\sigma_x,\sigma_y\} =0$
$[\sigma_x,\sigma_y] =\imath\sigma_z$ etc.
% --- end paragraph admon ---



% !split
\subsection*{Components of the force and quantum numbers}

% --- begin paragraph admon ---
\paragraph{}
When we look at the expectation value of 
$\langle \mathbf{\sigma}_1\cdot\mathbf{\sigma}_2\rangle$, we can rewrite this expression in terms of the
spin $\mathbf{S}=\mathbf{s}_1+\mathbf{s}_2$, resulting in 
\[
\langle\mathbf{\sigma}_1\cdot\mathbf{\sigma}_2\rangle=2(S^2-s_1^2-s_2^2)=2S(S+1)-3,
\]
where we $s_1=s_2=1/2$ leading to
\[
\left\{ \begin{array}{cc} \langle\mathbf{\sigma}_1\cdot\mathbf{\sigma}_2\rangle=1 &  \mathrm{if} \hspace{0.2cm} S=1\\
\langle\mathbf{\sigma}_1\cdot\mathbf{\sigma}_2\rangle=-3 & \mathrm{if} \hspace{0.2cm} S=0\\\end{array}\right.
\]
% --- end paragraph admon ---



% !split
\subsection*{Components of the force and quantum numbers}

% --- begin paragraph admon ---
\paragraph{}

Similarly, the expectation value of the spin-orbit term is 
\[
\langle \mathbf{l}\mathbf{S} \rangle = \frac{1}{2}\left( J(J+1)-l(l+1)-S(S+1)\right),
\]
which means that for $s$-waves with either $S=0$ and thereby $J=0$ or $S=1$ and $J=1$, 
the expectation value for the
spin-orbit force is zero. With the above phenomenological model, the
only contributions to the expectation value of the potential energy for $s$-waves
stem  from the central and the spin-spin components since the
expectation value of the tensor force is also zero.
% --- end paragraph admon ---



% !split
\subsection*{Components of the force and quantum numbers}

% --- begin paragraph admon ---
\paragraph{}
 For $s=1/2$ spin values only for two nucleons, the expectation value of the tensor force operator is 


\begin{quote}
\begin{tabular}{cccc}
\hline
\multicolumn{1}{c}{  } & \multicolumn{1}{c}{ $l'$ } & \multicolumn{1}{c}{  } & \multicolumn{1}{c}{  } \\
\hline
$l$   & $J+1$                         & $J$ & $J-1$                         \\
\hline
      &                               &     &                               \\
$J+1$ & $-\frac{2J(J+2)}{2J+1}$       & 0   & $\frac{6\sqrt{J(J+1)}}{2J+1}$ \\
      &                               &     &                               \\
$J$   & 0                             & 2   & 0                             \\
      &                               &     &                               \\
$J-1$ & $\frac{6\sqrt{J(J+1)}}{2J+1}$ & 0   & $-\frac{2(2J+1)}{2J+1}$       \\
      &                               &     &                               \\
\hline
\end{tabular}
\end{quote}

\noindent
We will derive these expressions after we have discussed the Wigner-Eckart theorem.
% --- end paragraph admon ---






% !split
\subsection*{Components of the force and isospin}

% --- begin paragraph admon ---
\paragraph{}
If we now add isospin to our simple $V_4$ interaction model, we end up with $8$ operators, popularly dubbed $V_8$ interaction model. The explicit form reads
\[
V(\mathbf{r})= \left\{ C_c + C_\mathbf{\sigma} \mathbf{\sigma}_1\cdot\mathbf{\sigma}_2
 + C_T \left( 1 + {3\over m_\alpha r} + {3\over
\left(m_\alpha r\right)^2}\right) S_{12} (\hat r)\right. 
\]
\[
\left. + C_{SL} \left( {1\over m_\alpha r} + {1\over \left( m_\alpha r\right)^2}
\right) \mathbf{L}\cdot \mathbf{S}
\right\} \frac{e^{-m_\alpha r}}{m_\alpha r}
\]
\[
+ \left\{ C_{c\tau} + C_{\sigma\tau}\mathbf{\sigma}_1\cdot\mathbf{\sigma}_2
 + C_{T\tau} \left( 1 + {3\over m_\alpha r} + {3\over
\left(m_\alpha r\right)^2}\right) S_{12} (\hat r)\right. 
\]
\[
\left. + C_{SL\tau} \left( {1\over m_\alpha r} + {1\over \left( m_\alpha r\right)^2}
\right) \mathbf{L}\cdot \mathbf{S}
\right\}\mathbf{\tau}_1\cdot\mathbf{\tau}_2 \frac{e^{-m_\alpha r}}{m_\alpha r}
\]
% --- end paragraph admon ---



% !split
\subsection*{Phenomenology of nuclear forces}

% --- begin paragraph admon ---
\paragraph{References for Various Phenomenological Interactions.}

From 1950 till approximately 2000: One-Boson-Exchange (OBE) models dominate. These are models which typically include several low-mass mesons, that is with masses below 1 GeV.   Potentials which are based upon the standard non-relativistic operator structure
 are called "Phenomenological Potentials" Some historically important examples are   
\begin{itemize}
\item Gammel-Thaler potential ( Phys. Rev. \textbf{107}, 291, 1339 (1957) and the 

\item Hamada-Johnston potential, Nucl. Phys. \textbf{34}, 382 (1962)), both with a hard core.

\item Reid potential (Ann. Phys. (N.Y.) \textbf{50}, 411 (1968)), soft core.

\item Argonne $V_{14}$ potential (Wiringa et al., Phys. Rev. C \textbf{29}, 1207 (1984)) with 14 operators and  the  Argonne $V_{18}$ potential (Wiringa et al., Phys. Rev. C \textbf{51}, 38 (1995)), uses 18 operators

\item A good historical reference: R.~Machleidt, Adv.~Nucl.~Phys.  \textbf{19}, 189 (1989).
\end{itemize}

\noindent
Now: models based on chiral perturbation theory. These are effective models with nucleons and pions as degrees of freedom only. The other mesons which appeared in standard one-boson model appear as multi-pion resonances.
% --- end paragraph admon ---






% !split
\subsection*{Phenomenology of nuclear forces}

% --- begin paragraph admon ---
\paragraph{}
The total two-nucleon state function has to be anti-symmetric. The total function contains a spatial part, a spin part and an isospin part. If isospin is conserved, this leads to in case we have an $s$-wave with spin $S=0$ to an isospin 
two-body state with $T=1$ since the spatial part is symmetric and the spin part is anti-symmetric. 

Since the projections for $T$ are $T_z=-1,0,1$, we can have a $pp$, an $nn$ and a $pn$ state.

For $l=0$ and $S=1$, a so-called triplet state, $^3S_1$, we must have $T=0$, meaning that we have only one state, a $pn$ state. For other partial waves, see exercises below.
% --- end paragraph admon ---



% !split
\subsection*{Phenomenology of nuclear forces}

% --- begin paragraph admon ---
\paragraph{Phenomenology of one-pion exchange.}

The one-pion exchange contribution (see derivation below), can be written as 
\[
V_{\pi}(\mathbf{r})= -\frac{f_{\pi}^{2}}{4\pi m_{\pi}^{2}}\mathbf{ \tau}_1\cdot\mathbf{\tau}_2
\frac{1}{3}\left\{\mathbf{ \sigma}_1\cdot\mathbf{ \sigma}_2+\left( 1 + {3\over m_\pi r} + {3\over\left(m_\pi r\right)^2}\right) S_{12} (\hat r)\right\} \frac{e^{-m_\pi r}}{m_\pi r}.
\]
Here the constant $f_{\pi}^{2}/4\pi\approx 0.08$ and the mass of the pion is $m_\pi\approx 140$ MeV$/\mbox{c}^2$.
% --- end paragraph admon ---


% !split
\subsection*{Phenomenology of nuclear forces}

% --- begin paragraph admon ---
\paragraph{}
Let us look closer at specific partial waves for which one-pion exchange is applicable. If we have $S=0$ and $T=0$, the 
orbital momentum has to be an odd number in order for the total anti-symmetry to be obeyed. For $S=0$, the tensor force component is zero, meaning that 
the only contribution is 
\[
V_{\pi}(\mathbf{r})=\frac{3f_{\pi}^{2}}{4\pi m_{\pi}^{2}}\frac{e^{-m_\pi r}}{m_\pi r},
\]
since $\langle\mathbf{ \sigma}_1\cdot\mathbf{ \sigma}_2\rangle=-3$, that is we obtain a repulsive contribution to partial waves like 
$^1P_0$.
% --- end paragraph admon ---


% !split
\subsection*{Phenomenology of nuclear forces}

% --- begin paragraph admon ---
\paragraph{}
Since $S=0$ yields always a zero tensor force contribution, for the combination of $T=1$ and then even $l$ values, we get an attractive contribution
\[
V_{\pi}(\mathbf{r})=-\frac{f_{\pi}^{2}}{4\pi m_{\pi}^{2}}\frac{e^{-m_\pi r}}{m_\pi r}.
\]
With $S=1$ and $T=0$, $l$ can only take even values in order to obey the anti-symmetry requirements and we get
\[
V_{\pi}(\mathbf{r})= -\frac{f_{\pi}^{2}}{4\pi m_{\pi}^{2}}
\left(1+( 1 + {3\over m_\pi r} + {3\over\left(m_\pi r\right))^2}) S_{12} (\hat r)\right) \frac{e^{-m_\pi r}}{m_\pi r},
\]
while for $S=1$ and $T=1$, $l$ can only take odd values, resulting in a repulsive contribution 
\[
V_{\pi}(\mathbf{r})= \frac{1}{3}\frac{f_{\pi}^{2}}{4\pi m_{\pi}^{2}}\left(1+( 1 + {3\over m_\pi r} + {3\over\left(m_\pi r\right)^2}) S_{12} (\hat r)\right) \frac{e^{-m_\pi r}}{m_\pi r}.
\]
% --- end paragraph admon ---



% !split
\subsection*{Phenomenology of nuclear forces}

% --- begin paragraph admon ---
\paragraph{}

The central part of one-pion exchange interaction, arising from the spin-spin term,  
is thus attractive for $s$-waves and all even $l$ values. For $p$-waves and all other odd values
it is repulsive. However, its overall strength is weak. This is discussed further in one of exercises below.
% --- end paragraph admon ---







% ------------------- end of main content ---------------


\printindex

\end{document}

