%%
%% Automatically generated file from DocOnce source
%% (https://github.com/hplgit/doconce/)
%%
%%
% #ifdef PTEX2TEX_EXPLANATION
%%
%% The file follows the ptex2tex extended LaTeX format, see
%% ptex2tex: http://code.google.com/p/ptex2tex/
%%
%% Run
%%      ptex2tex myfile
%% or
%%      doconce ptex2tex myfile
%%
%% to turn myfile.p.tex into an ordinary LaTeX file myfile.tex.
%% (The ptex2tex program: http://code.google.com/p/ptex2tex)
%% Many preprocess options can be added to ptex2tex or doconce ptex2tex
%%
%%      ptex2tex -DMINTED myfile
%%      doconce ptex2tex myfile envir=minted
%%
%% ptex2tex will typeset code environments according to a global or local
%% .ptex2tex.cfg configure file. doconce ptex2tex will typeset code
%% according to options on the command line (just type doconce ptex2tex to
%% see examples). If doconce ptex2tex has envir=minted, it enables the
%% minted style without needing -DMINTED.
% #endif

% #define PREAMBLE

% #ifdef PREAMBLE
%-------------------- begin preamble ----------------------

\documentclass[%
twoside,                 % oneside: electronic viewing, twoside: printing
final,                   % or draft (marks overfull hboxes, figures with paths)
10pt]{article}

\listfiles               % print all files needed to compile this document

\usepackage{relsize,makeidx,color,setspace,amsmath,amsfonts}
\usepackage[table]{xcolor}
\usepackage{bm,microtype}

\usepackage[T1]{fontenc}
%\usepackage[latin1]{inputenc}
\usepackage{ucs}
\usepackage[utf8x]{inputenc}

\usepackage{lmodern}         % Latin Modern fonts derived from Computer Modern

% Hyperlinks in PDF:
\definecolor{linkcolor}{rgb}{0,0,0.4}
\usepackage{hyperref}
\hypersetup{
    breaklinks=true,
    colorlinks=true,
    linkcolor=linkcolor,
    urlcolor=linkcolor,
    citecolor=black,
    filecolor=black,
    %filecolor=blue,
    pdfmenubar=true,
    pdftoolbar=true,
    bookmarksdepth=3   % Uncomment (and tweak) for PDF bookmarks with more levels than the TOC
    }
%\hyperbaseurl{}   % hyperlinks are relative to this root

\setcounter{tocdepth}{2}  % number chapter, section, subsection

\usepackage[framemethod=TikZ]{mdframed}

% --- begin definitions of admonition environments ---

% Admonition style "mdfbox" is an oval colored box based on mdframed
% "notice" admon
\colorlet{mdfbox_notice_background}{gray!5}
\newmdenv[
  skipabove=15pt,
  skipbelow=15pt,
  outerlinewidth=0,
  backgroundcolor=mdfbox_notice_background,
  linecolor=black,
  linewidth=2pt,       % frame thickness
  frametitlebackgroundcolor=mdfbox_notice_background,
  frametitlerule=true,
  frametitlefont=\normalfont\bfseries,
  shadow=false,        % frame shadow?
  shadowsize=11pt,
  leftmargin=0,
  rightmargin=0,
  roundcorner=5,
  needspace=0pt,
]{notice_mdfboxmdframed}

\newenvironment{notice_mdfboxadmon}[1][]{
\begin{notice_mdfboxmdframed}[frametitle=#1]
}
{
\end{notice_mdfboxmdframed}
}

% Admonition style "mdfbox" is an oval colored box based on mdframed
% "summary" admon
\colorlet{mdfbox_summary_background}{gray!5}
\newmdenv[
  skipabove=15pt,
  skipbelow=15pt,
  outerlinewidth=0,
  backgroundcolor=mdfbox_summary_background,
  linecolor=black,
  linewidth=2pt,       % frame thickness
  frametitlebackgroundcolor=mdfbox_summary_background,
  frametitlerule=true,
  frametitlefont=\normalfont\bfseries,
  shadow=false,        % frame shadow?
  shadowsize=11pt,
  leftmargin=0,
  rightmargin=0,
  roundcorner=5,
  needspace=0pt,
]{summary_mdfboxmdframed}

\newenvironment{summary_mdfboxadmon}[1][]{
\begin{summary_mdfboxmdframed}[frametitle=#1]
}
{
\end{summary_mdfboxmdframed}
}

% Admonition style "mdfbox" is an oval colored box based on mdframed
% "warning" admon
\colorlet{mdfbox_warning_background}{gray!5}
\newmdenv[
  skipabove=15pt,
  skipbelow=15pt,
  outerlinewidth=0,
  backgroundcolor=mdfbox_warning_background,
  linecolor=black,
  linewidth=2pt,       % frame thickness
  frametitlebackgroundcolor=mdfbox_warning_background,
  frametitlerule=true,
  frametitlefont=\normalfont\bfseries,
  shadow=false,        % frame shadow?
  shadowsize=11pt,
  leftmargin=0,
  rightmargin=0,
  roundcorner=5,
  needspace=0pt,
]{warning_mdfboxmdframed}

\newenvironment{warning_mdfboxadmon}[1][]{
\begin{warning_mdfboxmdframed}[frametitle=#1]
}
{
\end{warning_mdfboxmdframed}
}

% Admonition style "mdfbox" is an oval colored box based on mdframed
% "question" admon
\colorlet{mdfbox_question_background}{gray!5}
\newmdenv[
  skipabove=15pt,
  skipbelow=15pt,
  outerlinewidth=0,
  backgroundcolor=mdfbox_question_background,
  linecolor=black,
  linewidth=2pt,       % frame thickness
  frametitlebackgroundcolor=mdfbox_question_background,
  frametitlerule=true,
  frametitlefont=\normalfont\bfseries,
  shadow=false,        % frame shadow?
  shadowsize=11pt,
  leftmargin=0,
  rightmargin=0,
  roundcorner=5,
  needspace=0pt,
]{question_mdfboxmdframed}

\newenvironment{question_mdfboxadmon}[1][]{
\begin{question_mdfboxmdframed}[frametitle=#1]
}
{
\end{question_mdfboxmdframed}
}

% Admonition style "mdfbox" is an oval colored box based on mdframed
% "block" admon
\colorlet{mdfbox_block_background}{gray!5}
\newmdenv[
  skipabove=15pt,
  skipbelow=15pt,
  outerlinewidth=0,
  backgroundcolor=mdfbox_block_background,
  linecolor=black,
  linewidth=2pt,       % frame thickness
  frametitlebackgroundcolor=mdfbox_block_background,
  frametitlerule=true,
  frametitlefont=\normalfont\bfseries,
  shadow=false,        % frame shadow?
  shadowsize=11pt,
  leftmargin=0,
  rightmargin=0,
  roundcorner=5,
  needspace=0pt,
]{block_mdfboxmdframed}

\newenvironment{block_mdfboxadmon}[1][]{
\begin{block_mdfboxmdframed}[frametitle=#1]
}
{
\end{block_mdfboxmdframed}
}

% --- end of definitions of admonition environments ---

% prevent orhpans and widows
\clubpenalty = 10000
\widowpenalty = 10000

% --- end of standard preamble for documents ---


% insert custom LaTeX commands...

\raggedbottom
\makeindex

%-------------------- end preamble ----------------------

\begin{document}

% #endif


% ------------------- main content ----------------------

% Slides for PHY981


% ----------------- title -------------------------

\title{Infinite matter}

% ----------------- author(s) -------------------------

\author{Morten Hjorth-Jensen, National Superconducting Cyclotron Laboratory and Department of Physics and Astronomy, Michigan State University, East Lansing, MI 48824, USA {\&} Department of Physics, University of Oslo, Oslo, Norway\inst{}}
\institute{}
% ----------------- end author(s) -------------------------

\date{Spring 2015
% <optional titlepage figure>
}

% !split
\subsection{Studies of infinite matter}

\begin{block}{}
\begin{itemize}
\item For historical and pedagogical  reasons we start with the electron gas in two and three dimensions 

\item Thereafter we discuss infinite nuclear matter
\end{itemize}

\noindent
\end{block}



% !split
\subsection{Electron gas and HF solution}
\begin{block}{}
The electron gas is perhaps the only realistic model of a 
system of many interacting particles that allows for a solution
of the Hartree-Fock equations on a closed form. Furthermore, to first order in the interaction, one can also
compute on a closed form the total energy and several other properties of a many-particle systems. 
The model gives a very good approximation to the properties of valence electrons in metals.
The assumptions are

\begin{itemize}
 \item System of electrons that is not influenced by external forces except by an attraction provided by a uniform background of ions. These ions give rise to a uniform background charge. The ions are stationary.

 \item The system as a whole is neutral.

 \item We assume we have $N_e$ electrons in a cubic box of length $L$ and volume $\Omega=L^3$. This volume contains also a
\end{itemize}

\noindent
uniform distribution of positive charge with density $N_ee/\Omega$. 


This is a homogeneous system and the one-particle wave functions are given by plane wave functions normalized to a volume $\Omega$ 
for a box with length $L$ (the limit $L\rightarrow \infty$ is to be taken after we have computed various expectation values)
\[
\psi_{{\bf k}\sigma}({\bf r})= \frac{1}{\sqrt{\Omega}}\exp{(i{\bf kr})}\xi_{\sigma}
\]
where ${\bf k}$ is the wave number and  $\xi_{\sigma}$ is a spin function for either spin up or down
\[ 
\xi_{\sigma=+1/2}=\left(\begin{array}{c} 1 \\ 0 \end{array}\right) \hspace{0.5cm}
\xi_{\sigma=-1/2}=\left(\begin{array}{c} 0 \\ 1 \end{array}\right).
\]
We assume that we have periodic boundary conditions which limit the allowed wave numbers to
\[
k_i=\frac{2\pi n_i}{L}\hspace{0.5cm} i=x,y,z \hspace{0.5cm} n_i=0,\pm 1,\pm 2, \dots
\]
We assume first that the electrons interact via a central, symmetric and translationally invariant
interaction  $V(r_{12})$ with
$r_{12}=|{\bf r}_1-{\bf r}_2|$.  The interaction is spin independent.

The total Hamiltonian consists then of kinetic and potential energy
\[
\hat{H} = \hat{T}+\hat{V}.
\]
The operator for the kinetic energy can be written as
\[
\hat{T}=\sum_{{\bf k}\sigma}\frac{\hbar^2k^2}{2m}a_{{\bf k}\sigma}^{\dagger}a_{{\bf k}\sigma}.
\]

The Hamilton operator is given by
\[
\hat{H}=\hat{H}_{el}+\hat{H}_{b}+\hat{H}_{el-b},
\]
with the electronic part
\[
\hat{H}_{el}=\sum_{i=1}^N\frac{p_i^2}{2m}+\frac{e^2}{2}\sum_{i\ne j}\frac{e^{-\mu |{\bf r}_i-{\bf r}_j|}}{|{\bf r}_i-{\bf r}_j|},
\]
where we have introduced an explicit convergence factor
(the limit $\mu\rightarrow 0$ is performed after having calculated the various integrals).
Correspondingly, we have
\[
\hat{H}_{b}=\frac{e^2}{2}\int\int d{\bf r}d{\bf r}'\frac{n({\bf r})n({\bf r}')e^{-\mu |{\bf r}-{\bf r}'|}}{|{\bf r}-{\bf r}'|},
\]
which is the energy contribution from the positive background charge with density
$n({\bf r})=N/\Omega$. Finally,
\[
\hat{H}_{el-b}=-\frac{e^2}{2}\sum_{i=1}^N\int d{\bf r}\frac{n({\bf r})e^{-\mu |{\bf r}-{\bf x}_i|}}{|{\bf r}-{\bf x}_i|},
\]
is the interaction between the electrons and the positive background.
\end{block}

% !split
\subsection{Electron gas and HF solution}
\begin{block}{}
In exercise 5 we show that the Hartree-Fock energy can be written as 
\[
\varepsilon_{k}^{HF}=\frac{\hbar^{2}k^{2}}{2m_e}-\frac{e^{2}}
{\Omega^{2}}\sum_{k'\leq
k_{F}}\int d{\bf r}e^{i({\bf k}'-{\bf k}){\bf r}}\int
d{\bf r'}\frac{e^{i({\bf k}-{\bf k}'){\bf r}'}}
{\vert{\bf r}-{\bf r}'\vert}
\]
resulting in
\[
\varepsilon_{k}^{HF}=\frac{\hbar^{2}k^{2}}{2m_e}-\frac{e^{2}
k_{F}}{2\pi}
\left[
2+\frac{k_{F}^{2}-k^{2}}{kk_{F}}ln\left\vert\frac{k+k_{F}}
{k-k_{F}}\right\vert
\right]
\]

The results can be rewritten in terms of the density
\[
n= \frac{k_F^3}{3\pi^2}=\frac{3}{4\pi r_s^3},
\]
where $n=N_e/\Omega$, $N_e$ being the number of electrons, and $r_s$ is the radius of a sphere which represents the volum per conducting electron.  
It can be convenient to use the Bohr radius $a_0=\hbar^2/e^2m_e$.
For most metals we have a relation $r_s/a_0\sim 2-6$.  The quantity $r_s$ is dimensionless.

In exercise 6 we find that
the total energy
$E_0/N_e=\langle\Phi_{0}|\hat{H}|\Phi_{0}\rangle/N_e$ for
for this system to first order in the interaction is given as 
\[
E_0/N_e=\frac{e^2}{2a_0}\left[\frac{2.21}{r_s^2}-\frac{0.916}{r_s}\right].
\]

\end{block}





% !split
\subsection{Exercises: Derivation of Hartree-Fock equations}
\begin{block}{Exercise 1 }
Consider a Slater determinant built up of single-particle orbitals $\psi_{\lambda}$, 
with $\lambda = 1,2,\dots,N$.

The unitary transformation
\[
\psi_a  = \sum_{\lambda} C_{a\lambda}\phi_{\lambda},
\]
brings us into the new basis.  
The new basis has quantum numbers $a=1,2,\dots,N$.
Show that the new basis is orthonormal.
Show that the new Slater determinant constructed from the new single-particle wave functions can be
written as the determinant based on the previous basis and the determinant of the matrix $C$.
Show that the old and the new Slater determinants are equal up to a complex constant with absolute value unity.
(Hint, $C$ is a unitary matrix). 

\end{block}

% !split
\subsection{Exercises: Derivation of Hartree-Fock equations}
\begin{block}{Exercise 2  }
Consider the  Slater  determinant
\[
\Phi_{0}=\frac{1}{\sqrt{n!}}\sum_{p}(-)^{p}P
\prod_{i=1}^{n}\psi_{\alpha_{i}}(x_{i}).
\]
A small variation in this function is given by
\[
\delta\Phi_{0}=\frac{1}{\sqrt{n!}}\sum_{p}(-)^{p}P
\psi_{\alpha_{1}}(x_{1})\psi_{\alpha_{2}}(x_{2})\dots
\psi_{\alpha_{i-1}}(x_{i-1})(\delta\psi_{\alpha_{i}}(x_{i}))
\psi_{\alpha_{i+1}}(x_{i+1})\dots\psi_{\alpha_{n}}(x_{n}).
\]
Show that
\[
\langle \delta\Phi_{0}|\sum_{i=1}^{n}\left\{t(x_{i})+u(x_{i})
\right\}+\frac{1}{2}
\sum_{i\neq j=1}^{n}v(x_{i},x_{j})|\Phi_{0}\rangle=\sum_{i=1}^{n}\langle \delta\psi_{\alpha_{i}}|\hat{t}+\hat{u}
|\phi_{\alpha_{i}}\rangle
+\sum_{i\neq j=1}^{n}\left\{\langle\delta\psi_{\alpha_{i}}
\psi_{\alpha_{j}}|\hat{v}|\psi_{\alpha_{i}}\psi_{\alpha_{j}}\rangle-
\langle\delta\psi_{\alpha_{i}}\psi_{\alpha_{j}}|\hat{v}
|\psi_{\alpha_{j}}\psi_{\alpha_{i}}\rangle\right\}
\]
\end{block}


% !split
\subsection{Exercises: Derivation of Hartree-Fock equations}
\begin{block}{Exercise 3  }
What is the diagrammatic representation of the HF equation?
\[
-\langle\alpha_{k}|u^{HF}|\alpha_{i}\rangle+\sum_{j=1}^{n}
\left[\langle\alpha_{k}\alpha_{j}|\hat{v}|\alpha_{i}\alpha_{j}\rangle-
\langle\alpha_{k}\alpha_{j}|v|\alpha_{j}\alpha_{i}\rangle\right]=0
\]
(Represent $(-u^{HF})$ by the symbol $---$X .)
\end{block}

% !split
\subsection{Exercises: Derivation of Hartree-Fock equations}
\begin{block}{Exercise 4  }
Consider the ground state $|\Phi\rangle$ 
of a bound many-particle system of fermions. Assume that we remove one particle
from the single-particle state $\lambda$ and that our system ends in a new state
$|\Phi_{n}\rangle$. 
Define the energy needed to remove this particle as
\[
E_{\lambda}=\sum_{n}\vert\langle\Phi_{n}|a_{\lambda}|\Phi\rangle\vert^{2}(E_{0}-E_{n}),
\]
where $E_{0}$ and $E_{n}$  are the ground state energies of the states
$|\Phi\rangle$  and  $|\Phi_{n}\rangle$, respectively.
\begin{itemize}
 \item Show that
\end{itemize}

\noindent
\[
E_{\lambda}=\langle\Phi|a_{\lambda}^{\dagger}\left[
a_{\lambda},H \right]|\Phi\rangle,
\]
where $H$ is the Hamiltonian of this system.
\begin{itemize}
 \item If we assume that $\Phi$ is the  Hartree-Fock result, find the 
\end{itemize}

\noindent
relation between $E_{\lambda}$ and the single-particle energy
$\varepsilon_{\lambda}$
for states $\lambda \leq F$ and $\lambda >F$, with
\[
\varepsilon_{\lambda}=\langle\lambda|\hat{t}+\hat{u}|\lambda\rangle,
\]
and
\[
\langle\lambda|\hat{u}|\lambda\rangle=\sum_{\beta \leq F}
\langle\lambda\beta|\hat{v}|\lambda\beta\rangle.
\]
We have assumed an antisymmetrized matrix element here.
Discuss the result.

The Hamiltonian operator is defined as
\[
H=\sum_{\alpha\beta}\langle\alpha|\hat{t}|\beta\rangle a_{\alpha}^{\dagger}a_{\beta}+
\frac{1}{2}\sum_{\alpha\beta\gamma\delta}\langle\alpha\beta|\hat{v}|\gamma\delta\rangle a_{\alpha}^{\dagger}a_{\beta}^{\dagger}a_{\delta}a_{\gamma}.
\]
\end{block}


% !split
\subsection{Exercises: Electron gas}
\begin{block}{Exercise 5 }
The electron gas model allows closed form solutions for quantities like the 
single-particle Hartree-Fock energy.  The latter quantity is given by the following expression
\[
\varepsilon_{k}^{HF}=\frac{\hbar^{2}k^{2}}{2m}-\frac{e^{2}}
{V^{2}}\sum_{k'\leq
k_{F}}\int d{\bf r}e^{i({\bf k'}-{\bf k}){\bf r}}\int
d{\bf r}'\frac{e^{i({\bf k}-{\bf k'}){\bf r}'}}
{\vert{\bf r}-{\bf r'}\vert}
\]
\begin{itemize}
 \item Show that
\end{itemize}

\noindent
\[
\varepsilon_{k}^{HF}=\frac{\hbar^{2}k^{2}}{2m}-\frac{e^{2}
k_{F}}{2\pi}
\left[
2+\frac{k_{F}^{2}-k^{2}}{kk_{F}}ln\left\vert\frac{k+k_{F}}
{k-k_{F}}\right\vert
\right]
\]
(Hint: Introduce the convergence factor 
$e^{-\mu\vert{\bf r}-{\bf r}'\vert}$
in the potential and use  $\sum_{{\bf k}}\rightarrow
\frac{V}{(2\pi)^{3}}\int d{\bf k}$ )
\begin{itemize}
 \item Rewrite the above result as a function of the density
\end{itemize}

\noindent
\[
n= \frac{k_F^3}{3\pi^2}=\frac{3}{4\pi r_s^3},
\]
where $n=N/V$, $N$ being the number of particles, and $r_s$ is the radius of a sphere which represents the volum per conducting electron.  
It can be convenient to use the Bohr radius $a_0=\hbar^2/e^2m$.

For most metals we have a relation $r_s/a_0\sim 2-6$.

Make a plot of the free electron energy and the Hartree-Fock energy and discuss the behavior around the Fermi surface. Extract also   the Hartree-Fock band width $\Delta\varepsilon^{HF}$ defined as
\[ 
\Delta\varepsilon^{HF}=\varepsilon_{k_{F}}^{HF}-
\varepsilon_{0}^{HF}.
\]
Compare this results with the corresponding one for a free electron and comment your results. How large is the contribution due to the exchange term in the Hartree-Fock equation?\newline
\begin{itemize}
 \item We will now define a quantity called the effective mass.
\end{itemize}

\noindent
For $\vert{\bf k}\vert$ near $k_{F}$, we can Taylor expand the Hartree-Fock energy as  
\[
\varepsilon_{k}^{HF}=\varepsilon_{k_{F}}^{HF}+
\left(\frac{\partial\varepsilon_{k}^{HF}}{\partial k}\right)_{k_{F}}(k-k_{F})+\dots
\]
If we compare the latter with the corresponding expressiyon for the non-interacting system
\[
\varepsilon_{k}^{(0)}=\frac{\hbar^{2}k^{2}_{F}}{2m}+
\frac{\hbar^{2}k_{F}}{m}\left(k-k_{F}\right)+\dots ,
\]
we can define the so-called effective Hartree-Fock mass as
\[
m_{HF}^{*}\equiv\hbar^{2}k_{F}\left(
\frac{\partial\varepsilon_{k}^{HF}}
{\partial k}\right)_{k_{F}}^{-1}
\]
Compute $m_{HF}^{*}$ and comment your results after you have done 
point d). \newline
\begin{itemize}
 \item Show that the level density (the number of single-electron states per unit energy) can be written as
\end{itemize}

\noindent
\[
n(\varepsilon)=\frac{Vk^{2}}{2\pi^{2}}\left(
\frac{\partial\varepsilon}{\partial k}\right)^{-1}
\]
Calculate $n(\varepsilon_{F}^{HF})$ and comment the results.


\end{block}



% !split
\subsection{Exercises: Electron gas}
\begin{block}{Solution to exercise 5 }
We want to show that, given the Hartree-Fock equation for the electron gas
\[
\varepsilon_{k}^{HF}=\frac{\hbar^{2}k^{2}}{2m}-\frac{e^{2}}
{V^{2}}\sum_{p\leq
k_{F}}\int d{\bf r}\exp{(i({\bf p}-{\bf k}){\bf r})}\int
d{\bf r}'\frac{\exp{(i({\bf k}-{\bf p}){\bf r}'})}
{\vert{\bf r}-{\bf r'}\vert}
\]
the single-particle energy can be written as
\[
\varepsilon_{k}^{HF}=\frac{\hbar^{2}k^{2}}{2m}-\frac{e^{2}
k_{F}}{2\pi}
\left[
2+\frac{k_{F}^{2}-k^{2}}{kk_{F}}ln\left\vert\frac{k+k_{F}}
{k-k_{F}}\right\vert
\right].
\]
We introduce the convergence factor 
$e^{-\mu\vert{\bf r}-{\bf r}'\vert}$
in the potential and use  $\sum_{{\bf k}}\rightarrow
\frac{V}{(2\pi)^{3}}\int d{\bf k}$. We can then rewrite the integral as 
\[
\frac{e^{2}}
{V^{2}}\sum_{k'\leq
k_{F}}\int d{\bf r}\exp{(i({\bf k'}-{\bf k}){\bf r})}\int
d{\bf r}'\frac{\exp{(i({\bf k}-{\bf p}){\bf r}'})}
{\vert{\bf r}-{\bf r'}\vert}=\frac{e^{2}}{V (2\pi)^3}  \int d{\bf r}\int
\frac{d{\bf r}'}{\vert{\bf r}-{\bf r'}\vert}\exp{(-i{\bf k}({\bf r}-{\bf r}'))}\int d{\bf p}\exp{(i{\bf p}({\bf r}-{\bf r}'))},
\]
and introducing the abovementioned convergence factor we have
\[
\lim_{\mu \to 0}\frac{e^{2}}{V (2\pi)^3}  \int d{\bf r}\int d{\bf r}'\frac{\exp{(-\mu\vert{\bf r}-{\bf r}'\vert})}{\vert{\bf r}-{\bf r'}\vert}\int d{\bf p}\exp{(i({\bf p}-{\bf k})({\bf r}-{\bf r}'))}.
\]
With a change variables to ${\bf x} = {\bf r}-{\bf r}'$ and ${\bf y}={\bf r}'$ we rewrite the last integral as
\[
\lim_{\mu \to 0}\frac{e^{2}}{V (2\pi)^3}  \int d{\bf p}\int d{\bf y}\int d{\bf x}\exp{(i({\bf p}-{\bf k}){\bf x})}\frac{\exp{(-\mu\vert{\bf x}\vert})}{\vert{\bf x}\vert}.
\]
The integration over ${\bf x}$ can be performed using spherical coordinates, resulting in (with $x=\vert {\bf x}\vert$)
\[
\int d{\bf x}\exp{(i({\bf p}-{\bf k}){\bf x})}\frac{\exp{(-\mu\vert{\bf x}\vert})}{\vert{\bf x}\vert}=\int x^2 dx d\phi d\cos{(\theta)}\exp{(i({\bf p}-{\bf k})x\cos{(\theta))}}\frac{\exp{(-\mu x)}}{x},
\]
which yields
\[
4\pi \int dx \frac{ \sin{(\vert {\bf p}-{\bf k}\vert)x} }{\vert {\bf p}-{\bf k}\vert}{\exp{(-\mu x)}}= \frac{4\pi}{\mu^2+\vert {\bf p}-{\bf k}\vert^2}.
\]
This results gives us 
\[
\lim_{\mu \to 0}\frac{e^{2}}{V (2\pi)^3}  \int d{\bf p}\int d{\bf y}\frac{4\pi}{\mu^2+\vert {\bf p}-{\bf k}\vert^2}=\lim_{\mu \to 0}\frac{e^{2}}{ 2\pi^2}  \int d{\bf p}\frac{1}{\mu^2+\vert {\bf p}-{\bf k}\vert^2},
\]
where we have used that the integrand on teh left-hand side does not depend on ${\bf y}$ and that $\int d{\bf y}=V$.Introducing spherical coordinates we can rewrite the integral as 
\[
\lim_{\mu \to 0}\frac{e^{2}}{ 2\pi^2}  \int d{\bf p}\frac{1}{\mu^2+\vert {\bf p}-{\bf k}\vert^2}=\frac{e^{2}}{ 2\pi^2}  \int d{\bf p}\frac{1}{\vert {\bf p}-{\bf k}\vert^2}=\frac{e^{2}}{\pi}  \int_0^{k_F} p^2dp\int_0^{\pi} d\theta\cos{(\theta)}\frac{1}{p^2+k^2-2pk\cos{(\theta)}},
\]
and with the change of variables $\cos{(\theta)}=u$ we have 
\[
\frac{e^{2}}{\pi}  \int_0^{k_F} p^2dp\int_{0}^{\pi} d\theta\cos{(\theta)}\frac{1}{p^2+k^2-2pk\cos{(\theta)}}=\frac{e^{2}}{\pi}  \int_0^{k_F} p^2dp\int_{-1}^{1} du\frac{1}{p^2+k^2-2pku},
\]
which gives
\[
\frac{e^{2}}{k\pi}  \int_0^{k_F} pdp\left\{ln(\vert p+k\vert)-ln(\vert p-k\vert)\right\},
\]
and introducing new variables $x=p+k$ and $y=p-k$, we obtain after some straightforward reordering of the integral
\[
\frac{e^{2}}{k\pi}\left[
kk_F+\frac{k_{F}^{2}-k^{2}}{kk_{F}}ln\left\vert\frac{k+k_{F}}
{k-k_{F}}\right\vert
\right],
\]
which gives the abovementioned expression for the single-particle energy.

Introducing the dimensionless quantity $x=k/k_F$ and the function
\[
F(x) = \frac{1}{2}+\frac{1-x^2}{4x}\ln{\left\vert \frac{1+x}{1-x}\right\vert},
\]
we can rewrite the single-particle Hartree-Fock energy as 
\[
\varepsilon_{k}^{HF}=\frac{\hbar^{2}k^{2}}{2m}-\frac{2e^{2}
k_{F}}{\pi}F(k/k_F),
\]
and dividing by the non-interacting contribution at the Fermi level, 
\[
\varepsilon_{0}^{HF}=\frac{\hbar^{2}k_F^{2}}{2m},
\]
we have
\[
\frac{\varepsilon_{k}^{HF} }{\varepsilon_{0}^{HF}}=x^2-\frac{e^2m}{\hbar^2 k_F\pi}F(x)=x^2-\frac{4}{\pi k_Fa_0}F(x),
\]
where $a_0=0.0529$ nm is the Bohr radius, setting thereby a natural length scale. 
By introducing the radius $r_s$ of a sphere whose volume is the volume occupied by each electron, we can rewrite the previous equation in terms of $r_s$ using that the electron density $n=N/V$
\[
n=\frac{k_F^3}{3\pi^2} = \frac{3}{4\pi r_s^3},
\]
we have (with $k_F=1.92/r_s$,
\[
\frac{\varepsilon_{k}^{HF} }{\varepsilon_{0}^{HF}}=x^2-\frac{e^2m}{\hbar^2 k_F\pi}F(x)=x^2-\frac{r_s}{a_0}0.663F(x),
\]
with $r_s \sim 2-6$ for most metals. 

We can now define the so-called bandwidth, that is the scatter between the maximal and the minimal value of the electrons in the conductance band of a metal (up to the Fermi level). For $x=1$, which corresponds to 



\end{block}


% !split
\subsection{Exercises: Electron gas}
\begin{block}{Exercise 6 }
We consider a system of electrons in infinite matter, the so-called electron gas. This is a homogeneous system and the one-particle states are given by plane wave function normalized to a volume $\Omega$ 
for a box with length $L$ (the limit $L\rightarrow \infty$ is to be taken after we have computed various expectation values)
\[
\psi_{{\bf k}\sigma}({\bf r})= \frac{1}{\sqrt{\Omega}}\exp{(i{\bf kr})}\xi_{\sigma}
\]
where ${\bf k}$ is the wave number and  $\xi_{\sigma}$ is a spin function for either spin up or down
\[ 
\xi_{\sigma=+1/2}=\left(\begin{array}{c} 1 \\ 0 \end{array}\right) \hspace{0.5cm}
\xi_{\sigma=-1/2}=\left(\begin{array}{c} 0 \\ 1 \end{array}\right).
\]
We assume that we have periodic boundary conditions which limit the allowed wave numbers to
\[
k_i=\frac{2\pi n_i}{L}\hspace{0.5cm} i=x,y,z \hspace{0.5cm} n_i=0,\pm 1,\pm 2, \dots
\]
We assume first that the particles interact via a central, symmetric and translationally invariant
interaction  $V(r_{12})$ with
$r_{12}=|{\bf r}_1-{\bf r}_2|$.  The interaction is spin independent.

The total Hamiltonian consists then of kinetic and potential energy
\[
\hat{H} = \hat{T}+\hat{V}.
\]
\begin{itemize}
 \item Show that the operator for the kinetic energy can be written as
\end{itemize}

\noindent
\[
\hat{T}=\sum_{{\bf k}\sigma}\frac{\hbar^2k^2}{2m}a_{{\bf k}\sigma}^{\dagger}a_{{\bf k}\sigma}.
\]
Find also the number operator $\hat{N}$ and find a corresponding expression for the interaction
$\hat{V}$ expressed with creation and annihilation operators.   The expression for the interaction
has to be written in  $k$ space, even though $V$ depends only on the relative distance. It means that you ned to set up the Fourier transform $\langle {\bf k}_i{\bf k}_j| V | {\bf k}_m{\bf k}_n\rangle$.
\begin{itemize}
 \item We will now study the electron gas. The Hamilton operator is given by
\end{itemize}

\noindent
\[
\hat{H}=\hat{H}_{el}+\hat{H}_{b}+\hat{H}_{el-b},
\]
with the electronic part
\[
\hat{H}_{el}=\sum_{i=1}^N\frac{p_i^2}{2m}+\frac{e^2}{2}\sum_{i\ne j}\frac{e^{-\mu |{\bf r}_i-{\bf r}_j|}}{|{\bf r}_i-{\bf r}_j|},
\]
where we have introduced an explicit convergence factor
(the limit $\mu\rightarrow 0$ is performed after having calculated the various integrals).
Correspondingly, we have
\[
\hat{H}_{b}=\frac{e^2}{2}\int\int d{\bf r}d{\bf r}'\frac{n({\bf r})n({\bf r}')e^{-\mu |{\bf r}-{\bf r}'|}}{|{\bf r}-{\bf r}'|},
\]
which is the energy contribution from the positive background charge with density
$n({\bf r})=N/\Omega$. Finally,
\[
\hat{H}_{el-b}=-\frac{e^2}{2}\sum_{i=1}^N\int d{\bf r}\frac{n({\bf r})e^{-\mu |{\bf r}-{\bf x}_i|}}{|{\bf r}-{\bf x}_i|},
\]
is the interaction between the electrons and the positive background.

Show that
\[
\hat{H}_{b}=\frac{e^2}{2}\frac{N^2}{\Omega}\frac{4\pi}{\mu^2},
\]
and
\[
\hat{H}_{el-b}=-e^2\frac{N^2}{\Omega}\frac{4\pi}{\mu^2}.
\]
Show thereafter that the final Hamiltonian can be written as 
\[
H=H_{0}+H_{I},
\]
with
\[
H_{0}={\displaystyle\sum_{{\bf k}\sigma}}
\frac{\hbar^{2}k^{2}}{2m}a_{{\bf k}\sigma}^{\dagger}
a_{{\bf k}\sigma},
\]
and
\[
H_{I}=\frac{e^{2}}{2\Omega}{\displaystyle\sum_{\sigma_{1}
\sigma_{2}}}{\displaystyle
\sum_{{\bf q}\neq 0,{\bf k},{\bf p}}}\frac{4\pi}{q^{2}}
a_{{\bf k}+{\bf q},\sigma_{1}}^{\dagger}
a_{{\bf p}-{\bf q},\sigma_{2}}^{\dagger}
a_{{\bf p}\sigma_{2}}a_{{\bf k}\sigma_{1}}.
\] 
\begin{itemize}
 \item Calculate $E_0/N=\bra{\Phi_{0}}H\ket{\Phi_{0}}/N$ for for this system to first order in the interaction. Show that, by using
\end{itemize}

\noindent
\[
\rho= \frac{k_F^3}{3\pi^2}=\frac{3}{4\pi r_0^3},
\]
with $\rho=N/\Omega$, $r_0$
being the radius of a sphere representing the volume an electron occupies 
and the Bohr radius $a_0=\hbar^2/e^2m$, 
that the energy per electron can be written as 
\[
E_0/N=\frac{e^2}{2a_0}\left[\frac{2.21}{r_s^2}-\frac{0.916}{r_s}\right].
\]
Here we have defined
$r_s=r_0/a_0$ to be a dimensionless quantity.

Plot your results. Why is this system stable?
\begin{itemize}
 \item Calculate thermodynamical quantities like the pressure, given by
\end{itemize}

\noindent
\[
P=-\left(\frac{\partial E}{\partial \Omega}\right)_N,
\]
and the bulk modulus
\[
B=-\Omega\left(\frac{\partial P}{\partial \Omega}\right)_N,
\]
and comment your results.

\end{block}


% !split
\subsection{Exercises: Electron gas}
\begin{block}{Solution to exercise 6 }
We have to show first  that
\[
\hat{H}_{b}=\frac{e^2}{2}\frac{N_e^2}{\Omega}\frac{4\pi}{\mu^2},
\]
and
\[
\hat{H}_{el-b}=-e^2\frac{N_e^2}{\Omega}\frac{4\pi}{\mu^2}.
\]
And then that the final Hamiltonian can be written as 
\[
H=H_{0}+H_{I},
\]
with
\[
H_{0}={\displaystyle\sum_{{\bf k}\sigma}}
\frac{\hbar^{2}k^{2}}{2m_e}a_{{\bf k}\sigma}^{\dagger}
a_{{\bf k}\sigma},
\]
and
\[
H_{I}=\frac{e^{2}}{2\Omega}{\displaystyle\sum_{\sigma_{1}
\sigma_{2}}}{\displaystyle
\sum_{{\bf q}\neq 0,{\bf k},{\bf p}}}\frac{4\pi}{q^{2}}
a_{{\bf k}+{\bf q},\sigma_{1}}^{\dagger}
a_{{\bf p}-{\bf q},\sigma_{2}}^{\dagger}
a_{{\bf p}\sigma_{2}}a_{{\bf k}\sigma_{1}}.
\] 

\end{block}

% !split
\subsection{Electron gas and HF solution}
\begin{block}{}
Let us now calculate the following part of the Hamiltonian
\[ \hat H_b = \frac{e^2}{2} \iint \frac{n({\bf r}) n({\bf r}')e^{-\mu|{\bf r} - {\bf r}'|}}{|{\bf r} - {\bf r}'|} d{\bf r} d{\bf r}' , 
\]
where $n({\bf r}) = N_e/\Omega$, the density of the positive background charge. We define ${\bf r}_{12} = {\bf r} - {\bf r}'$, resulting in $d{\bf r}_{12} = d{\bf r}$, and allowing us to rewrite the integral as
\[ 
\hat H_b = \frac{e^2 N_e^2}{2\Omega^2} \iint \frac{e^{-\mu |{\bf r}_{12}|}}{|{\bf r}_{12}|} d{\bf r}_{12} d{\bf r}' = \frac{e^2 N_e^2}{2\Omega} \int \frac{e^{-\mu |{\bf r}_{12}|}}{|{\bf r}_{12}|} d{\bf r}_{12} . 
\]
Here we have used that $\int \! {\bf r} = \Omega$. We change to spherical coordinates and the lack of angle 
dependencies yields a factor $4\pi$, resulting in
\[ 
\hat H_b = \frac{4\pi e^2 N_e^2}{2\Omega} \int_0^\infty re^{-\mu r} \, \mathrm{d} r . 
\]


Solving by partial integration
\[ \int_0^\infty re^{-\mu r} \, \mathrm{d} r = \left[ -\frac{r}{\mu} e^{-\mu r} \right]_0^\infty + \frac{1}{\mu} \int_0^\infty e^{-\mu r} \, \mathrm{d} r
= \frac{1}{\mu} \left[ - \frac{1}{\mu} e^{-\mu r} \right]_0^\infty = \frac{1}{\mu^2}, 
\]
gives
\[
\hat{H}_b = \frac{e^2}{2} \frac{N_e^2}{\Omega} \frac{4\pi}{\mu^2} .
\]
The next term is 
\[ 
\hat H_{el-b} = -e^2 \sum_{i = 1}^N \int \frac{n({\bf r}) e^{-\mu |{\bf r} - {\bf x}_i|}}{|{\bf r} - {\bf x}_i|} {\bf r} . 
\]
Inserting  $n({\bf r})$ and changing variables in the same way as in the previous integral ${\bf y} = {\bf r} - {\bf x}_i$, we get $\mathrm{d}^3 {\bf y} = \mathrm{d}^3 {\bf r}$. This gives
\[ 
\hat H_{el-b} = -\frac{e^2 N_e}{\Omega} \sum_{i = 1^N} \int \frac{e^{-\mu |{\bf y}|}}{|{\bf y}|} \, \mathrm{d}^3 {\bf y}
=  -\frac{4\pi e^2 N_e}{\Omega} \sum_{i = 1}^N \int_0^\infty y e^{-\mu y} \mathrm{d} y. 
\]
We have already seen this  type of integral. The answer is 
\[ 
\hat H_{el-b} = -\frac{4\pi e^2 N_e}{\Omega} \sum_{i = 1}^N \frac{1}{\mu^2}, 
\]
which gives
\[
\hat H_{el-b} = -e^2 \frac{N_e^2}{\Omega} \frac{4\pi}{\mu^2} .
\]

Finally, we need to evaluate $\hat H_{el}$. This term reads
\[ 
\hat H_{el} = \sum_{i=1}^{N_e} \frac{\hat{\vec p}_i^2}{2m_e} + \frac{e^2}{2} \sum_{i \neq j} \frac{e^{-\mu |{\bf r}_i - {\bf r}_j|}}{{\bf r}_i - {\bf r}_j} . 
\]
The last term represents the repulsion between two electrons. It is a central symmetric interaction
and is translationally invariant. The potential is given by the expression
\[ 
v(|{\bf r}|) = e^2 \frac{e^{\mu|{\bf r}|}}{|{\bf r}|}, 
\]
which we derived in connection with the Hartree-Fock derivation.


\end{block}




% ------------------- end of main content ---------------


% #ifdef PREAMBLE
\printindex

\end{document}
% #endif

