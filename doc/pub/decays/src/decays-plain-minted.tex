%%
%% Automatically generated file from DocOnce source
%% (https://github.com/hplgit/doconce/)
%%
%%


%-------------------- begin preamble ----------------------

\documentclass[%
twoside,                 % oneside: electronic viewing, twoside: printing
final,                   % or draft (marks overfull hboxes, figures with paths)
10pt]{article}

\listfiles               % print all files needed to compile this document

\usepackage{relsize,makeidx,color,setspace,amsmath,amsfonts}
\usepackage[table]{xcolor}
\usepackage{bm,microtype}

\usepackage[T1]{fontenc}
%\usepackage[latin1]{inputenc}
\usepackage{ucs}
\usepackage[utf8x]{inputenc}

\usepackage{lmodern}         % Latin Modern fonts derived from Computer Modern

% Hyperlinks in PDF:
\definecolor{linkcolor}{rgb}{0,0,0.4}
\usepackage{hyperref}
\hypersetup{
    breaklinks=true,
    colorlinks=true,
    linkcolor=linkcolor,
    urlcolor=linkcolor,
    citecolor=black,
    filecolor=black,
    %filecolor=blue,
    pdfmenubar=true,
    pdftoolbar=true,
    bookmarksdepth=3   % Uncomment (and tweak) for PDF bookmarks with more levels than the TOC
    }
%\hyperbaseurl{}   % hyperlinks are relative to this root

\setcounter{tocdepth}{2}  % number chapter, section, subsection

\usepackage[framemethod=TikZ]{mdframed}

% --- begin definitions of admonition environments ---

% --- end of definitions of admonition environments ---

% prevent orhpans and widows
\clubpenalty = 10000
\widowpenalty = 10000

% --- end of standard preamble for documents ---


% insert custom LaTeX commands...

\raggedbottom
\makeindex

%-------------------- end preamble ----------------------

\begin{document}



% ------------------- main content ----------------------

% Slides for PHY981


% ----------------- title -------------------------

\thispagestyle{empty}

\begin{center}
{\LARGE\bf
\begin{spacing}{1.25}
Electromagnetic transitions and $\beta$-decay
\end{spacing}
}
\end{center}

% ----------------- author(s) -------------------------

\begin{center}
{\bf Morten Hjorth-Jensen, National Superconducting Cyclotron Laboratory and Department of Physics and Astronomy, Michigan State University, East Lansing, MI 48824, USA {\&} Department of Physics, University of Oslo, Oslo, Norway${}^{}$} \\ [0mm]
\end{center}

    \begin{center}
% List of all institutions:
\end{center}
    
% ----------------- end author(s) -------------------------

\begin{center} % date
Spring 2015
\end{center}

\vspace{1cm}


% !split
\subsection*{Electromagnetic multipole moments and transitions}

% --- begin paragraph admon ---
\paragraph{}
The reduced
transition probability $B$ is defined in terms of reduced matrix
elements of a one-body operator by
\[
B(i \rightarrow f)= \frac{\langle J_{f}||{\cal O}(\lambda )||J_{i}\rangle^{2}}{(2J_{i}+1)}. 
\]
With our definition of the reduced matrix element,
\[
\langle J_{f}||{\cal O}(\lambda )||J_{i}\rangle^{2} =\langle J_{i}||{\cal O}(\lambda )||J_{f}\rangle^{2},  
\]
the transition probability $B$ depends upon the direction of the transition by the factor
of $(2J_{i}+1)$. For electromagnetic
transitions $J_{i}$ is that for the higher-energy initial state. But in
Coulomb excitation the initial state is usually
taken as the ground state, and it is normal to use the notation $B(\uparrow)$ for transitions from the ground state.
% --- end paragraph admon ---



% !split
\subsection*{Electromagnetic multipole moments and transitions}

% --- begin paragraph admon ---
\paragraph{}
The one-body operators $  {\cal O}(\lambda )  $ represent a sum over
the operators for the individual nucleon degrees of freedom  $  i  $
\[
{\cal O}(\lambda ) = \sum_{i} O(\lambda ,i). 
\]

The electric transition operator is given by
\[
O(E\lambda ) = r^{\lambda } \; Y^{\lambda }_{\mu }(\hat{r}) \; e_{q} e, 
\]
were $Y^{\lambda }_{\mu }$ are the spherical harmonics
and $q$ stands for proton $q=p$ or neutron $q=n$.
% --- end paragraph admon ---




% !split
\subsection*{Electromagnetic multipole moments and transitions}

% --- begin paragraph admon ---
\paragraph{}
Gamma transitions
with $\lambda=0$ are forbidden because the photon must carry off
at least one unit of angular momentum. The $e_{q}$
are the electric charges for the proton and neutron in units of $  e  $.
For the free-nucleon
charge we would take $e_{p}=1$ and $e_{n}=0$, for the
proton and neutron, respectively.
Although the bare operator acts upon the protons,
we will keep the general expression in terms of $e_{q}$ in order
to incorporate the \textbf{effective charges} for the proton and
neutron, which represent the center-of-mass corrections and the
average effects of the renormalization from wavefunction
admixtures outside the model space.
% --- end paragraph admon ---



% !split
\subsection*{Electromagnetic multipole moments and transitions}

% --- begin paragraph admon ---
\paragraph{}
The magnetic transition operator is given by:
\[
O(M\lambda)=\left[\mathbf{l}\frac{2g^{l }_{q}}{(\lambda +1)}+ \mathbf{s}g^{s}_{q}\right]\mathbf{\nabla}[r^{\lambda }Y^{\lambda }_{\mu }(\hat{r})]\mu _{N}
\]
\[
= \sqrt{\lambda (2\lambda +1)}\left[[Y^{\lambda -1}(\hat{r})\otimes \mathbf{l}\,]^{\lambda }_{\mu }\frac{2g^{l}_{q}}{(\lambda +1)}
+ [Y^{\lambda -1}(\hat{r})\otimes \mathbf{s}\,]^{\lambda }_{\mu }g^{s}_{q}\right]r^{\lambda -1}\mu _{N}, 
\]
 where $\mu_{N}$ is the nuclear magneton,
\[
\mu _{N}=\frac{e\hbar }{2m_{p}c} = 0.105 \; e {\rm fm}, 
\]
and where $m_{p}$ is the mass of the proton.
% --- end paragraph admon ---



% !split
\subsection*{Electromagnetic multipole moments and transitions}

% --- begin paragraph admon ---
\paragraph{}
The g-factors $g^{l}_{q}$ and $g^{s}_{q}$
are the orbital and spin g-factors
for the proton and neutron, respectively.
The
free-nucleon values for the g-factors are $g^{l}_{p}=1$, $g^{l}_{n}=0$,
$g^{s}_{p}=5.586$ and $g^{s}_{n}=-3.826$. We may use effective values
for these g-factors to take into account the truncation of the model
space.
% --- end paragraph admon ---



% !split
\subsection*{Electromagnetic multipole moments and transitions}

% --- begin paragraph admon ---
\paragraph{}
The most common types of transitions are $E1$, $E2$ and $M1$.
The $E1$ transition operator is given by $\lambda$=1:
\[
O(E1) = rY^{(1)}_{\mu }(\hat{r})e_{q} e= \sqrt{\frac{3}{4\pi }}\mathbf{r}e_{q} e.
\]
The $E2$ transition operator with $\lambda$=2:
\[
O(E2) = r^{2}Y^{(2)}_{\mu }(\hat{r})e_{q} e,  
\]
% --- end paragraph admon ---



% !split
\subsection*{Electromagnetic multipole moments and transitions}

% --- begin paragraph admon ---
\paragraph{}
The $M1$ transition operator with $\lambda=1$ and with
\[
  Y^{0}=1/\sqrt{4\pi },
\]
we have
\[
O(M1)=\sqrt{\frac{3}{4\pi }}[\mathbf{l}g^{l }_{q}+\mathbf{s} \; g^{s}_{q}]\mu _{N}. 
\]
% --- end paragraph admon ---



% !split
\subsection*{Electromagnetic multipole moments and transitions}

% --- begin paragraph admon ---
\paragraph{}
The selection rules are given by the triangle condition for the
angular momenta, $  \Delta (J_{i},J_{f},\lambda )  $. The electromagnetic
interaction conserves parity, and the elements of the
operators for $  E\lambda   $ and $  M\lambda   $
can be classified according to their transformation under parity
change
\[
\hat{P}\hat{O}\hat{P}^{-1}=\pi_{O}\hat{O}, 
\]
where we have $\pi _{O}=(-1)^{\lambda }$ for $Y^{\lambda }$,
$\pi _{O}=-1$ for the vectors
$\mathbf{r}$, $\mathbf{\nabla}$ and $\mathbf{p}$, and $  \pi _{O}=+1  $ for the 
pseudo vectors
$\mathbf{l}=\mathbf{r}\times\mathbf{p}$ and $\mathbf{\sigma}$. For a given matrix element we have:
\[
\langle\Psi _{f}\vert {\cal O}\vert \Psi _{i}\rangle =\langle\Psi _{f}\vert P^{-1}P{\cal O}P^{-1}P\vert \Psi _{i}\rangle=\pi _{i}\pi _{f}\pi _{O} \langle\Psi _{f}\vert {\cal O}\vert \Psi _{i}\rangle. 
\]
The matrix element will vanish unless  $\pi _{i}\pi _{f}\pi _{O}=+1$.
% --- end paragraph admon ---



% !split
\subsection*{Electromagnetic multipole moments and transitions}

% --- begin paragraph admon ---
\paragraph{}
The transitions are divided into two classes, those
which do not change parity change $\pi _{i}\pi _{f}=+1$ which go by the
operators with $\pi _{O}=+1$:
\[
\pi _{i}\pi _{f}=+1 \; \mathrm{for} \; M1, E2, M3, E4 \dots,
\]
and the ones which do change parity change $\pi _{i}\pi _{f}=-1$
which go by the operators with $\pi _{O}=-1$:
\[
\pi _{i}\pi _{f}=-1 \; \mathrm{for} \; E1, M2, E3, M4 \dots.
\]
% --- end paragraph admon ---



% !split
\subsection*{Electromagnetic multipole moments and transitions}

% --- begin paragraph admon ---
\paragraph{}
The 
electromagnetic moment operator can be expressed in terms of the
electromagnetic transition operators.
By the parity selection rule of the moments are nonzero
only for $M1$, $E2$, $M3$, $E4,\ldots$.
The most common are:
\[
\mu =\sqrt{\frac{4\pi }{3}}\langle J,M=J\vert {\cal O}(M1)\vert J,M=J\rangle= \sqrt{\frac{4\pi }{3}}\left\{\begin{array}{ccc} J & 1 &J \\  -J &  0 &  J \end{array}\right\}
\langle J\vert \vert {\cal O}(M1)\vert \vert J\rangle,
\]
and
\[
Q = \sqrt{\frac{16\pi }{5}}\langle J,M=J\vert {\cal O}(E2)\vert J,M=J\rangle= \sqrt{\frac{16\pi }{5}}\,
  \left(\begin{array}{ccc}  {J}&  {2} & {J}\\  {-J} & {0}&  {J}\end{array}\right)\langle J\vert\vert {\cal O}(E2)\vert\vert J\rangle.
\]
% --- end paragraph admon ---



% !split
\subsection*{Electromagnetic multipole moments and transitions}

% --- begin paragraph admon ---
\paragraph{}
Electromagnetic transitions and moments depend upon the reduced nuclear
matrix elements $\langle f\vert\vert {\cal O}(\lambda )\vert\vert i\rangle$. These can be expressed as a sum over one-body transition
densities (OBTD) times single-particle matrix elements
\[
\langle f\vert\vert {\cal O}(\lambda )\vert\vert i\rangle
=\sum _{k_{\alpha } k_{\beta }}\mathrm{OBTD}(f i k_{\alpha } k_{\beta } \lambda )
 \langle k_{\alpha }\vert\vert O(\lambda )\vert\vert k_{\beta }\rangle, 
\]
where the OBTD is given by
\[
\mathrm{OBTD}(f i k_{\alpha} k_{\beta}\lambda)= \frac{\langle f\vert\vert [a^{+}_{k_{\alpha }}\otimes \tilde{a}_{k_{\beta }}]^{\lambda }\vert\vert i\rangle}{\sqrt{(2\lambda +1)}}. 
\]
The labels $i$ and $f$ are a short-hand notation for the initial
and final state quantum numbers $(n \omega _{i}J_{i})$ and $(n\omega_{f}J_{f})$,
respectively. Thus the problem is divided into two parts, one
involving the nuclear structure dependent one-body transition
densities OBTD, and the other involving the reduced
single-particle matrix
elements (SPME).
% --- end paragraph admon ---





% !split
\subsection*{Electromagnetic multipole moments and transitions}

% --- begin paragraph admon ---
\paragraph{}
The SPME for the $  E\lambda   $ operator is given by:
\[
\langle k_{a}\vert\vert O(E\lambda )\vert\vert k_{b}\rangle=(-1)^{j_{a}+1/2}\frac{[1+(-1)^{l_{a}+\lambda +l_{b}}]}{2}
\]
\[
 \times\sqrt{ {(2j_{a}+1)(2\lambda +1)(2j_{b}+1)\over4\pi }}\left(\begin{array}{ccc}  {j_{a}}&  {\lambda} &  {j_{b}}\\  {1/2} & {0}&  {-1/2}\end{array}\right)\langle k_{a}\vert r^{\lambda }\vert k_{b}\rangle e_{q} e.
\]
% --- end paragraph admon ---



% !split
\subsection*{Electromagnetic multipole moments and transitions}

% --- begin paragraph admon ---
\paragraph{}
The SPME for the spin part of the magnetic operator is
\[
\langle k_{a}\vert\vert O(M\lambda ,s)\vert\vert k_{b}\rangle =
\]
\[
=\sqrt{\lambda (2\lambda +1)}<j_{a}\vert\vert [Y^{\lambda -1}(\hat{r})\otimes\mathbf{s}\,]^{\lambda }\vert\vert j_{b}><k_{a}\vert r^{\lambda -1}\vert k_{b}>g^{s}_{q}\mu _{N},
\]
\[
= \sqrt{\lambda (2\lambda +1)}\, \sqrt{(2j_{a}+1)(2j_{b}+1)(2\lambda +1)}\left\{\begin{array}{ccc}  {l _{a}}&  {1/2} & {j_{a}}\\  {l _{b}}&  {1/2} & {j_{b}}\\  {\lambda -1} & {1} & {\lambda}\end{array}\right\}
\]
\[
\times \langle l _{a}\vert\vert Y^{\lambda -1}(\hat{r})\vert\vert l _{b}\rangle\langle\vert\vert \mathbf{s}\vert\vert s\rangle\langle k_{a}\vert r^{\lambda -1}\vert k_{b}\rangle g^{s}_{q}\mu _{N},
\]
where
\[
\langle\vert\vert \mathbf{s}\vert\vert s\rangle = \sqrt{3/2}.
\]
% --- end paragraph admon ---



% !split
\subsection*{Electromagnetic multipole moments and transitions}

% --- begin paragraph admon ---
\paragraph{}
The SPME for the orbital part of the magnetic operator is:
\[
\langle k_{a}\vert\vert O(M\lambda ,l )\vert\vert k_{b}\rangle=
\]
\[
= \frac{\sqrt{\lambda (2\lambda +1)}\, }{\lambda +1}
\langle j_{a}\vert\vert [Y^{\lambda -1}(\hat{r})\otimes\mathbf{l}\,]^{\lambda }\vert\vert j_{b}\rangle
\langle k_{a}\vert r^{\lambda -1}\vert k_{b}\rangle g^{l }_{q}\mu _{N}
\]
\[
=\frac{\sqrt{\lambda (2\lambda +1)}\, }{\lambda +1}(-1)^{l _{a}+1/2+j_{b}+\lambda } \sqrt{(2j_{a}+1)(2j_{b}+1)}
\]
\[
\times\left\{\begin{array}{ccc}  {l _{a}} &  {l _{b}} & {\lambda} \\  {j_{b}}&  {j_{a}}&  {1/2}\end{array}\right\}
\langle l _{a}\vert\vert [Y^{\lambda -1}(\hat{r})\otimes\mathbf{l}\,]^{\lambda }\vert\vert l _{b}\rangle
\langle k_{a}\vert r^{\lambda -1}\vert k_{b}\rangle g^{l }_{q}\mu _{N}, 
\]
% --- end paragraph admon ---



% !split
\subsection*{Electromagnetic multipole moments and transitions}

% --- begin paragraph admon ---
\paragraph{}
where we have defined
\[
\langle l _{a}\vert\vert [Y^{\lambda -1}(\hat{r})\otimes\mathbf{l}]^{\lambda }\vert\vert l _{b}\rangle=(-1)^{\lambda +l _{a}+l _{b}} \sqrt{(2\lambda +1)l _{b}(l _{b}+1)(2l _{b}+1)}
\]
\[
\times\left\{\begin{array}{ccc}  {\lambda -1} & {1}&  {\lambda}\\   {l _{b}}&  {l _{a}} &  {l _{b}} \end{array}\right\}
\langle l _{a}\vert\vert Y^{\lambda -1}(\hat{r})\vert\vert l _{b}\rangle,
\]
with
\[
\langle l _{a}\vert\vert Y^{\lambda -1}(\hat{r})\vert\vert l _{b}\rangle=(-1)^{l _{a}} \sqrt{\frac{(2l _{a}+1)(2l _{b}+1)(2\lambda -1)}{4\pi }}\left(\begin{array}{ccc}  {l _{a}} & {\lambda -1} & {l _{b}}\\  {0} & {0}&  {0}\end{array}\right)
. 
\]
% --- end paragraph admon ---



% !split
\subsection*{Electromagnetic multipole moments and transitions}

% --- begin paragraph admon ---
\paragraph{}
For the $M1$ operator the radial matrix element is
\[
<k_{a}\vert r^{0}\vert k_{b}>\, = \delta _{n_{a},n_{b}},
\]
and the SPME simplify to:
\[
\langle k_{a}\vert\vert O(M1,s)\vert\vert k_{b}\rangle=\sqrt{ \frac{3}{4\pi }}\langle j_{a}\vert\vert \mathbf{s}\,\vert\vert j_{b}\rangle \delta _{n_{a},n_{b}}g^{s}_{q}\mu_{N}
\]
\[
=\sqrt{ \frac{3}{4\pi }}(-1)^{l _{a}+j_{a}+3/2}
\sqrt{(2j_{a}+1)(2j_{b}+1)}\left\{\begin{array}{ccc}  {1/2}&  {1/2} & {1} \\ {j_{b}} & {j_{a}}&  {l _{a}}\end{array}\right\}
\]
\[
\times\langle s\vert\vert \mathbf{s}\vert\vert s\rangle \delta _{l _{a},l _{b}} \delta _{n_{a},n_{b}}g^{s}_{q}\mu _{N},
\]
% --- end paragraph admon ---



% !split
\subsection*{Electromagnetic multipole moments and transitions}

% --- begin paragraph admon ---
\paragraph{}
where we have
\[
<s\vert\vert \mathbf{s}\,\vert\vert s>\, = \sqrt{3/2}\, ,
\]
and
\[
<k_{a}\vert\vert O(M1,l )\vert\vert k_{b}>\, =
\sqrt{ \frac{3}{4\pi }}\,
<j_{a}\vert\vert \mathbf{l}\,\vert\vert j_{b}> \delta _{n_{a},n_{b}} \; g^{l }_{q} \; \mu _{N}
\]
\[
=\sqrt{ \frac{3}{4\pi }}\,
(-1)^{l _{a}+j_{b}+3/2} \sqrt{(2j_{a}+1)(2j_{b}+1)}\,
   \left\{\begin{array}{ccc}  {l _{a}}&  {l _{b}} & {1}\\  {j_{b}}&  {j_{a}}&  {1/2}\end{array}\right\}
\]
\[
\times\langle l _{a}\vert\vert \mathbf{l}\,\vert\vert l _{b}\rangle \delta _{n_{a},n_{b}}g^{l }_{q}\mu _{N} ,       
\]
where
\[
\langle l _{a}\vert\vert \mathbf{l}\,\vert\vert l _{b}\rangle = \delta _{l _{a},l _{b}} \sqrt{l _{a}(l_{a}+1)(2l _{a}+1)}.
\]
Thus the $M1$ operator connects only
those orbitals which have the same $n$ and $l$ values.
% --- end paragraph admon ---



% !split
\subsection*{Electromagnetic multipole moments and transitions}

% --- begin paragraph admon ---
\paragraph{}
For further reading, see Suhonen's chapters 6-7 and Alex Brown's chapters 23, 28 and 29. 
Before we proceed wit $\beta$-decay, we need to say something about so-called core-polarization effects.
To do this, we have to introduce elements from many-body perturbation theory.

We assume here that we are only interested in the ground state of the system and 
expand the exact wave function in term of a series of Slater determinants
\[
\vert \Psi_0\rangle = \vert \Phi_0\rangle + \sum_{m=1}^{\infty}C_m\vert \Phi_m\rangle,
\]
where we have assumed that the true ground state is dominated by the 
solution of the unperturbed problem, that is
\[
\hat{H}_0\vert \Phi_0\rangle= W_0\vert \Phi_0\rangle.
\]
The state $\vert \Psi_0\rangle$ is not normalized, rather we have used an intermediate 
normalization $\langle \Phi_0 \vert \Psi_0\rangle=1$ since we have $\langle \Phi_0\vert \Phi_0\rangle=1$.
% --- end paragraph admon ---



% !split
\subsection*{Electromagnetic multipole moments and transitions}

% --- begin paragraph admon ---
\paragraph{}
The Schroedinger equation is
\[
\hat{H}\vert \Psi_0\rangle = E\vert \Psi_0\rangle,
\]
and multiplying the latter from the left with $\langle \Phi_0\vert $ gives
\[
\langle \Phi_0\vert \hat{H}\vert \Psi_0\rangle = E\langle \Phi_0\vert \Psi_0\rangle=E,
\]
and subtracting from this equation
\[
\langle \Psi_0\vert \hat{H}_0\vert \Phi_0\rangle= W_0\langle \Psi_0\vert \Phi_0\rangle=W_0,
\]
and using the fact that the both operators $\hat{H}$ and $\hat{H}_0$ are hermitian 
results in
\[
\Delta E=E-W_0=\langle \Phi_0\vert \hat{H}_I\vert \Psi_0\rangle,
\]
which is an exact result. We call this quantity the correlation energy.
% --- end paragraph admon ---



% !split
\subsection*{Electromagnetic multipole moments and transitions}

% --- begin paragraph admon ---
\paragraph{}
This equation forms the starting point for all perturbative derivations. However,
as it stands it represents nothing but a mere formal rewriting of Schroedinger's equation and is not of much practical use. The exact wave function $\vert \Psi_0\rangle$ is unknown. In order to obtain a perturbative expansion, we need to expand the exact wave function in terms of the interaction $\hat{H}_I$. 

Here we have assumed that our model space defined by the operator $\hat{P}$ is one-dimensional, meaning that
\[
\hat{P}= \vert \Phi_0\rangle \langle \Phi_0\vert ,
\]
and
\[
\hat{Q}=\sum_{m=1}^{\infty}\vert \Phi_m\rangle \langle \Phi_m\vert .
\]
% --- end paragraph admon ---



% !split
\subsection*{Electromagnetic multipole moments and transitions}

% --- begin paragraph admon ---
\paragraph{}
We can thus rewrite the exact wave function as
\[
\vert \Psi_0\rangle= (\hat{P}+\hat{Q})\vert \Psi_0\rangle=\vert \Phi_0\rangle+\hat{Q}\vert \Psi_0\rangle.
\]
Going back to the Schr\"odinger equation, we can rewrite it as, adding and a subtracting a term $\omega \vert \Psi_0\rangle$ as
\[
\left(\omega-\hat{H}_0\right)\vert \Psi_0\rangle=\left(\omega-E+\hat{H}_I\right)\vert \Psi_0\rangle,
\]
where $\omega$ is an energy variable to be specified later.
% --- end paragraph admon ---



% !split
\subsection*{Electromagnetic multipole moments and transitions}

% --- begin paragraph admon ---
\paragraph{}
We assume also that the resolvent of $\left(\omega-\hat{H}_0\right)$ exits, that is
it has an inverse which defined the unperturbed Green's function as
\[
\left(\omega-\hat{H}_0\right)^{-1}=\frac{1}{\left(\omega-\hat{H}_0\right)}.
\]

We can rewrite Schroedinger's equation as
\[
\vert \Psi_0\rangle=\frac{1}{\omega-\hat{H}_0}\left(\omega-E+\hat{H}_I\right)\vert \Psi_0\rangle,
\]
and multiplying from the left with $\hat{Q}$ results in
\[
\hat{Q}\vert \Psi_0\rangle=\frac{\hat{Q}}{\omega-\hat{H}_0}\left(\omega-E+\hat{H}_I\right)\vert \Psi_0\rangle,
\]
which is possible since we have defined the operator $\hat{Q}$ in terms of the eigenfunctions of $\hat{H}$.
% --- end paragraph admon ---



% !split
\subsection*{Electromagnetic multipole moments and transitions}

% --- begin paragraph admon ---
\paragraph{}
These operators commute meaning that
\[
\hat{Q}\frac{1}{\left(\omega-\hat{H}_0\right)}\hat{Q}=\hat{Q}\frac{1}{\left(\omega-\hat{H}_0\right)}=\frac{\hat{Q}}{\left(\omega-\hat{H}_0\right)}.
\]
With these definitions we can in turn define the wave function as 
\[
\vert \Psi_0\rangle=\vert \Phi_0\rangle+\frac{\hat{Q}}{\omega-\hat{H}_0}\left(\omega-E+\hat{H}_I\right)\vert \Psi_0\rangle.
\]
This equation is again nothing but a formal rewrite of Schr\"odinger's equation
and does not represent a practical calculational scheme.  
It is a non-linear equation in two unknown quantities, the energy $E$ and the exact
wave function $\vert \Psi_0\rangle$. We can however start with a guess for $\vert \Psi_0\rangle$ on the right hand side of the last equation.
% --- end paragraph admon ---



% !split
\subsection*{Electromagnetic multipole moments and transitions}

% --- begin paragraph admon ---
\paragraph{}
 The most common choice is to start with the function which is expected to exhibit the largest overlap with the wave function we are searching after, namely $\vert \Phi_0\rangle$. This can again be inserted in the solution for $\vert \Psi_0\rangle$ in an iterative fashion and if we continue along these lines we end up with
\[
\vert \Psi_0\rangle=\sum_{i=0}^{\infty}\left\{\frac{\hat{Q}}{\omega-\hat{H}_0}\left(\omega-E+\hat{H}_I\right)\right\}^i\vert \Phi_0\rangle, 
\]
for the wave function and
\[
\Delta E=\sum_{i=0}^{\infty}\langle \Phi_0\vert \hat{H}_I\left\{\frac{\hat{Q}}{\omega-\hat{H}_0}\left(\omega-E+\hat{H}_I\right)\right\}^i\vert \Phi_0\rangle, 
\]
which is now  a perturbative expansion of the exact energy in terms of the interaction
$\hat{H}_I$ and the unperturbed wave function $\vert \Psi_0\rangle$.
% --- end paragraph admon ---



% !split
\subsection*{Electromagnetic multipole moments and transitions}

% --- begin paragraph admon ---
\paragraph{}
In our equations for $\vert \Psi_0\rangle$ and $\Delta E$ in terms of the unperturbed
solutions $\vert \Phi_i\rangle$  we have still an undetermined parameter $\omega$
and a dependecy on the exact energy $E$. Not much has been gained thus from a practical computational point of view. 

In Brilluoin-Wigner perturbation theory it is customary to set $\omega=E$. This results in the following perturbative expansion for the energy $\Delta E$
\[
\Delta E=\sum_{i=0}^{\infty}\langle \Phi_0\vert \hat{H}_I\left\{\frac{\hat{Q}}{\omega-\hat{H}_0}\left(\omega-E+\hat{H}_I\right)\right\}^i\vert \Phi_0\rangle=
\]
\[
\langle \Phi_0\vert \left(\hat{H}_I+\hat{H}_I\frac{\hat{Q}}{E-\hat{H}_0}\hat{H}_I+
\hat{H}_I\frac{\hat{Q}}{E-\hat{H}_0}\hat{H}_I\frac{\hat{Q}}{E-\hat{H}_0}\hat{H}_I+\dots\right)\vert \Phi_0\rangle. 
\]
% --- end paragraph admon ---



% !split
\subsection*{Electromagnetic multipole moments and transitions}

% --- begin paragraph admon ---
\paragraph{}
\[
\Delta E=\sum_{i=0}^{\infty}\langle \Phi_0\vert \hat{H}_I\left\{\frac{\hat{Q}}{\omega-\hat{H}_0}\left(\omega-E+\hat{H}_I\right)\right\}^i\vert \Phi_0\rangle=\]
\[
\langle \Phi_0\vert \left(\hat{H}_I+\hat{H}_I\frac{\hat{Q}}{E-\hat{H}_0}\hat{H}_I+
\hat{H}_I\frac{\hat{Q}}{E-\hat{H}_0}\hat{H}_I\frac{\hat{Q}}{E-\hat{H}_0}\hat{H}_I+\dots\right)\vert \Phi_0\rangle. 
\]
This expression depends however on the exact energy $E$ and is again not very convenient from a practical point of view. It can obviously be solved iteratively, by starting with a guess for  $E$ and then solve till some kind of self-consistency criterion has been reached. 

Actually, the above expression is nothing but a rewrite again of the full Schr\"odinger equation.
% --- end paragraph admon ---


Defining $e=E-\hat{H}_0$ and recalling that $\hat{H}_0$ commutes with 
$\hat{Q}$ by construction and that $\hat{Q}$ is an idempotent operator
$\hat{Q}^2=\hat{Q}$. 
Using this equation in the above expansion for $\Delta E$ we can write the denominator 
\[
\hat{Q}\frac{1}{\hat{e}-\hat{Q}\hat{H}_I\hat{Q}}=
\]
\[
\hat{Q}\left[\frac{1}{\hat{e}}+\frac{1}{\hat{e}}\hat{Q}\hat{H}_I\hat{Q}
\frac{1}{\hat{e}}+\frac{1}{\hat{e}}\hat{Q}\hat{H}_I\hat{Q}
\frac{1}{\hat{e}}\hat{Q}\hat{H}_I\hat{Q}\frac{1}{\hat{e}}+\dots\right]\hat{Q}.
\]
% !split
\subsection*{Electromagnetic multipole moments and transitions}

% --- begin paragraph admon ---
\paragraph{}
Inserted in the expression for $\Delta E$ leads to 
\[
\Delta E=
\langle \Phi_0\vert \hat{H}_I+\hat{H}_I\hat{Q}\frac{1}{E-\hat{H}_0-\hat{Q}\hat{H}_I\hat{Q}}\hat{Q}\hat{H}_I\vert \Phi_0\rangle. 
\]
In RS perturbation theory we set $\omega = W_0$ and obtain the following expression for the energy difference
\[
\Delta E=\sum_{i=0}^{\infty}\langle \Phi_0\vert \hat{H}_I\left\{\frac{\hat{Q}}{W_0-\hat{H}_0}\left(\hat{H}_I-\Delta E\right)\right\}^i\vert \Phi_0\rangle=
\]
\[
\langle \Phi_0\vert \left(\hat{H}_I+\hat{H}_I\frac{\hat{Q}}{W_0-\hat{H}_0}(\hat{H}_I-\Delta E)+
\hat{H}_I\frac{\hat{Q}}{W_0-\hat{H}_0}(\hat{H}_I-\Delta E)\frac{\hat{Q}}{W_0-\hat{H}_0}(\hat{H}_I-\Delta E)+\dots\right)\vert \Phi_0\rangle.
\]
% --- end paragraph admon ---



% !split
\subsection*{Electromagnetic multipole moments and transitions}

% --- begin paragraph admon ---
\paragraph{}
Recalling that $\hat{Q}$ commutes with $\hat{H_0}$ and since $\Delta E$ is a constant we obtain that
\[
\hat{Q}\Delta E\vert \Phi_0\rangle = \hat{Q}\Delta E\vert \hat{Q}\Phi_0\rangle = 0.
\]
Inserting this results in the expression for the energy results in
\[
\Delta E=\langle \Phi_0\vert \left(\hat{H}_I+\hat{H}_I\frac{\hat{Q}}{W_0-\hat{H}_0}\hat{H}_I+
\hat{H}_I\frac{\hat{Q}}{W_0-\hat{H}_0}(\hat{H}_I-\Delta E)\frac{\hat{Q}}{W_0-\hat{H}_0}\hat{H}_I+\dots\right)\vert \Phi_0\rangle.
\]
% --- end paragraph admon ---



% !split
\subsection*{Electromagnetic multipole moments and transitions}

% --- begin paragraph admon ---
\paragraph{}
We can now this expression in terms of a perturbative expression in terms
of $\hat{H}_I$ where we iterate the last expression in terms of $\Delta E$
\[
\Delta E=\sum_{i=1}^{\infty}\Delta E^{(i)}.
\]
We get the following expression for $\Delta E^{(i)}$
\[
\Delta E^{(1)}=\langle \Phi_0\vert \hat{H}_I\vert \Phi_0\rangle,
\] 
which is just the contribution to first order in perturbation theory,
\[
\Delta E^{(2)}=\langle\Phi_0\vert \hat{H}_I\frac{\hat{Q}}{W_0-\hat{H}_0}\hat{H}_I\vert \Phi_0\rangle, 
\]
which is the contribution to second order.
% --- end paragraph admon ---



% !split
\subsection*{Electromagnetic multipole moments and transitions}

% --- begin paragraph admon ---
\paragraph{}
\[
\Delta E^{(3)}=\langle \Phi_0\vert \hat{H}_I\frac{\hat{Q}}{W_0-\hat{H}_0}\hat{H}_I\frac{\hat{Q}}{W_0-\hat{H}_0}\hat{H}_I\Phi_0\rangle-
\langle\Phi_0\vert \hat{H}_I\frac{\hat{Q}}{W_0-\hat{H}_0}\langle \Phi_0\vert \hat{H}_I\vert \Phi_0\rangle\frac{\hat{Q}}{W_0-\hat{H}_0}\hat{H}_I\vert \Phi_0\rangle,
\]
being the third-order contribution.
% --- end paragraph admon ---



% !split
\subsection*{Electromagnetic multipole moments and transitions}

% --- begin paragraph admon ---
\paragraph{}
Let us analyse a given contribution to first first order in perturbation, that is, the contribution includes (more material to come)
% --- end paragraph admon ---



% !split
\subsection*{$\beta$-decay}

% --- begin paragraph admon ---
\paragraph{}
We will now focus on  allowed $\beta$-decay.  
Suhonen's chapter 7 and Alex Brown's chapter 29 cover much of the material to be discussed on $\beta$-decay.
% --- end paragraph admon ---



% !split
\subsection*{$\beta$-decay}

% --- begin paragraph admon ---
\paragraph{}
The allowed beta decay rate $W$ between a specific set of
initial and final states is given by
\[
W_{i,f} = (f/K_{o}) \left[ g_{V}^{2} \; B_{i,f}(F_{\pm})+ g_{A}^{2}B_{i,f}(GT_{\pm})\right], 
\]
where $f$ is dimensionless three-body
phase-space factor which depends upon the
beta-decay $Q$ value,
and $K_{o}$ is a specific combination of fundamental constants
\[
  K_{o}=\frac{2\pi^{3}\hbar^{7}}{ m_{e}^{5} c^{4}}= 1.8844 \times 10^{-94}\mathrm{erg}^{2}\mathrm{cm}^{6}\mathrm{s}. 
\]
The $\pm$ signrefer to $\beta_{\pm}$ decay of nucleus
$(A_{i},Z_{i})$ into nucleus $(A_{i},Z_{i} \mp 1)  $.
The weak-interaction vector ($V$) and axial-vector ($A$) coupling
constants for the decay of neutron into a proton are denoted by $g_{V}$
and $g_{A}$, respectively.
% --- end paragraph admon ---



% !split
\subsection*{$\beta$-decay}

% --- begin paragraph admon ---
\paragraph{}
The total decay rate for a given
initial state is obtained by summing the partial rates over all
final states
\[
W = \displaystyle\sum _{f} W_{if}, 
\]
with the branching fraction to a specific final state given by
\[
b_{if} = \frac{W_{if}}{W}.  
\]
Beta decay lifetime are usually given in terms of the half-life with
a total half-life of
\[
T_{1/2} = \frac{{\rm ln}(2)}{W}.
\]
The partial half-life for a particular final state will be
denoted by $  t_{1/2}  $
\[
t_{1/2} = \frac{T_{1/2}}{b_{if}}. 
\]
% --- end paragraph admon ---



% !split
\subsection*{$\beta$-decay}

% --- begin paragraph admon ---
\paragraph{}
Historically
one combines the partial half-life for a particular decay
with the calculated
phase-space factor $f$ to obtain  an \textbf{ft} value given by
\[
  f t_{1/2}=\frac{C }{\left[B(F_{\pm})+(g_{A}/g_{V})^{2} B(GT_{\pm}) \right] }  
\]
where
\[
  C = \frac{{\rm ln}(2)K_{o}}{(g_{V})^{2}}. 
\]
% --- end paragraph admon ---



% !split
\subsection*{$\beta$-decay}

% --- begin paragraph admon ---
\paragraph{}
One often compiles the allowed beta decay in terms of a \textbf{logft}
which stands for log$_{10}$ of the $ft_{1/2}$ value.

     The values of the coupling constants for Fermi decay,
$g_{V}$, and Gamow-Teller decay, $g_{A}$ are obtained as follows.
For a $0^{+} \rightarrow  0^{+}$ nuclear transition $B(GT)=0$, and for a
transition between $  T=1  $ analogue states with $B(F)=2$ we find 
\[
       C = 2 t_{1/2} f.  
\]
The partial half-lives and $Q$ values for several $0^{+} \rightarrow  0^{+}$ analogue
transitions have been measured to an accuracy of about one part in
10000. With  phase space factors one obtains 
\[
      C = 6170(4)    
\]
This result, together with the value of $K_{o}$ can be used to obtain $g_{V}$.
% --- end paragraph admon ---



% !split
\subsection*{$\beta$-decay}

% --- begin paragraph admon ---
\paragraph{}
At the quark level $g_{V}=-g_{A}$.
But for nuclear structure we use the value obtained from the
neutron to proton beta decay
\[
     \vert g_{A}/g_{V}\vert  = 1.261(8). 
\]
% --- end paragraph admon ---



% !split
\subsection*{$\beta$-decay}

% --- begin paragraph admon ---
\paragraph{}
The operator for Fermi beta decay in terms of sums
over the nucleons is
\[
{\cal O}(F_{\pm}) =  \displaystyle\sum _{k} t_{k\pm}.
\]
The matrix element is
\[
B(F) =\, \vert \langle f\vert  T_{\pm} \vert i\rangle\vert ^{2}, 
\]
where
\[
T_{\pm} = \sum _{k} t_{\pm}  
\]
is the total isospin raising and lowering operator for total
isospin constructed out of the
basic nucleon isospin raising and lowering operators
\[
t_{-}\vert n\rangle=\vert p\rangle \hspace{1cm} t_{-}\vert p\rangle= 0,
\]
and
\[
t_{+}\vert p\rangle=\vert n\rangle, \;\; t_{+}\vert n\rangle= 0.  
\]
% --- end paragraph admon ---



% !split
\subsection*{$\beta$-decay}

% --- begin paragraph admon ---
\paragraph{}
The matrix elements obey the triangle
conditions $J_{f}=J_{i}$ ($\Delta J=0$). The Fermi operator has
$\pi _{O}=+1$, and thus the
initial and final nuclear states must have $\pi _{i}\pi _{f}=+1$ for
the matrix element to be
nonzero under the parity transform.

When isospin is conserved
the Fermi matrix element must obey the isospin triangle condition
$T_{f}=T_{i}$ $(\Delta T=0)$, and the Fermi operator can only connect
isobaric analogue states.
% --- end paragraph admon ---



% !split
\subsection*{$\beta$-decay}

% --- begin paragraph admon ---
\paragraph{}
For $\beta_{-}$ decay
\[
T_{-}\vert \omega _{i},J_{i},M_{i},T_{i},T_{zi}\rangle
\]
\[
= \sqrt{(T_{i}(T_{i}+1)-T_{zi}(T_{zi}-1)}\vert \omega _{i},J_{i},M_{i},T_{i},T_{zi}-1\rangle,      
\]
and
\[
B(F_{-}) =\, \vert \langle \omega _{f},J_{f},M_{f},T_{f},T_{zi}-1\vert T_{-}\vert \omega 
_{i},J_{i},M_{i},T_{i},T_{zi}\rangle\vert ^{2}
\]
\[
= [T_{i}(T_{i}+1)-T_{zi}(T_{zi}-1)]
\delta _{\omega _{f},\omega }\;\delta _{J_{i},J_{f}}\;\delta _{M_{i},M_{f}}\;\delta _{T_{i},T_{f}}. 
\]
% --- end paragraph admon ---



% !split
\subsection*{$\beta$-decay}

% --- begin paragraph admon ---
\paragraph{}
For $\beta_{ + }$ we have
\[
B(F_{+}) =\, \vert \langle \omega _{f},J_{f},M_{f},T_{f},T_{zi}+1\vert T_{+}\vert \omega 
_{i},J_{i},M_{i},T_{i},T_{zi}\rangle\vert ^{2}
\]
\[
= [T_{i}(T_{i}+1)-T_{zi}(T_{zi}+1)]
\delta _{\omega _{f},\omega }\;\delta _{J_{i},J_{f}}\delta _{M_{i},M_{f}}\delta _{T_{i},T_{f}}. 
\]
For neutron-rich nuclei ($N_{i}> Z_{i}$) we have $T_{i}=T_{zi}$ and thus
\[
B(F_{-})(N_{i}> Z_{i}) = 2T_{zi} = (N_{i}-Z_{i})\delta _{\omega _{f},\omega }\delta _{J_{i},J_{f}}\;\delta _{M_{i},M_{f}}\;\delta _{T_{i},T_{f}}, 
\]
and
\[
B(F_{+})(N_{i}> Z_{i}) = 0. 
\]
% --- end paragraph admon ---



% !split
\subsection*{$\beta$-decay}

% --- begin paragraph admon ---
\paragraph{}
The reduced single-particle
matrix elements are given by
\[
\langle k_{a},p\vert\vert \sigma  t_{-}\vert\vert k_{b},n\rangle=\langle k_{a},n\vert\vert \sigma  t_{+}\vert\vert k_{b},p\rangle= 2\langle k_{a}\vert\vert \mathbf{s}\vert\vert k_{b}\rangle, 
\]
where the matrix elements of $\mathbf{s}$ are given by
\[
\langle k_{a}\vert\vert \mathbf{s}\vert\vert k_{b}\rangle=\langle j_{a}\vert\vert \mathbf{s}\vert\vert j_{b}\rangle \delta_{n_{a},n_{b}}
\]
\[
=(-1)^{l_{a}+j_{a}+3/2}
\sqrt{(2j_{a}+1)(2j_{b}+1)}\left\{\begin{array}{ccc}  {1/2}&  {1/2}&  {1} \\ {j_{b}}&  {j_{a}}&  {l_{a}}\end{array}\right\}
\langle s\vert\vert \mathbf{s}\,\vert\vert s\rangle \delta _{\ell _{a},\ell _{b}} \delta_{n_{a},n_{b}}
,
\]
with
\[
\langle s\vert\vert \mathbf{s}\vert\vert s\rangle= \sqrt{3/2}.
\]
% --- end paragraph admon ---



% !split
\subsection*{$\beta$-decay}

% --- begin paragraph admon ---
\paragraph{}
The matrix elements of $\mathbf{s}$ has the selection rules $\delta_{ \ell_{a} , \ell_{b} }$
and $\delta_{n _{a} ,n _{b} }$. Thus the orbits which are connected by the GT operator
are very selective; they are those in the same major oscillator shell
with the same $\ell$ value. The matrix elements such as
$1s_{1/2}-0d_{3/2}$ which have the allowed $\Delta j$ coupling but
are zero due to the $\Delta\ell$ coupling are called $\ell$-\textbf{forbidden} matrix
elements.
% --- end paragraph admon ---



% !split
\subsection*{$\beta$-decay}

% --- begin paragraph admon ---
\paragraph{}
Sum rules for Fermi and Gamow-Teller matrix elements can be obtained easily.

The sum rule for Fermi is obtained from the sum
\[
\sum _{f} \left[ B_{fi}(F_{-}) - B_{fi}(F_{+}) \right]
=\sum _{f} \left[ \vert \langle f\vert  T_{-} \vert i\rangle\vert ^{2} -  \vert \langle f\vert  T_{+} \vert 
i\rangle\vert ^{2} \right]
\]
The final states $f$ in the $T_{-}$ matrix element go
with the $Z_{f}=Z_{i}+1$ nucleus and those in the $T_{+}$ matrix element
to with the $Z_{f}=Z_{i}-1$ nucleus. One can explicitly sum over the
final states to obtain
\[
\sum _{f} \left[ \langle i\vert  T_{+} \vert f\rangle \langle f\vert T_{-}\vert i\rangle -  \langle i\vert  T_{-} \vert f\rangle \langle f\vert T_{+}\vert i\rangle \right]
\]
\[
= \langle i\vert  T_{+} T_{-} -  T_{-} T_{+}\vert i\rangle =\langle i\vert  2T_{z}\vert i\rangle  = (N_{i}-Z_{i}).       
\]
% --- end paragraph admon ---



% !split
\subsection*{$\beta$-decay}

% --- begin paragraph admon ---
\paragraph{}
The sum rule for Gamow-Teller is obtained as follows
\[
\sum_{f,\mu}  \vert \langle f\vert  \sum_{k} \sigma_{k,\mu} t_{k-} \vert i\rangle\vert^{2}
- \sum_{f,\mu} \vert \langle f\vert  \sum_{k} \sigma_{k,\mu } t_{k+} \vert i\rangle\vert^{2}
\]
\[
= \sum_{f,\mu}\langle i\vert  \sum_{k} \sigma_{k,\mu} t_{k+} \vert f\rangle\langle f\vert  \sum_{k'} 
\sigma_{k',\mu} t_{k'-} \vert i\rangle
\]
\[
-  \sum_{f,\mu}
\langle i\vert  \sum_{k} \sigma_{k,\mu } t_{k-} \vert f\rangle\langle f\vert\sum_{k'} 
\sigma_{k',\mu } t_{k'+} \vert i\rangle
\]
\[
= \sum_{\mu} \left[\langle i\vert  \left(\sum _{k} \sigma_{k,\mu} t_{k+} \right)
    \left( \sum_{k'} \sigma_{k',\mu} t_{k'-}\right)
-   \left( \sum_{k} \sigma_{k,\mu} t_{k-} \right)
\left( \sum_{k'} \sigma_{k',\mu} t_{k'+} \right) \vert i\rangle
\right]
\]
\[
= \sum_{\mu }
\langle i\vert  \sum_{k}  \sigma ^{2}_{k,\mu } \left[ t_{k+} t_{k-} - t_{k-} t_{k+} \right] 
\vert i\rangle
= 3
\langle i\vert  \sum_{k} \left[ t_{k+} t_{k-} - t_{k-} t_{k+} \right] \vert i\rangle
\]
\[
=   3\langle i\vert  T_{+} T_{-} -  T_{-} T_{+}\vert i\rangle= 3\langle i\vert  2T_{z}\vert i\rangle  = 3(N_{i}-Z_{i}). 
\]
% --- end paragraph admon ---



% !split
\subsection*{$\beta$-decay}

% --- begin paragraph admon ---
\paragraph{}
We have used the fact that $\sigma ^{2}_{x} = \sigma ^{2}_{y}=\sigma ^{2}_{z}=1$.
When $k \neq k'$ the operators commute and cancel.
Thus
\[
\sum_{f} \left[B_{fi}(F_{-}) - B_{fi}(F_{+}) \right] = (N_{i}-Z_{i}),       
\]
and
\[
\sum_{f} \left[ B_{fi}(GT_{-}) - B_{fi}(GT_{+}) \right] = 3(N_{i}-Z_{i}).       
\]

The sum-rule for the Fermi matrix elements applies even
when isospin is not conserved.
% --- end paragraph admon ---



% !split
\subsection*{$\beta$-decay}

% --- begin paragraph admon ---
\paragraph{}
For $N > Z$ we usually have
$T_{i}=T_{zi}$ which means that $B(F_{+})=0$.

For $N=Z (T_{zi}=0)$ and $T_{i}=0$ we get 
$B(F_{+})=B(F_{-})=0$, and for $T_{i}=1$ we
have $B(F_{+}) = B(F_{-}) = 2$. Fermi transitions which would be zero
if isospin is conserved are called isospin-forbidden Fermi transitions.

When $N > Z$ there are some situations where one has $B(GT_{+})=0$,
and then we obtain $B(GT_{-}) = 3(N_{i}-Z_{i})$. In particular
for the $\beta_{-}$ decay of the neutron we have $B(F_{-})=1$
and $B(GT_{-})=3$.
% --- end paragraph admon ---

























% ------------------- end of main content ---------------


\printindex

\end{document}

