\documentclass[letterpaper,12pt]{article}
\usepackage[latin1]{inputenc}
\usepackage{amsmath,array,theorem,amssymb,mathrsfs}
\usepackage{graphicx} 
\usepackage{fancyhdr}
\usepackage{fancybox}

\newtheorem{convention}{Convention}
\newtheorem{propriete}{Property}
\theorembodyfont{\sf}
\theoremheaderfont{\scshape}

\newcommand{\gras}[1]{\boldsymbol{#1}}

\setlength{\parskip}{0.3cm}
\setlength{\parindent}{0.0cm}
\setlength{\fboxrule}{0.03cm}
\setlength{\fboxsep}{0.25cm}
\addtolength{\voffset}{-2.5cm} 
\addtolength{\textwidth}{2.5cm} 
\addtolength{\textheight}{4.5cm} 
\addtolength{\oddsidemargin}{-1.5cm}

\setcounter{MaxMatrixCols}{25}
\setcounter{topnumber}{5}
\setcounter{bottomnumber}{5}
\setcounter{totalnumber}{15}
\setcounter{tocdepth}{6}
\setcounter{secnumdepth}{6}

\title{Working in the Spherical Harmonic Oscillator Basis}
\author{Nicolas Schunck}

\begin{document}
\maketitle

%%%%%%%%%%%%%%%%%%%%%%%%%%%%%%%%%%%%%%%%%%%%%%%%%%%%%%%%%%%%%%%%%%%%%%%%%%%%%%%%
%%%%%%%%%%%%%%%%%%%%%%%%%%%%%%%%%%%%%%%%%%%%%%%%%%%%%%%%%%%%%%%%%%%%%%%%%%%%%%%%
%%%%%%%%%%%%%%%%%%%%%%%%%%%%%%%%%%%%%%%%%%%%%%%%%%%%%%%%%%%%%%%%%%%%%%%%%%%%%%%%
%%%%%%%%%%%%%%%%%%%%%%%%%%%%%%%%%%%%%%%%%%%%%%%%%%%%%%%%%%%%%%%%%%%%%%%%%%%%%%%%

The purpose of these notes is to help you computing the matrix elements of the 
Minnesota potential in the harmonic oscillator basis. 

%%%%%%%%%%%%%%%%%%%%%%%%%%%%%%%%%%%%%%%%%%%%%%%%%%%%%%%%%%%%%%%%%%%%%%%%%%%%%%%%
%%%%%%%%%%%%%%%%%%%%%%%%%%%%%%%%%%%%%%%%%%%%%%%%%%%%%%%%%%%%%%%%%%%%%%%%%%%%%%%%
%%%%%%%%%%%%%%%%%%%%%%%%%%%%%%%%%%%%%%%%%%%%%%%%%%%%%%%%%%%%%%%%%%%%%%%%%%%%%%%%
%%%%%%%%%%%%%%%%%%%%%%%%%%%%%%%%%%%%%%%%%%%%%%%%%%%%%%%%%%%%%%%%%%%%%%%%%%%%%%%%

\section{The Spherical Harmonic Oscillator Basis}

In this section, we look at the eigenstates of the spherical quantum harmonic 
oscillator 
\begin{equation}
\hat{H}_{0} = \frac{\gras{p}^{2}}{2m} + \frac{1}{2}m\omega\gras{r}^{2}
\end{equation}
in the special case of spherical symmetry.

%%%%%%%%%%%%%%%%%%%%%%%%%%%%%%%%%%%%%%%%%%%%%%%%%%%%%%%%%%%%%%%%%%%%%%%%%%%%%%%%
%%%%%%%%%%%%%%%%%%%%%%%%%%%%%%%%%%%%%%%%%%%%%%%%%%%%%%%%%%%%%%%%%%%%%%%%%%%%%%%%

\subsection{Eigenstates of the Harmonic Oscillator}

{\bf General Form - } The solutions to the Schr\"{o}dinger equation for an 
arbitrary central potential in spherical symmetry are entirely characterized by 
the quantum numbers $n$, $\ell$, $j$ and $m$; the wave-functions factorize 
according to
\begin{equation}
\boxed{
\psi_{n\ell jm}(r,\theta, \varphi) 
= 
R_{n\ell}(r)\mathfrak{Y}_{\ell jm}(\theta, \varphi)
      }
\label{A-wave-wf}
\end{equation}
where $R_{n\ell}(r)$ is the radial wave-function and $\mathfrak{Y}_{\ell
jm}(\theta, \varphi)$ are the solid harmonics. The solid harmonics $\ell, j,m$
correspond to the tensor product of the spherical harmonics $Y_{\ell
m_{\ell}}(\theta, \varphi)$ with the spin functions $\chi_{s m_{s}}$,
\begin{equation}
\mathfrak{Y}_{\ell jm}(\theta, \varphi) 
= 
\left[Y_{\ell m_{\ell}} (\theta, \varphi) \otimes \chi_{s m_{s}} \right]_{jm}.
\end{equation}
More explicitely, this can be re-written
\begin{equation}
\mathfrak{Y}_{\ell jm}(\theta, \varphi) 
= 
\sum_{m_{s}=\pm 1/2} 
C_{\ell m_{\ell},s m_{s}}^{jm} Y_{\ell m_{\ell}}(\theta, \varphi),
\chi_{s m_{s}}
\end{equation}
where the symbols $C_{\ell m_{\ell},s m_{s}}^{jm}$ are the Clebsch-Gordan
coefficients.

{\bf Radial Function for the Harmonic Oscillator - } In the case where the 
potential is the harmonic oscillator, the radial wave function $R_{n\ell}(r)$ 
becomes
\begin{equation}
\boxed{
R_{n\ell}(r) = \frac{A_{n\ell}}{b^{3/2}} \xi^{\ell}e^{-\xi^{2}/2} L_{n}^{\ell+1/2}(\xi^{2})
      }
\label{A-wave-radial}
\end{equation}
where $\xi = r/b$ is a dimensionless variable and $b = \sqrt{\hbar/(m\omega)}$ 
is the oscillator length (in fm). The quantities $L_{n}^{\ell+1/2}$ are the generalized 
Laguerre polynomials. In Eq.~(\ref{A-wave-radial}), $A_{n\ell}$ is a 
normalization constant. To determine it, we use the orthonormality of the 
wave functions $\psi_{n\ell j m}$ and find
\begin{equation}
\boxed{
A_{n\ell} = \sqrt{\frac{2^{n+\ell+2}n!}{\pi^{1/2}(2n + 2\ell + 1)!!} }
      }
\end{equation}

%%%%%%%%%%%%%%%%%%%%%%%%%%%%%%%%%%%%%%%%%%%%%%%%%%%%%%%%%%%%%%%%%%%%%%%%%%%%%%%
%%%%%%%%%%%%%%%%%%%%%%%%%%%%%%%%%%%%%%%%%%%%%%%%%%%%%%%%%%%%%%%%%%%%%%%%%%%%%%%

\subsection{Generalized Laguerre Polynomials}

{\bf Recurrence Relation} - The generalized Laguerre polynomials verify the
following recurrence relations (Abramowitz, 22.7.29, 22.7.30)
\begin{eqnarray}
\displaystyle L_{n}^{(\alpha +1)}(x) & = &
\frac{1}{x}\left[ (x-n)L_{n}^{(\alpha)}(x) + (\alpha+n)L_{n-1}^{(\alpha)}(x) 
           \right]
\label{A-wave-inter1}\medskip\\
\displaystyle L_{n}^{(\alpha-1)}(x) & = & L_{n}^{(\alpha)}(x) 
            - L_{n-1}^{(\alpha)}(x),
\label{A-wave-inter2}
\end{eqnarray} 
where $n$ is an integer, $n\in\mathbb{N}$, and $\alpha$ is a real number. In the
following, we will only need $\alpha$ half-integer. The two relations
(\ref{A-wave-inter1})-(\ref{A-wave-inter2}) are equivalent to
\begin{equation}
L_{n}^{(\alpha +1)}(x) = \frac{1}{x}\left[ (x+\alpha)L_{n}^{(\alpha)}(x) -
(\alpha+n)L_{n}^{(\alpha -1)}(x) \right].
\end{equation}
The first two polynoms are obtained from
\begin{eqnarray}
\displaystyle L_{n}^{(-1/2)}(x) & = &
\frac{(-1)^{n}}{n!2^{n}}H_{2n}(\sqrt{x})\medskip\\
\displaystyle L_{n}^{(+1/2)}(x) & = & 
              \frac{(-1)^{n}}{n!2^{n+1}}H_{2n+1}(\sqrt{x})
\end{eqnarray} 
where $H_{n}(x)$ is the Hermite polynomials of order $n$.

{\bf Orthonormality} - The generalized Laguerre polynomials verify the 
following orthonormality condition
\begin{equation}
\boxed{
\int_{0}^{+\infty}e^{-u}u^{\alpha} L_{n}^{(\alpha)}(u)L_{n'}^{(\alpha)}(u)du 
=
\delta_{nn'}\frac{\Gamma(n+\alpha+1)}{n!}},
\label{A-wave-ortho-lag}
\end{equation}
for $\alpha> -1$ and $n\in\mathbb{N}$. The Gamma function is, for any integer 
$k$ (Abramowitz, 6.1.12),
\begin{equation}
\Gamma\left(k + \frac{1}{2}\right) 
= 
\frac{1\times 3\times\dots\times (2k-1)}{2^{k}}\Gamma\left(\frac{1}{2}\right)
\end{equation}
which can be recast into
\begin{equation}
\Gamma\left(k + \frac{1}{2}\right) = \frac{(2k)!}{2^{2k}p!}
\Gamma\left(\frac{1}{2}\right)
\end{equation}
with $\Gamma(1/2) = \sqrt{\pi}$.

%%%%%%%%%%%%%%%%%%%%%%%%%%%%%%%%%%%%%%%%%%%%%%%%%%%%%%%%%%%%%%%%%%%%%%%%%%%%%%%%
%%%%%%%%%%%%%%%%%%%%%%%%%%%%%%%%%%%%%%%%%%%%%%%%%%%%%%%%%%%%%%%%%%%%%%%%%%%%%%%%
%%%%%%%%%%%%%%%%%%%%%%%%%%%%%%%%%%%%%%%%%%%%%%%%%%%%%%%%%%%%%%%%%%%%%%%%%%%%%%%%
%%%%%%%%%%%%%%%%%%%%%%%%%%%%%%%%%%%%%%%%%%%%%%%%%%%%%%%%%%%%%%%%%%%%%%%%%%%%%%%%

\section{Matrix Elements of the Hamiltonian}

We now move to the problem of computing the matrix elements of the Minnesota 
Hamiltonian in the HO basis. Recall that the Hamiltonian reads
\begin{equation}
\hat{H} 
= 
\sum_{ab} t_{ab} c_{a}^{\dagger}c_{b} 
+ 
\frac{1}{2}\sum_{abcd}\bar{v}_{abcd} c_{a}^{\dagger}c_{b}^{\dagger}c_{d}c_{c},
\end{equation}
with the antisymmetrized matrix elements defined by
\begin{equation}
\bar{v}_{abcd} 
= 
\int d^{3}\gras{r}_{1}\int d^{3}\gras{r}_{2}\;
\phi_{a}^{*}(x_{1})\phi_{b}^{*}(x_{2})
\hat{V}\left( 1 - \hat{P}_{\sigma}\hat{P}_{r} \right)
\phi_{c}^{*}(x_{1})\phi_{d}(x_{2})
\end{equation}
with $a$ a generic notation for $a \equiv (n_{a},\ell_{a}, j_{a}, m_{a})$.

%%%%%%%%%%%%%%%%%%%%%%%%%%%%%%%%%%%%%%%%%%%%%%%%%%%%%%%%%%%%%%%%%%%%%%%%%%%%%%%%
%%%%%%%%%%%%%%%%%%%%%%%%%%%%%%%%%%%%%%%%%%%%%%%%%%%%%%%%%%%%%%%%%%%%%%%%%%%%%%%%

\subsection{Matrix of the Kinetic Energy Operator}

We give below, without demonstration, the matrix elements of the kinetic 
energy operator, i.e., the elements $t_{ac}$. By virtue of the spherical 
symmetry, we have
\begin{equation}
t_{ac} 
= \langle a | \hat{T} | c \rangle 
= \langle n_{a} \ell_{a} j_{a} m_{a}| \hat{T} | n_{c} \ell_{c} j_{c} m_{c} \rangle
= \delta_{\ell_{a}\ell_{c}}\delta_{j_{a}j_{c}}\delta_{m_{a}m_{c}}
\langle n_{a} \ell_{a} j_{a} m_{a}| \hat{T} | n_{c} \ell_{a} j_{a} m_{a} \rangle
\end{equation}
In practice, straightforward but somewhat lengthy calculations (involving 
various tricks from angular momentum algebra) give
\begin{equation*}
\boxed{
\begin{array}{ll}
\langle n_{a} \ell_{a} j_{a} m_{a}| \hat{T} | n_{c} \ell_{a} j_{a} m_{a} \rangle
= \displaystyle
\frac{1}{2}\hbar\omega \left( N + \frac{3}{2} \right) 
& \text{for}\ n_{a} = n_{c} 
\medskip\\
\langle n_{a} \ell_{a} j_{a} m_{a} | \hat{T} | n_{c} \ell_{a} j_{a} m_{a} \rangle
= \displaystyle
\frac{1}{2}\hbar\omega
\sqrt{
n_{c}(n_{c} + \ell_{a} + 1/2)
     }
& \text{for}\ n_{a} = n_{c} - 1 
\medskip\\
\langle n_{a} \ell_{a} j_{a} m_{a}  | \hat{T} | n_{c} \ell_{a} j_{a} m_{a} \rangle
=\displaystyle
\frac{1}{2}\hbar\omega
\sqrt{
n_{a}(n_{a} + \ell_{a} + 1/2)
     }
& \text{for}\ n_{a} = n_{c} + 1
\end{array}
}
\end{equation*}
In this expression, $N = 2n + \ell$ is the main oscillator number.

%%%%%%%%%%%%%%%%%%%%%%%%%%%%%%%%%%%%%%%%%%%%%%%%%%%%%%%%%%%%%%%%%%%%%%%%%%%%%%%%
%%%%%%%%%%%%%%%%%%%%%%%%%%%%%%%%%%%%%%%%%%%%%%%%%%%%%%%%%%%%%%%%%%%%%%%%%%%%%%%%

\subsection{Gauss-Laguerre Quadratures}

{\bf Presentation - } Gauss quadratures are general mathematical methods used to compute integrals of a function. They are based on the properties of orthogonal polynomials and come in several variants. The Gauss-Laguerre quadrature formula reads
\begin{equation}
\int_{0}^{+\infty} x^{\alpha}e^{-x}f(x)dx = \sum_{n=1}^{N_{q}} w_{n}f(x_n) + R_{N_{q}},
\end{equation}
where the weights $w_{n}$ are given by
\begin{equation}
w_{n} = \frac{\Gamma(n+\alpha+1)x_{n}}{n!(n+1)^2\left[L_{n+1}^{\alpha}(x_{n})\right]^{2}},
\end{equation}
the nodes $x_n$ are the zeros of the generalized Laguerre polynomials, and $R_{N_{q}}$ is a remainder. The integer $N_{q}$ is the order of the quadrature.

\fbox{
\parbox{\textwidth}{
{\em The essential property of all types of Gauss quadrature is that the quadrature formula is {\em exact} if $f(x)$ is a polynomial of order $p \leq 2N_{q}-1$, that is:
\begin{equation*}
\int_{0}^{+\infty} x^{\alpha}e^{-x}f(x)dx = \sum_{n=1}^{N_{q}} w_{n}f(x_n).
\end{equation*}
}
}
}

{\bf Example - } To illustrate how useful quadrature formula can be in practice, consider the calculation of the radial integral giving the matrix element of some operator $\hat{O}(r)$ in spherical symmetry. For the sake of simplicity, let us assume that $\hat{O}(r)$ does not contain differential operators for the time being. We have to compute something like
\begin{equation}
\langle n_{a} | \hat{O}(r) | n_{c} \rangle 
\propto
\int_{0}^{+\infty} r^{2}dr  \times
e^{-\xi^{2}/2} \xi^{\ell_{a}} L_{n_{a}}^{\ell_{a}+1/2}(\xi^{2}) \times
\hat{O}(r) \times
e^{-\xi^{2}/2} \xi^{\ell_{a}} L_{n_{c}}^{\ell_{a}+1/2}(\xi^{2}),
\end{equation}
which can be simplified into something like
\begin{equation}
\langle n_{a} | \hat{O}(r) | n_{c} \rangle 
\propto
\int_{0}^{+\infty}   
u^{\alpha}e^{-u}
\hat{O}(u) L_{n_{a}}^{\alpha}(u) L_{n_{c}}^{\alpha}(u) du, \ \ \ \alpha=\ell_{a}+1/2
\end{equation}
Depending on the properties of the operator $\hat{O}(r)$, we can try to choose the order of the quadrature $N_{q}$ such that these integrations are exact.



\end{document}
