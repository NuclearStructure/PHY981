{\bf RPA equations - } Recall that the RPA equations for channel $\nu$ read
\begin{equation}
\left(
\begin{array}{cc}
A & B \\
-B & -A
\end{array}
\right)
\left(
\begin{array}{c}
X^{(\nu)} \\
Y^{(\nu)} 
\end{array}
\right)
=
E_{\nu}
\left(
\begin{array}{c}
X^{(\nu)} \\
Y^{(\nu)}
\end{array}
\right)
\end{equation}
with
\begin{eqnarray}
A_{mi,nj} & = & (\epsilon_{m} - \epsilon_{i})\delta_{mn}\delta_{ij} + 
\underset{J}{
\overset{J}{
\acontraction{\langle }{m }{j | \hat{V}^{\text{res}} | }{i}
\bcontraction{\langle m }{j}{| \hat{V}^{\text{res}} i | }{n}
\langle m j | \hat{V}^{\text{res}} | i n \rangle
}
}
\label{eq:A}
\\
B_{mi,nj} & = & 
\underset{J}{
\overset{J}{
\acontraction{\langle }{m }{j | \hat{V}^{\text{res}} | }{i}
\bcontraction{\langle m }{j}{| \hat{V}^{\text{res}} i | }{n}
\langle m j | \hat{V}^{\text{res}} | i n \rangle
}
}
\label{eq:B}
\end{eqnarray}
where
\begin{itemize}
\item As before, labels $m, n, \dots$ refer to particle states (above the Fermi 
level) and labels $i,j, \dots$ refer to hole states (below the Fermi level)
\item $\epsilon_{i}$ are the eigenvalues of the HF equations
\item $\hat{V}_{\text{res}}$ is the residual interaction; in your project, you 
will take the same interaction as for the HF equations, i.e., 
$\hat{V}^{\text{res}}$ will be the Minnesota potential
\end{itemize}

{\bf Matrix elements - } The notation for the matrix elements indicates that 
they are computed in the $J$-scheme; since the couping is between {\em particle 
and holes of the HF states}, you do not have access to these matrix elements: 
the ones that you obtain from Morten's code are matrix elements {\em in the HO 
basis}, and the coupling is between s.p. states of the bra and kets. 

The simplest way to proceed is the brute force method (as usual). The coupled 
matrix elements for the RPA are defined from the uncoupled ones as
\begin{multline}
\underset{J}{
\overset{J}{
\acontraction{\langle }{m }{j | \hat{V}^{\text{res}} | }{i}
\bcontraction{\langle m }{j}{| \hat{V}^{\text{res}} i | }{n}
\langle m j | \hat{V}^{\text{res}} | i n \rangle
}
}
=
\sum_{\text{all\ m}} (-1)^{j_{b}-m_{b}+j_{c}-m_{c}}
(j_{a}j_{c}m_{a}-m_{c}|JM)(j_{b}j_{d}m_{b}-m_{d}|JM) \\
\times\langle j_{a}m_{a},j_{b}m_{b} | \hat{V}^{\text{res}} | j_{c}m_{c},j_{d}m_{d}\rangle
\label{eq:matV}
\end{multline}
Now, the matrix elements $V^{\text{res}}_{abcd} = \langle j_{a}m_{a},j_{b}m_{b} | \hat{V}_{\text{res}} | j_{c}m_{c},j_{d}m_{d}\rangle$ 
are uncoupled (in terms of angular momentum), but the states are still HF 
states. However, from the diagonalization of the HF Hamiltonian, you get the 
expansion of these states as function of HO states,
\begin{equation}
|\bar{n}ljm\rangle_{HF} = \sum_{n'} D^{lj}_{n'\bar{n}} |n'ljm\rangle_{HO}
\end{equation}
You can, therefore, express all the matrix elements $V^{\text{res}}_{abcd}$ as 
a function of the original matrix elements of the potential in the HO basis. 

{\bf Practical implementation}
\begin{enumerate}
\item For a given $J$ and parity $\pi$, define the basis of coupled 
particle-hole excitations, i.e., the states $|mi\rangle$ by taking all hole 
states below the Fermi level, and all the particle states up to a cut-off 
$E_{\text{cut}}$ that should be an input of your code. Note that in the various 
expressions given above, I dropped the indices related to the $n$ and $l$ 
quantum numbers. In practice, the HF states coming out as eigenstates of the 
HF matrix are
\begin{eqnarray*}
k=0: & |0,0,1/2\rangle \equiv 0s1/2 \\
k=1: & |0,1,3/2\rangle \equiv 0p3/2 \\
k=2: & |0,1,1/2\rangle \equiv 0p1/2 \\
k=3: & |0,2,3/2\rangle \equiv 0d3/2 \\
k=4: & |0,2,5/2\rangle \equiv 0d5/2 \\
k=5: & |2,0,1/2\rangle \equiv 1s1/2 \\
\vdots & \vdots
\end{eqnarray*}
The first step is, therefore, to set up some bookkeeping mechanism to keep 
track of your s.p. states and compute the p.h. two-body states.
\item Compute the $JJ$ matrix elements according to Eq.(\ref{eq:matV})
\item Compute the matrix elements of your RPA matrix according to 
Eqs.(\ref{eq:A})-(\ref{eq:B})
\item Diagonalize the RPA matrix
\end{enumerate}
