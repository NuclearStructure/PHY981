\documentclass[letterpaper,12pt]{article}

\usepackage{amsmath}           % Include AMSTeX style
\usepackage{amsfonts}
\usepackage{graphicx}          % Include figure files
\usepackage{dcolumn}           % Align table columns on decimal point
\usepackage{bm}
\usepackage{array}
\usepackage{lscape}
\usepackage{hyperref}
\usepackage{dcolumn}% Align table columns on decimal point
\usepackage{bm}% bold math
\usepackage{pst-plot}
\usepackage{colortbl}
\usepackage{listings}
\usepackage{simplewick}

\newcommand{\gras}[1]{\boldsymbol{#1}}

% Page size customizations

\setlength{\parskip}{0.3cm}
\setlength{\parindent}{0.0cm}
\setlength{\fboxrule}{0.025cm}
\setlength{\fboxsep}{0.25cm}

\addtolength{\headsep}{1.0cm}
\addtolength{\voffset}{-1.0cm}
\addtolength{\textheight}{2.0cm}
\addtolength{\textwidth}{2.5cm}
\addtolength{\oddsidemargin}{-1.25cm}
\addtolength{\evensidemargin}{-1.25cm}

\setcounter{MaxMatrixCols}{25}
\setcounter{topnumber}{5}
\setcounter{bottomnumber}{5}
\setcounter{totalnumber}{15}
\setcounter{tocdepth}{5}
\setcounter{secnumdepth}{5}

%
% BEGINNING
%


\lstset{language=c++}
\lstset{basicstyle=\small}
%\lstset{backgroundcolor=\color{white}}
%\lstset{frame=single}
\lstset{stringstyle=\ttfamily}
%\lstset{keywordstyle=\color{red}\bfseries}
%\lstset{commentstyle=\itshape\color{blue}}
\lstset{showspaces=false}
\lstset{showstringspaces=false}
\lstset{showtabs=false}
\lstset{breaklines}

\newcommand{\One}{\hat{\mathbf{1}}}
\newcommand{\eff}{\text{eff}}
\newcommand{\Heff}{\hat{H}_\text{eff}}
\newcommand{\Veff}{\hat{V}_\text{eff}}
\newcommand{\braket}[1]{\langle#1\rangle}
\newcommand{\Span}{\operatorname{sp}}
\newcommand{\tr}{\operatorname{trace}}
\newcommand{\diag}{\operatorname{diag}}
\newcommand{\bra}[1]{\left\langle #1 \right|}
\newcommand{\ket}[1]{\left| #1 \right\rangle}
\newcommand{\element}[3]
    {\bra{#1}#2\ket{#3}}

\newcommand{\normord}[1]{
    \left\{#1\right\}
}


\begin{document}


\title{Hartree-Fock Calculations of Neutron Drops\\
\Large{Phase 3: Development of Extensions Beyond Hartree-Fock}}
\maketitle


%%%%%%%%%%%%%%%%%%%%%%%%%%%%%%%%%%%%%%%%%%%%%%%%%%%%%%%%%%%%%%%%%%%%%%%%%%%%%%%%
%%%%%%%%%%%%%%%%%%%%%%%%%%%%%%%%%%%%%%%%%%%%%%%%%%%%%%%%%%%%%%%%%%%%%%%%%%%%%%%%
%%%%%%%%%%%%%%%%%%%%%%%%%%%%%%%%%%%%%%%%%%%%%%%%%%%%%%%%%%%%%%%%%%%%%%%%%%%%%%%%
%%%%%%%%%%%%%%%%%%%%%%%%%%%%%%%%%%%%%%%%%%%%%%%%%%%%%%%%%%%%%%%%%%%%%%%%%%%%%%%%

\section{Introduction}

The HF solver that you have developed is the starting point to study a number 
of different approximations to the many-body problem. In phase III of the 
computational projects, we propose that you try and implement one of the 
following extensions beyond HF.
\begin{itemize}
\item Implement the density matrix expansion of the Minnesota potential. The 
goal here is to benchmark the DME -- which produces a functional of the local 
density -- against the exact result.
\item Solve the HFB equations in spherical symmetry. The level of difficulty 
should not be too high if you understand your HF solver well. In addition to 
doubling the size of the matrix to diagonalize and dealing with the pairing 
tensor, one technical difficulty is to determine the Fermi energy by solving a 
Newton-like method.
\item Solve the RPA equations in spherical symmetry. It is also relatively 
simple, but you will need to pay attention to your implementation, since naive 
brute force could make your code very slow.
\item Solve the deformed HF equations. This is probably the most ambitious
project: you need to set up new quadrature rules and basis functions, and put 
up some significant work into designing your implementation, else your code 
will run for ever...
\end{itemize}

%%%%%%%%%%%%%%%%%%%%%%%%%%%%%%%%%%%%%%%%%%%%%%%%%%%%%%%%%%%%%%%%%%%%%%%%%%%%%%%%
%%%%%%%%%%%%%%%%%%%%%%%%%%%%%%%%%%%%%%%%%%%%%%%%%%%%%%%%%%%%%%%%%%%%%%%%%%%%%%%%
%%%%%%%%%%%%%%%%%%%%%%%%%%%%%%%%%%%%%%%%%%%%%%%%%%%%%%%%%%%%%%%%%%%%%%%%%%%%%%%%
%%%%%%%%%%%%%%%%%%%%%%%%%%%%%%%%%%%%%%%%%%%%%%%%%%%%%%%%%%%%%%%%%%%%%%%%%%%%%%%%

%\section{Density matrix expansion of the Minnesota potential}

%As discussed in the lectures, the density matrix expansion (DME) is a promising technique to build a quasi-local EDF starting from the underlying NN and NNN interactions, working either at the Hartree-Fock (HF) or Brueckner-Hartree-Fock (BHF) level.  In the following, we outline the basic steps to derive and implement quasi-local EDF approximations to the fully non-local HF calculations you implemented in the first two phases of the project. 

{\bf Local Density Approximation to Hartree-Fock } The LDA gives the simplest path for deriving a local EDF starting from a microscopic hamiltonian. In the most sophisticated implementation, one only applies the LDA to the non-local exchange (Fock) energy, treating the finite-range Hartree contribution exactly since it only probes the diagonal densities.  With your HF code, it is no problem in principle to treat the Hartree term exactly. However, this requires that one treats the direct and exchange matrix elements of the anti-symmetrized NN potential separately.  Unfortunately, the ``black box'' code used to generate the m-scheme and J-scheme matrix elements has the antisymmetry built in, making it hard to separate the direct and exchange contributions without digging deep into the workings of the code. Therefore, for the present problem you will apply the LDA to both the Hartree and Fock energy contributions to avoid this technicality.

At the heart of the LDA is a calculation of the energy/particle of the infinite homogenous system-- pure neutron matter in the present problem.  Here is an outline of the steps you need to do this. 
\begin{enumerate}
\item Starting from the general expression for the HF interaction energy, show that one gets the following expression for the Minnesota potential for spin-saturated systems (i.e., systems with vanishing spin-vector density matrices)
\begin{eqnarray}
E^{HF}_{int} &=& E_H + E_F\\
\label{eq:HF}
E_H &=& \frac{1}{2}\int d{\bf R}\!\int  d{\bf r}\, V_C(r)\rho({\bf R}+{\bf r}/2)\rho({\bf R}-{\bf r}/2) \nonumber\\
E_F &=&  \frac{1}{2}\int d{\bf R}\!\int  d{\bf r}\, V_C(r)\rho({\bf R}+{\bf r}/2,{\bf R}-{\bf r}/2)\rho({\bf R}-{\bf r}/2,{\bf R}+{\bf r}/2)\,, \nonumber
\end{eqnarray}
where $V_C = \frac{1}{4}(V_R + V_S)$.  

\item Now apply this to the system of infinite homogenous neutron matter ($A\rightarrow\infty$, $V\rightarrow \infty$, $\rho\rightarrow const$), remembering that $\rho({\bf r}_1,{\bf r}_2) = \rho\, \rho_{SL}(k_Fr)$, where $\rho_{SL}(x) = 3j_1(x)/x$. This gives the following expression for the HF interaction energy per particle 

\begin{equation}
\frac{E^{HF}_{int}}{A} \equiv e_{HF}(\rho) = \frac{1}{2}\rho \int d{\bf r} V_C(r)  + \frac{1}{2}\rho \int d{\bf r}\rho_{SL}^2(k_Fr)V_C(r)\,.
\end{equation}

\item Now that you have the HF energy for the infinite system, the LDA amounts to defining the interaction energy piece of the EDF as 
\begin{equation}
E_{int}[\rho] \equiv \int d{\bf r} \rho({\bf r}) e_{HF}(\rho({\bf r}))\,. 
\end{equation}
The LDA approximation to the HF s.p. hamiltonian is then purely local, and is given by
\begin{equation}
h({\bf r}) = h_0({\bf r}) + \frac{\delta}{\delta\rho({\bf r})} E_{int}[\rho]\equiv h_0({\bf r}) + \Gamma_{LDA}({\bf r})\,,
\end{equation}
where $h_0$ is the HO hamiltonian and the s.p. field $\Gamma_{LDA}$ is given by
\begin{equation}
\Gamma_{LDA}({\bf r}) = \frac{\delta}{\delta\rho({\bf r})} E_{int}[\rho] = e_{HF}(\rho({\bf r})) + \rho({\bf r})\frac{\partial}{\partial \rho}e_{HF}(\rho)|_{\rho({\bf r})}
\end{equation}
  Note that $\Gamma_{LDA}({\bf r})$ depends on the density, so the resulting s.p. equations need to be solved self-consistently as in the original HF calculations.  You can use your existing HF code to do this, once you have implemented functions to calculate $e_{HF}(\rho)$ and $\Gamma_{LDA}({\bf r})$, and to take matrix elements of  $\Gamma_{LDA}({\bf r})$ on the HO basis
\begin{equation}
\langle nljm|\Gamma_{LDA}|n'ljm\rangle = \int r^2dr R_{nl}(r) \Gamma_{LDA}(r)R_{n'l}(r)
\end{equation}
\item Once you have reached self-consistency, the LDA approximation to the HF energy is evaluated as 
\begin{equation}
E^{HF}_{LDA} = \sum_{i=1}^{A} \langle \phi_i|h_0|\phi_i\rangle + E_{int}[\rho]\,, 
\end{equation}
where $\phi_i$ are the self-consistent HF-LDA orbitals and $\rho$ is the self-consistent density
\begin{equation}
\rho({\bf r}) = \sum_{i=1}^{A} \phi^*_i({\bf r})\phi_i({\bf r})
\end{equation}
\end{enumerate}
{\bf Density Matrix Expansion -  }  The DME approximation to HF looks very similar to the simple LDA outlined above, but now with explicit gradient corrections and dependence on the local kinetic energy density, $\tau({\bf R}) = \nabla_1\cdot\nabla_2\rho({\bf r}_1,{\bf r}_2)\big|_{{\bf r}_1={\bf r}_2={\bf R}}$. As with the LDA calculation, we will not treat the full finite-range Hartree energy exactly. Rather, we will apply a naive Taylor expansion to map the Hartree energy into a Skyrme-like form. Here are the steps to implement the DME approximation to HF:

\begin{enumerate}
\item In the non-local Fock energy expression, plug in the DME expression for the density matrix
\begin{eqnarray}
\rho({\bf R}+{\bf r}/2,{\bf R}-{\bf r}/2) \approx \pi_0(k_Fr)\rho({\bf R}) + \frac{r^2}{6}\pi_2(k_Fr)\bigl[\frac{1}{4}\nabla^2\rho - \tau + \frac{3}{5}k_F^2\rho\bigr]\,,
\end{eqnarray}
keeping only terms to 2nd-order in small quantities (i.e., treat the terms involving $\pi_2$ as 2nd-order), so that
\begin{equation}
\rho^2({\bf R}+{\bf r}/2,{\bf R}-{\bf r}/2) \approx \pi_0^2\rho^2 + \frac{r^2\pi_0\pi_2}{3}\biggl[\frac{\rho\nabla^2\rho}{4} - \rho\tau + \frac{3}{5}k_F^2\rho^2\biggr]
\end{equation}
 \item Simplify your expression for $E_{F}[\rho,\tau,\nabla^2\rho]$ to get it into the form
 \begin{equation}
 E_{F} \approx \int d{\bf R} \biggl\{ C^{\rho\rho}\rho^2 + C^{\rho\tau}\rho\tau + C^{\rho\nabla^2\rho}\rho\nabla^2\rho\biggr\}\,.
 \end{equation}
 Your expressions for the density-dependent couplings should take the form as integrals of the $\pi$-functions over the finite range NN potential.  For example, you should find
 \begin{equation}
 C^{\rho\rho} = \int d{\bf r} V_C(r)\bigl[\pi_0^2(k_Fr) + \frac{1}{5}(k_Fr)^2\pi_0(k_Fr)\pi_2(k_Fr)\bigr]
 \end{equation}
 In both Negele-Vautherin and PSA flavors of the DME, you should be able to get analytical expressions  for all couplings. As a reminder, the PSA $\pi$-functions are all equal to $\rho_{SL}$, while the NV ones are given by $\pi_0(x) = \rho_{SL}(x)$, $\pi_2(x) = 105j_3(x)/x^3$.

\end{enumerate}






%%%%%%%%%%%%%%%%%%%%%%%%%%%%%%%%%%%%%%%%%%%%%%%%%%%%%%%%%%%%%%%%%%%%%%%%%%%%%%%%
%%%%%%%%%%%%%%%%%%%%%%%%%%%%%%%%%%%%%%%%%%%%%%%%%%%%%%%%%%%%%%%%%%%%%%%%%%%%%%%%
%%%%%%%%%%%%%%%%%%%%%%%%%%%%%%%%%%%%%%%%%%%%%%%%%%%%%%%%%%%%%%%%%%%%%%%%%%%%%%%%
%%%%%%%%%%%%%%%%%%%%%%%%%%%%%%%%%%%%%%%%%%%%%%%%%%%%%%%%%%%%%%%%%%%%%%%%%%%%%%%%
\section{Building EDFs at the LDA and DME level}
As discussed in the lectures, the density matrix expansion (DME) is a promising technique to build a quasi-local EDF starting from the underlying NN and NNN interactions, working either at the Hartree-Fock (HF) or Brueckner-Hartree-Fock (BHF) level.  In the following, we outline the basic steps to derive and implement quasi-local EDF approximations to the fully non-local HF calculations you implemented in the first two phases of the project. 

{\bf Local Density Approximation to Hartree-Fock } The LDA gives the simplest path for deriving a local EDF starting from a microscopic hamiltonian. In the most sophisticated implementation, one only applies the LDA to the non-local exchange (Fock) energy, treating the finite-range Hartree contribution exactly since it only probes the diagonal densities.  With your HF code, it is no problem in principle to treat the Hartree term exactly. However, this requires that one treats the direct and exchange matrix elements of the anti-symmetrized NN potential separately.  Unfortunately, the ``black box'' code used to generate the m-scheme and J-scheme matrix elements has the antisymmetry built in, making it hard to separate the direct and exchange contributions without digging deep into the workings of the code. Therefore, for the present problem you will apply the LDA to both the Hartree and Fock energy contributions to avoid this technicality.

At the heart of the LDA is a calculation of the energy/particle of the infinite homogenous system-- pure neutron matter in the present problem.  Here is an outline of the steps you need to do this. 
\begin{enumerate}
\item Starting from the general expression for the HF interaction energy, show that one gets the following expression for the Minnesota potential for spin-saturated systems (i.e., systems with vanishing spin-vector density matrices)
\begin{eqnarray}
E^{HF}_{int} &=& E_H + E_F\\
\label{eq:HF}
E_H &=& \frac{1}{2}\int d{\bf R}\!\int  d{\bf r}\, V_C(r)\rho({\bf R}+{\bf r}/2)\rho({\bf R}-{\bf r}/2) \nonumber\\
E_F &=&  \frac{1}{2}\int d{\bf R}\!\int  d{\bf r}\, V_C(r)\rho({\bf R}+{\bf r}/2,{\bf R}-{\bf r}/2)\rho({\bf R}-{\bf r}/2,{\bf R}+{\bf r}/2)\,, \nonumber
\end{eqnarray}
where $V_C = \frac{1}{4}(V_R + V_S)$.  

\item Now apply this to the system of infinite homogenous neutron matter ($A\rightarrow\infty$, $V\rightarrow \infty$, $\rho\rightarrow const$), remembering that $\rho({\bf r}_1,{\bf r}_2) = \rho\, \rho_{SL}(k_Fr)$, where $\rho_{SL}(x) = 3j_1(x)/x$. This gives the following expression for the HF interaction energy per particle 

\begin{equation}
\frac{E^{HF}_{int}}{A} \equiv e_{HF}(\rho) = \frac{1}{2}\rho \int d{\bf r} V_C(r)  + \frac{1}{2}\rho \int d{\bf r}\rho_{SL}^2(k_Fr)V_C(r)\,.
\end{equation}

\item Now that you have the HF energy for the infinite system, the LDA amounts to defining the interaction energy piece of the EDF as 
\begin{equation}
E_{int}[\rho] \equiv \int d{\bf r} \rho({\bf r}) e_{HF}(\rho({\bf r}))\,. 
\end{equation}
The LDA approximation to the HF s.p. hamiltonian is then purely local, and is given by
\begin{equation}
h({\bf r}) = h_0({\bf r}) + \frac{\delta}{\delta\rho({\bf r})} E_{int}[\rho]\equiv h_0({\bf r}) + \Gamma_{LDA}({\bf r})\,,
\end{equation}
where $h_0$ is the HO hamiltonian and the s.p. field $\Gamma_{LDA}$ is given by
\begin{equation}
\Gamma_{LDA}({\bf r}) = \frac{\delta}{\delta\rho({\bf r})} E_{int}[\rho] = e_{HF}(\rho({\bf r})) + \rho({\bf r})\frac{\partial}{\partial \rho}e_{HF}(\rho)|_{\rho({\bf r})}
\end{equation}
  Note that $\Gamma_{LDA}({\bf r})$ depends on the density, so the resulting s.p. equations need to be solved self-consistently as in the original HF calculations.  You can use your existing HF code to do this, once you have implemented functions to calculate $e_{HF}(\rho)$ and $\Gamma_{LDA}({\bf r})$, and to take matrix elements of  $\Gamma_{LDA}({\bf r})$ on the HO basis
\begin{equation}
\langle nljm|\Gamma_{LDA}|n'ljm\rangle = \int r^2dr R_{nl}(r) \Gamma_{LDA}(r)R_{n'l}(r)
\end{equation}
\item Once you have reached self-consistency, the LDA approximation to the HF energy is evaluated as 
\begin{equation}
E^{HF}_{LDA} = \sum_{i=1}^{A} \langle \phi_i|h_0|\phi_i\rangle + E_{int}[\rho]\,, 
\end{equation}
where $\phi_i$ are the self-consistent HF-LDA orbitals and $\rho$ is the self-consistent density
\begin{equation}
\rho({\bf r}) = \sum_{i=1}^{A} \phi^*_i({\bf r})\phi_i({\bf r})
\end{equation}
\end{enumerate}
{\bf Density Matrix Expansion -  }  The DME approximation to HF looks very similar to the simple LDA outlined above, but now with explicit gradient corrections and dependence on the local kinetic energy density, $\tau({\bf R}) = \nabla_1\cdot\nabla_2\rho({\bf r}_1,{\bf r}_2)\big|_{{\bf r}_1={\bf r}_2={\bf R}}$. As with the LDA calculation, we will not treat the full finite-range Hartree energy exactly. Rather, we will apply a naive Taylor expansion to map the Hartree energy into a Skyrme-like form. Here are the steps to implement the DME approximation to HF:

\begin{enumerate}
\item In the non-local Fock energy expression, plug in the DME expression for the density matrix
\begin{eqnarray}
\rho({\bf R}+{\bf r}/2,{\bf R}-{\bf r}/2) \approx \pi_0(k_Fr)\rho({\bf R}) + \frac{r^2}{6}\pi_2(k_Fr)\bigl[\frac{1}{4}\nabla^2\rho - \tau + \frac{3}{5}k_F^2\rho\bigr]\,,
\end{eqnarray}
keeping only terms to 2nd-order in small quantities (i.e., treat the terms involving $\pi_2$ as 2nd-order), so that
\begin{equation}
\rho^2({\bf R}+{\bf r}/2,{\bf R}-{\bf r}/2) \approx \pi_0^2\rho^2 + \frac{r^2\pi_0\pi_2}{3}\biggl[\frac{\rho\nabla^2\rho}{4} - \rho\tau + \frac{3}{5}k_F^2\rho^2\biggr]
\end{equation}
 \item Simplify your expression for $E_{F}[\rho,\tau,\nabla^2\rho]$ to get it into the form
 \begin{equation}
 E_{F} \approx \int d{\bf R} \biggl\{ C^{\rho\rho}\rho^2 + C^{\rho\tau}\rho\tau + C^{\rho\nabla^2\rho}\rho\nabla^2\rho\biggr\}\,.
 \end{equation}
 Your expressions for the density-dependent couplings should take the form as integrals of the $\pi$-functions over the finite range NN potential.  For example, you should find
 \begin{equation}
 C^{\rho\rho} = \int d{\bf r} V_C(r)\bigl[\pi_0^2(k_Fr) + \frac{1}{5}(k_Fr)^2\pi_0(k_Fr)\pi_2(k_Fr)\bigr]
 \end{equation}
 In both Negele-Vautherin and PSA flavors of the DME, you should be able to get analytical expressions  for all couplings. As a reminder, the PSA $\pi$-functions are all equal to $\rho_{SL}$, while the NV ones are given by $\pi_0(x) = \rho_{SL}(x)$, $\pi_2(x) = 105j_3(x)/x^3$.

\end{enumerate}





\section{HFB equations in spherical symmetry}

{\bf HFB equations - } Remember that the most general form of the HFB equations 
is
\begin{equation}
\left( \begin{array}{cc}
h - \lambda & \Delta \\
-\Delta^{*} & -h^{*} + \lambda
\end{array} \right)
\left( \begin{array}{cc} 
U & V^{*} \\ 
V & U^{*} 
\end{array} \right)
=
\left( \begin{array}{cc} 
U & V^{*} \\ 
V & U^{*} 
\end{array} \right)
\left( \begin{array}{cc}
E & 0 \\
0 & -E
\end{array} \right)
\end{equation}
In this expression, $h$ and $\Delta$ are matrices of size $n\times n$, with $n$ 
the size of the single-particle basis (i.e. the total number of states in the 
HO basis in our example). $\lambda$ is a shorthand notation for $\lambda I$, 
with $I$ the $n\times n$ identity matrix; $\lambda$ is the Fermi level 
introduced to constrain the average value of the particle number to its actual 
value. For each of the $n$ eigenvectors of energy $E_{\mu}$, there is one with 
eigenvector $-E_{\mu}$. 

{\bf Bogoliubov transformation - } In spherical symmetry, the conjugate 
single-particle states $a$ and $\bar{a}$ are characterized by 
$|a\rangle \equiv |n_{a} l_{a} j_{a} m_{a}\rangle$ and 
$|\bar{a}\rangle \equiv |n_{a} l_{a} j_{a} -m_{a}\rangle$. As Peter discussed 
in this lecture on pairing, $a$ and $\bar{a}$ are related by the time-reversal 
operator. Just as in the HF case, the $U$ and $V$ matrices are block diagonal, 
i.e., they take the generic form
\begin{equation}
U_{ab} \equiv \delta_{l_{a}l_{b}} \delta_{j_{a}j_{b}} U^{(l_{a}j_{a})}, 
\ \ \ 
V_{ab} \equiv \delta_{l_{a}l_{b}} \delta_{j_{a}j_{b}} V^{(l_{a}j_{a})}. 
\end{equation}
The difference with the HF case is that things are not entirely independent of 
the projection $m$. Suppose we reorder the labelling of the $2j_{a}+1$ 
$m$-projections in each $(l_{a},j_{a})$ block according to
\begin{equation}
m = -j, -j+1, \dots, +j \rightarrow m = +j,-j, +j-1, -j+1, \dots, +1/2, -1/2.
\end{equation}
In other words, we form pairs of states $(+m,-m)$; there are $j+1/2$ such 
pairs in each block $j$. Spherical symmetry imposes that the $U$ and $V$ 
matrices are block diagonal in each of these $j+1/2$ blocks. Denoting 
generically $U^{(lj)}_{|m|}$ and $V^{(lj)}_{|m|}$ such blocks, we find
\begin{equation}
U^{(lj)}_{|m|} =
\left(
\begin{array}{cc}
u & 0 \\
0 & u
\end{array}
\right),
\ \ \ \ \ 
V^{(lj)}_{|m|} =
\left(
\begin{array}{cc}
0 & v\\
\bar{v} & 0 \\
\end{array}
\right).
\end{equation}
For each s.p. state $a$, we will note
\begin{equation}
u \equiv u^{(a)}, 
\ \ \ 
v \equiv (-1)^{j_{a}-m_{a}} v^{(a)},
\ \ \ 
\bar{v} \equiv (-1)^{j_{a}+m_{a}} v^{(a)},
\end{equation}
The dimension of the block matrices $u^{(a)}$ and $v^{(a)}$ is 
$(N_{0} - l_{a})/2$, where $N_{0}$ is the 
number of oscillator shells.

{\bf Densities - } From these relations, it is straightforward to compute the 
density matrix $\rho$ and pairing tensor $\kappa$. Remember the general 
definition,
\begin{equation}
\rho_{ab} = (V^{*}V^{T})_{ab}, \ \ \ \ \kappa_{ab} = (V^{*}U^{T})_{ab}
\end{equation}
The density matrix and pairing tensor have a similar block structure as the 
$U$ and $V$ in the (reordered) s.p. basis. We find
\begin{equation}
\rho^{(lj)}_{|m|} = 
\left(
\begin{array}{cc}
\rho & 0 \\
0 & \rho
\end{array}
\right),\ \ \ \ 
\kappa^{(lj)}_{|m|} = 
\left(
\begin{array}{cc}
0 & \kappa \\
\bar{\kappa} & 0
\end{array}
\right),
\end{equation}
with 
\begin{equation}
\rho \equiv v^{(a)}v^{(a)T}, 
\ \ \ 
\kappa \equiv (-1)^{j_{a}-m_{a}} v^{(a)}u^{(a)T},
\ \ \ 
\bar{\kappa} \equiv (-1)^{j_{a}+m_{a}} v^{(a)}u^{(a)T},
\end{equation}

{\bf HFB equations - } Based on the previous remarks, one can show (i) first 
that the HFB equations can also be reduced to a block diagonal form in each of 
the $j+1/2$ blocks characterized by $l$, $j$ and $|m|$, (ii) then that these 
new equations can be further reduced so that they also become block-diagonal in 
the subspace of $(+m,-m)$. The end result is that the HFB equations take the 
following form
\begin{equation}
\left( \begin{array}{cc}
h^{(a)} - \lambda & -\Delta^{(a)} \\
-\Delta^{(a)} & -h^{(a)} + \lambda
\end{array} \right)
\left( \begin{array}{cc} 
u^{(a)} \\ 
v^{(a)} 
\end{array} \right)
= E^{(a)}
\left( \begin{array}{c} 
u^{(a)} \\ 
v^{(a)} 
\end{array} \right)
\end{equation}
where the $u^{(a)}$ and $v^{(a)}$ have been introduced before. Denoting
\begin{equation}
\rho^{(a)} = v^{(a)}v^{(a)T}, \ \ \ \kappa^{(a)} = v^{(a)}u^{(a)T}
\end{equation}
we find the mean-field,
\begin{equation}
h_{n_{a}n_{c}}^{(a)} = t_{n_{a}n_{c}}^{(a)} + \Gamma_{n_{a}n_{c}}^{(a)} 
\end{equation}
the HF potential,
\begin{equation}
\Gamma_{n_{a}n_{c}}^{(a)} =
\sum_{n_{b}n_{d}} 
\langle n_{a}m_{a} n_{b}m_{a} | \bar{v} | n_{c}m_{a} n_{d}m_{a}\rangle
\rho^{(a)}_{n_{d}n_{a}}
\end{equation}
and the pairing field
\begin{equation}
\Delta_{n_{a}n_{c}}^{(a)} =
\sum_{n_{b}n_{d}} 
\langle n_{a}m_{a} n_{b}-m_{a} | \bar{v} | n_{c}m_{a} n_{d}-m_{a}\rangle
\kappa^{(a)}_{n_{d}n_{a}}
\end{equation}


{\bf Practical implementation}
\begin{enumerate}
\item In your HF code, you were already dealing with the blocks of the density 
matrix, which I denoted by $\rho^{(a)}$; you must now introduce another such 
object that will contain the pairing tensor $\kappa^{(a)}$. This object will 
also have to be initialized before to start the HFB iterations [See what 
happens if you initialize it with zeros only].
\item The number of particles is not conserved in HFB. You must, therefore, not 
forget to readjust $\lambda$ at each iteration. This is done by requesting that 
\begin{equation}
\text{Tr} \rho = A
\end{equation}
where $A$ is your input particle number. The simplest way to do this is to use  
a BCS-like expression for quasiparticle occupations. At a given iteration, we 
thus define $\bar{\varepsilon}_{n}$ and $\bar{\Delta}_{n}$ 
according to
\begin{eqnarray}
\displaystyle E_{n} &=& \displaystyle \sqrt{ (\bar{\varepsilon}_{n} - \lambda)^{2} + \bar{\Delta}_{n}^{2} } \medskip \\
\displaystyle N_{n} &=& \displaystyle \frac{1}{2} \left[ 1 - \frac{\bar{\varepsilon}_{n} - \lambda}{\sqrt{ (\bar{\varepsilon}_{n} - \lambda)^{2} + \bar{\Delta}_{n}^{2} }} \right]
\end{eqnarray}
where $E_{n}$ is the energy of quasiparticle number $n$, 
$E_{n} \equiv E^{(l_{a}j_{a})}$, and $N_{n}$ is the norm of the $V$ matrix for 
this q.p.
\begin{equation}
N_{n} \equiv N^{(l_{a}j_{a})}= \sum_{n'} |v^{(a)}_{n'n}|^{2}
\end{equation}
[recall that you should get 
$\sum_{n'} |u^{(a)}_{n'n}|^{2} + |v^{(a)}_{n'n}|^{2} = 1$ from the 
diagonalization of the HFB matrix]. By computing the particle number as
\begin{equation}
A(\lambda) = \sum_{l_{a}j_{a}} (2j_{a}+1) N^{(l_{a}j_{a})}
\end{equation}
we can set up a Newton-like method to obtain $\lambda$.
\item The corollary of the previous steps is that you do not need to impose 
that only the first $n_{occ}$ occupied states are included when defining the 
density matrix.
\end{enumerate}



%%%%%%%%%%%%%%%%%%%%%%%%%%%%%%%%%%%%%%%%%%%%%%%%%%%%%%%%%%%%%%%%%%%%%%%%%%%%%%%%
%%%%%%%%%%%%%%%%%%%%%%%%%%%%%%%%%%%%%%%%%%%%%%%%%%%%%%%%%%%%%%%%%%%%%%%%%%%%%%%%
%%%%%%%%%%%%%%%%%%%%%%%%%%%%%%%%%%%%%%%%%%%%%%%%%%%%%%%%%%%%%%%%%%%%%%%%%%%%%%%%
%%%%%%%%%%%%%%%%%%%%%%%%%%%%%%%%%%%%%%%%%%%%%%%%%%%%%%%%%%%%%%%%%%%%%%%%%%%%%%%%

\section{RPA equations in spherical symmetry}

{\bf RPA equations - } Recall that the RPA equations for channel $\nu$ read
\begin{equation}
\left(
\begin{array}{cc}
A & B \\
-B & -A
\end{array}
\right)
\left(
\begin{array}{c}
X^{(\nu)} \\
Y^{(\nu)} 
\end{array}
\right)
=
E_{\nu}
\left(
\begin{array}{c}
X^{(\nu)} \\
Y^{(\nu)}
\end{array}
\right)
\end{equation}
with
\begin{eqnarray}
A_{mi,nj} & = & (\epsilon_{m} - \epsilon_{i})\delta_{mn}\delta_{ij} + 
\underset{J}{
\overset{J}{
\acontraction{\langle }{m }{j | \hat{V}^{\text{res}} | }{i}
\bcontraction{\langle m }{j}{| \hat{V}^{\text{res}} i | }{n}
\langle m j | \hat{V}^{\text{res}} | i n \rangle
}
}
\label{eq:A}
\\
B_{mi,nj} & = & 
\underset{J}{
\overset{J}{
\acontraction{\langle }{m }{j | \hat{V}^{\text{res}} | }{i}
\bcontraction{\langle m }{j}{| \hat{V}^{\text{res}} i | }{n}
\langle m j | \hat{V}^{\text{res}} | i n \rangle
}
}
\label{eq:B}
\end{eqnarray}
where
\begin{itemize}
\item As before, labels $m, n, \dots$ refer to particle states (above the Fermi 
level) and labels $i,j, \dots$ refer to hole states (below the Fermi level)
\item $\epsilon_{i}$ are the eigenvalues of the HF equations
\item $\hat{V}_{\text{res}}$ is the residual interaction; in your project, you 
will take the same interaction as for the HF equations, i.e., 
$\hat{V}^{\text{res}}$ will be the Minnesota potential
\end{itemize}

{\bf Matrix elements - } The notation for the matrix elements indicates that 
they are computed in the $J$-scheme; since the couping is between {\em particle 
and holes of the HF states}, you do not have access to these matrix elements: 
the ones that you obtain from Morten's code are matrix elements {\em in the HO 
basis}, and the coupling is between s.p. states of the bra and kets. 

The simplest way to proceed is the brute force method (as usual). The coupled 
matrix elements for the RPA are defined from the uncoupled ones as
\begin{multline}
\underset{J}{
\overset{J}{
\acontraction{\langle }{m }{j | \hat{V}^{\text{res}} | }{i}
\bcontraction{\langle m }{j}{| \hat{V}^{\text{res}} i | }{n}
\langle m j | \hat{V}^{\text{res}} | i n \rangle
}
}
=
\sum_{\text{all\ m}} (-1)^{j_{b}-m_{b}+j_{c}-m_{c}}
(j_{a}j_{c}m_{a}-m_{c}|JM)(j_{b}j_{d}m_{b}-m_{d}|JM) \\
\times\langle j_{a}m_{a},j_{b}m_{b} | \hat{V}^{\text{res}} | j_{c}m_{c},j_{d}m_{d}\rangle
\label{eq:matV}
\end{multline}
Now, the matrix elements $V^{\text{res}}_{abcd} = \langle j_{a}m_{a},j_{b}m_{b} | \hat{V}_{\text{res}} | j_{c}m_{c},j_{d}m_{d}\rangle$ 
are uncoupled (in terms of angular momentum), but the states are still HF 
states. However, from the diagonalization of the HF Hamiltonian, you get the 
expansion of these states as function of HO states,
\begin{equation}
|\bar{n}ljm\rangle_{HF} = \sum_{n'} D^{lj}_{n'\bar{n}} |n'ljm\rangle_{HO}
\end{equation}
You can, therefore, express all the matrix elements $V^{\text{res}}_{abcd}$ as 
a function of the original matrix elements of the potential in the HO basis. 

{\bf Practical implementation}
\begin{enumerate}
\item For a given $J$ and parity $\pi$, define the basis of coupled 
particle-hole excitations, i.e., the states $|mi\rangle$ by taking all hole 
states below the Fermi level, and all the particle states up to a cut-off 
$E_{\text{cut}}$ that should be an input of your code. Note that in the various 
expressions given above, I dropped the indices related to the $n$ and $l$ 
quantum numbers. In practice, the HF states coming out as eigenstates of the 
HF matrix are
\begin{eqnarray*}
k=0: & |0,0,1/2\rangle \equiv 0s1/2 \\
k=1: & |0,1,3/2\rangle \equiv 0p3/2 \\
k=2: & |0,1,1/2\rangle \equiv 0p1/2 \\
k=3: & |0,2,3/2\rangle \equiv 0d3/2 \\
k=4: & |0,2,5/2\rangle \equiv 0d5/2 \\
k=5: & |2,0,1/2\rangle \equiv 1s1/2 \\
\vdots & \vdots
\end{eqnarray*}
The first step is, therefore, to set up some bookkeeping mechanism to keep 
track of your s.p. states and compute the p.h. two-body states.
\item Compute the $JJ$ matrix elements according to Eq.(\ref{eq:matV})
\item Compute the matrix elements of your RPA matrix according to 
Eqs.(\ref{eq:A})-(\ref{eq:B})
\item Diagonalize the RPA matrix
\end{enumerate}


%%%%%%%%%%%%%%%%%%%%%%%%%%%%%%%%%%%%%%%%%%%%%%%%%%%%%%%%%%%%%%%%%%%%%%%%%%%%%%%%
%%%%%%%%%%%%%%%%%%%%%%%%%%%%%%%%%%%%%%%%%%%%%%%%%%%%%%%%%%%%%%%%%%%%%%%%%%%%%%%%
%%%%%%%%%%%%%%%%%%%%%%%%%%%%%%%%%%%%%%%%%%%%%%%%%%%%%%%%%%%%%%%%%%%%%%%%%%%%%%%%
%%%%%%%%%%%%%%%%%%%%%%%%%%%%%%%%%%%%%%%%%%%%%%%%%%%%%%%%%%%%%%%%%%%%%%%%%%%%%%%%

\section{Deformed HF equations}

In principle, one could solve the deformed HF equations in the spherical basis 
by ``simply'' adding several loops over $l_{c}$, $j_{c}$, $l_{d}$, $j_{d}$ 
instead of assuming a block diagonal structure of the HF matrix. However, since 
our neutrons are confined in a {\em spherical} trap, one would probably never 
be able to observe symmetry breaking and deformation for the HF potential 
and/or the density. To avoid this, we will introduce the Cartesian version of 
the HO oscillator basis functions and solve the HF functions in this basis. To 
make calculations doable, we also need to explicitly introduce the Moshinsky 
transformation, which, luckily, is simpler in Cartesian coordinates than it is 
in spherical coordinates.

In the following, we will see how to compute the matrix elements of a general 
Gaussian potential in the Cartesian deformed HO basis. The potential is defined 
as
\begin{equation}
\hat{V}(\gras{r}_{1}, \gras{r}_{2}) = 
\sum_{w=1}^{N_{w}} \alpha_{w} e^{-\beta_{w}(\gras{r}_{1} - \gras{r}_{2})^{2}},
\end{equation}
where the dimensions are $[\alpha_{w}] = \text{MeV}$ and 
$[\beta_{w}] = \text{fm}^{-2}$.

{\bf Harmonic Oscillator Basis in Cartesian Coordinates - } The eigenfunctions 
of a spin-less three-dimensional harmonic oscillator (HO) are given by
\begin{equation}
\varphi^{(\gras{b})}_{\gras{n}}(\gras{r}) = \langle \gras{r} | \gras{n} \rangle
\end{equation}
with $\varphi^{(\gras{b})}_{\gras{n}}(\gras{r}) = 
\varphi^{(\gras{b})}_{\gras{n}}(x,y,z)$ and
\begin{equation}
\varphi^{(\gras{b})}_{\gras{n}}(x,y,z) =
\left(\sqrt{b_{x}}e^{-\frac{1}{2}\xi_{x}^{2}}H^{(0)}_{n_{x}}(\xi_{x})\right)
\left(\sqrt{b_{y}}e^{-\frac{1}{2}\xi_{y}^{2}}H^{(0)}_{n_{y}}(\xi_{y})\right)
\left(\sqrt{b_{z}}e^{-\frac{1}{2}\xi_{z}^{2}}H^{(0)}_{n_{z}}(\xi_{z})\right)
\label{HO-3D}
\end{equation}
where
\begin{itemize}
\item The variables in the Hermite polynomials are dimensionless and are 
defined as
\begin{equation}
\xi_{\mu} = b_{\mu}x_{\mu},\ \ \ \ \mu=x,y,z\ \text{and}\ x_{\mu} = x,y,z,
\end{equation}
with the oscillator scales $\gras{b} \equiv (b_{x},b_{y}, b_{z}) $ given by
\begin{equation}
b_{\mu} = \sqrt{\frac{m\omega_{\mu}}{\hbar}},\ \ \ \mu = x,y,z.
\end{equation}
The oscillator scales have dimensions $[ b_{\mu} ] = [\text{fm}]^{-1}$.
\item The Hermite polynomials are normalized
\begin{equation}
H^{(0)}_{n_{\mu}}(\xi_{\mu}) =
\frac{1}{\left( \sqrt{\pi}2^{n_{\mu}}n_{\mu}! \right)^{1/2} }H_{n_{\mu}}(\xi_{\mu}),
\end{equation}
and the $H_{n_{\mu}}(\xi_{\mu})$ are the standard Hermite polynomials as can be 
found, e.g., in Abramovitz \& Stegun, {\em Handbook of mathematical functions}, 
Chapter 22, Eq. 22.2.14. 
\item The spatial quantum numbers are $\gras{n} = ( n_{x}, n_{y}, n_{z})$, and 
the energy of the HO is given by
\begin{equation}
E^{\text{HO}}_{\gras{n}} = 
\hbar\omega_{x}\left( n_{x} + \frac{1}{2} \right)
+
\hbar\omega_{y}\left( n_{y} + \frac{1}{2} \right)
+
\hbar\omega_{z}\left( n_{z} + \frac{1}{2} \right)
\end{equation}
\end{itemize}

{\bf Matrix Elements of the Potential - } The matrix elements of the Gaussian 
potential in the HO basis are
\begin{equation}
\langle \gras{n}'\gras{m}' | \hat{V} | \gras{n}\gras{m}\rangle
=
\langle \gras{n}'\gras{m}' | 
\sum_{w} \alpha_{w} e^{-\beta_{w}(\gras{r}_{1} - \gras{r}_{2})^{2}} 
| \gras{n}\gras{m}\rangle.
\end{equation}
Explicitly, they read
\begin{equation}
\langle \gras{n}'\gras{m}' | \hat{V} | \gras{n}\gras{m}\rangle
= \sum_{w} \prod_{\mu=x,y,z}
\iint dx_{\mu}dx'_{\mu} \; \varphi^{(b_{\mu})}_{n'_{\mu}}(x_{\mu})\varphi^{(b_{\mu})}_{m'_{\mu}}(x'_{\mu})
\alpha_{w}^{1/3} e^{-\beta_{w}(x_{\mu} - x'_{\mu})^{2}} 
\varphi^{(b_{\mu})}_{n_{\mu}}(x_{\mu})\varphi^{(b_{\mu})}_{m_{\mu}}(x'_{\mu})
\end{equation}
In the following, we will focus on the generic matrix elements
\begin{equation}
v_{n'm'nm}^{w}
= \iint dxdx' \; \varphi^{(b)}_{n'}(x)\varphi^{(b)}_{m'}(x)
\alpha_{w}^{1/3} e^{-\beta_{w}(x - x')^{2}} 
\varphi^{(b)}_{n}(x)\varphi^{(b)}_{m}(x')
\label{eq:1DME}
\end{equation}

{\bf Expansion of Hermite Polynomials - } We search for an expansion of the 
product of two Hermite functions in the form
\begin{equation}
\varphi^{(b)}_{n'}(x)\varphi^{(b)}_{n}(x) = \sum_{A=0}^{n'+n} C_{n'n}^{A} \varphi^{(b')}_{A}(x).
\label{eq:2Hermites}
\end{equation}
After some not-so-lengthy algebra, we find
\begin{equation}
C_{n'n}^{A} = 
\sum_{i=1}^{N} w_{i}
\varphi_{n'}^{(b/\sqrt{\alpha})}(x_{i}) 
\varphi_{n}^{(b/\sqrt{\alpha})} (x_{i}) 
\varphi_{A}^{(b'/\sqrt{\alpha})}(x_{i}) 
\alpha^{1/4}
e^{x_{i}^{2}}.
\end{equation}
At this point, we choose $b' = b\sqrt{2}$, hence $\alpha = 2b^{2}$ (and 
$b' = \sqrt{\alpha}$). Our coefficients read
\begin{equation}
C_{n'n}^{A} = 
\sum_{i=1}^{N} \left( b\sqrt{2} \right)^{1/2} w_{i} e^{x_{i}^{2}}\;
\varphi_{n'}^{(1/\sqrt{2})}(x_{i}) 
\varphi_{n}^{(1/\sqrt{2})} (x_{i}) 
\varphi_{A}^{(1)}(x_{i}) .
\end{equation}
Let us then rescale the Gauss-Hermite nodes and weights in the following way:
\begin{equation}
w_{i} \rightarrow \omega_{i} = \frac{w_{i}}{b\sqrt{2}}e^{x_{i}^{2}}, \ \ \ \ 
x_{i} \rightarrow X_{i} = \frac{x_{i}}{b\sqrt{2}}.
\end{equation}
We have
\begin{equation}
\left\{
\begin{array}{l}
\varphi_{n}^{(1/\sqrt{2})}(x_{i}) 
= \displaystyle 
\frac{1}{2^{1/4}} H_{n}^{(0)}\left(\frac{x_{i}}{\sqrt{2}}\right) e^{-\frac{1}{2}\left(\frac{x_{i}}{\sqrt{2}} \right)^{2}}
= \frac{1}{\sqrt{b\sqrt{2}}} \varphi_{n}^{(b)}(X_{i}) \medskip\\
\varphi_{B}^{(1)}(x_{i}) 
= \displaystyle 
H_{B}^{(0)}(x_{i}) e^{-\frac{1}{2}x_{i}^{2}}
= \frac{1}{\sqrt{b\sqrt{2}}} \varphi_{B}^{(b\sqrt{2})}(X_{i}) \medskip\\
\end{array}
\right.
\end{equation}
After these simplifications, we find
\begin{equation}
C_{n'n}^{A} = 
\sum_{i=1}^{N} \omega_{i}
\varphi_{n'}^{(b)}(X_{i}) 
\varphi_{n}^{(b)} (X_{i}) 
\varphi_{A}^{(b\sqrt{2})}(X_{i}).
\end{equation}

{\bf Moshinsky Transformation - } After introducing the expansion 
(\ref{eq:2Hermites}) of the two product of two Hermite functions in the matrix 
element (\ref{eq:1DME}), we arrive at
\begin{equation}
v_{n'm'nm}^{w}
= \alpha_{w}^{1/3} \sum_{AB} C_{n'n}^{A}C_{m'm'}^{B} \iint dxdx' \; 
\varphi^{(b\sqrt{2})}_{A}(x)e^{-\beta_{w}(x - x')^{2}} 
\varphi^{(b\sqrt{2})}_{B}(x').
\end{equation}
The Moshinsky transformation consists in introducing the variables
\begin{equation}
\left\{
\begin{array}{l}
U = \displaystyle\frac{1}{\sqrt{2}} (x + x') \medskip\\
u = \displaystyle\frac{1}{\sqrt{2}} (x - x')
\end{array}
\right.
\end{equation}
The Jacobian of this transformation is 1. Note that, in our case, since $x$ and 
$x'$ are in fermis, so are the $U$ and $u$ variables. In fact, the Moshinsky 
transformation below can be applied either on the normalized Hermite 
polynomials, $H_{n}^{(0)}$, or on the Hermite functions, $\varphi_{n}$, 
irrespective of whether the latter are properly normalized by the $\sqrt{b}$ 
scale, in exactly the same way. The new variables $(U,u)$ have always the same 
units as the old one $(x,x')$. In our case, we apply the transformation on the 
fully normalized Hermite functions to find
\begin{equation}
\varphi^{(b\sqrt{2})}_{A}(x)\varphi^{(b\sqrt{2})}_{B}(x')
=
\sum_{N,n} D_{AB}^{Nn}\; \varphi^{(b\sqrt{2})}_{N}(U)\varphi^{(b\sqrt{2})}_{n}(u).
\end{equation}
Hence, our matrix element becomes
\begin{equation}
v_{n'm'nm}^{w}
= \alpha_{w}^{1/3} \sum_{AB} C_{n'n}^{A}C_{m'm'}^{B} \sum_{N,n} D_{AB}^{Nn}
I_{N} J_{n}
\end{equation}
with
\begin{equation}
\begin{array}{l}
I_{N} = \displaystyle\int dU\; \varphi^{(b\sqrt{2})}_{N}(U) \medskip\\
J_{n} = \displaystyle\int du\; e^{-2\beta_{w}u^{2}} \varphi^{(b\sqrt{2})}_{n}(u).
\end{array}
\end{equation}

{\bf Calculation of the integrals - } We give below the analytical results for 
the integrals [You can calculate them yourselves as a very simple and very 
boring little exercise]. We have
\begin{equation}
I_{N} 
= (-1)^{N}\left( \frac{\sqrt{2}}{b} \right)^{1/2}
\frac{\sqrt{N!}}{2^{N/2}(N/2)!}.
\end{equation}
and
\begin{equation}
J_{n} 
= \frac{1}{\sqrt{\alpha}}\sum_{n'} D_{nn'}\left( \frac{1}{\alpha} \right)  I_{n'}
\end{equation}
where the $D_{nn'}$ matrices are defined by
\begin{equation}
H_{n}^{(0)}\left(\frac{x}{b}\right) =
\sqrt{\frac{b}{b'}} \sum_{n'} D_{nn'}\left( \frac{b'}{b} \right) 
H_{n}^{(0)}\left(\frac{x}{b'}\right)
\end{equation}

{\bf Summary - } We define the auxiliary integral $G_{AB}$ as
\begin{equation}
G_{AB}^{w}
= \alpha_{w}^{1/3}\sum_{Nn} D_{AB}^{Nn} I_{N} J_{n}
= \alpha_{w}^{1/3}\frac{1}{\sqrt{\alpha}} \sum_{Nn} D_{AB}^{Nn} I_{N} 
\sum_{n'} D_{nn'}\left( \frac{1}{\alpha} \right)  I_{n'}
\label{eq:G}
\end{equation}
The matrix element of a single Gaussian $w$ in one-dimension reads
\begin{equation}
v^{w}_{n'm'nm} = \sum_{AB} C_{n'n}^{A}C_{m'm'}^{B} G_{AB}^{w}
\end{equation}
In three dimensions, we have
\begin{equation}
V^{w}_{\gras{n'}\gras{m'}\gras{n}\gras{m}} = 
v^{w}_{n'_{x}m'_{x}n_{x}m_{x}}
v^{w}_{n'_{y}m'_{y}n_{y}m_{y}}
v^{w}_{n'_{z}m'_{z}n_{z}m_{z}},
\end{equation}
which expands into
\begin{equation}
V^{w}_{\gras{n'}\gras{m'}\gras{n}\gras{m}} 
=
\sum_{A_{x}B_{x}} C_{n'_{x}n_{x}}^{A_{x}}C_{m'_{x}m_{x}}^{B_{x}} G_{A_{x}B_{x}}^{w}
\sum_{A_{y}B_{y}} C_{n'_{y}n_{y}}^{A_{y}}C_{m'_{y}m_{y}}^{B_{y}} G_{A_{y}B_{y}}^{w}
\sum_{A_{z}B_{z}} C_{n'_{z}n_{z}}^{A_{z}}C_{m'_{z}m_{z}}^{B_{z}} G_{A_{z}B_{z}}^{w}
\label{eq:TBME}
\end{equation}

{\bf Practical implementation}
\begin{enumerate}
\item Set up a new module/class/set of routines to compute Gauss-Hermite 
quadratures and the HO functions in Cartesian coordinates. Some unit tests 
to check the accuracy of GH quadratures would be welcome.
\item In the next step, you should compute the various coefficients appearing 
in Eqs.(\ref{eq:G})-(\ref{eq:TBME}). The coefficients $D_{AB}^{Nn}$ and 
$C_{n'n}^{A}$, and the integrals $I_{N}$ and $J_{n}$ can be precalculated once 
and for all.
\item In Cartesian coordinates, pre-computing all matrix elements and storing 
them on disk (or in memory) is {\em not a viable option}. There are 12 indices 
involved, the $n_{\mu}, \mu = x, y,z $ for each of the 4 s.p. states $a$, $b$, 
$c$ and $d$, and both the CPU time and the disk space needed to store all 
matrix elements would be prohibitive. In current DFT solvers implementing a 
finite-range force (such as Gogny) in the Cartesian HO basis, the {\em HF 
potential and two-body matrix elements are computed on the fly} at each 
iteration as 
\begin{equation}
\Gamma_{ac} = \sum_{bd} \bar{v}_{abcd}\rho_{db}
\end{equation}
Even then, you need to carefully think of how you will set up your loops in 
order to obtain a reasonable compute time.
\item The rest of the HF loop is no different from the spherical case. Note 
that each HF state is not characterized by any particular quantum number.
\end{enumerate}




%%%%%%%%%%%%%%%%%%%%%%%%%%%%%%%%%%%%%%%%%%%%%%%%%%%%%%%%%%%%%%%%%%%%%%%%%%%%%%%%
%%%%%%%%%%%%%%%%%%%%%%%%%%%%%%%%%%%%%%%%%%%%%%%%%%%%%%%%%%%%%%%%%%%%%%%%%%%%%%%%
%%%%%%%%%%%%%%%%%%%%%%%%%%%%%%%%%%%%%%%%%%%%%%%%%%%%%%%%%%%%%%%%%%%%%%%%%%%%%%%%
%%%%%%%%%%%%%%%%%%%%%%%%%%%%%%%%%%%%%%%%%%%%%%%%%%%%%%%%%%%%%%%%%%%%%%%%%%%%%%%%


\end{document}
