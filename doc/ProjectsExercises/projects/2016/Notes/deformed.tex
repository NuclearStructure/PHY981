In principle, one could solve the deformed HF equations in the spherical basis 
by ``simply'' adding several loops over $l_{c}$, $j_{c}$, $l_{d}$, $j_{d}$ 
instead of assuming a block diagonal structure of the HF matrix. However, since 
our neutrons are confined in a {\em spherical} trap, one would probably never 
be able to observe symmetry breaking and deformation for the HF potential 
and/or the density. To avoid this, we will introduce the Cartesian version of 
the HO oscillator basis functions and solve the HF functions in this basis. To 
make calculations doable, we also need to explicitly introduce the Moshinsky 
transformation, which, luckily, is simpler in Cartesian coordinates than it is 
in spherical coordinates.

In the following, we will see how to compute the matrix elements of a general 
Gaussian potential in the Cartesian deformed HO basis. The potential is defined 
as
\begin{equation}
\hat{V}(\gras{r}_{1}, \gras{r}_{2}) = 
\sum_{w=1}^{N_{w}} \alpha_{w} e^{-\beta_{w}(\gras{r}_{1} - \gras{r}_{2})^{2}},
\end{equation}
where the dimensions are $[\alpha_{w}] = \text{MeV}$ and 
$[\beta_{w}] = \text{fm}^{-2}$.

{\bf Harmonic Oscillator Basis in Cartesian Coordinates - } The eigenfunctions 
of a spin-less three-dimensional harmonic oscillator (HO) are given by
\begin{equation}
\varphi^{(\gras{b})}_{\gras{n}}(\gras{r}) = \langle \gras{r} | \gras{n} \rangle
\end{equation}
with $\varphi^{(\gras{b})}_{\gras{n}}(\gras{r}) = 
\varphi^{(\gras{b})}_{\gras{n}}(x,y,z)$ and
\begin{equation}
\varphi^{(\gras{b})}_{\gras{n}}(x,y,z) =
\left(\sqrt{b_{x}}e^{-\frac{1}{2}\xi_{x}^{2}}H^{(0)}_{n_{x}}(\xi_{x})\right)
\left(\sqrt{b_{y}}e^{-\frac{1}{2}\xi_{y}^{2}}H^{(0)}_{n_{y}}(\xi_{y})\right)
\left(\sqrt{b_{z}}e^{-\frac{1}{2}\xi_{z}^{2}}H^{(0)}_{n_{z}}(\xi_{z})\right)
\label{HO-3D}
\end{equation}
where
\begin{itemize}
\item The variables in the Hermite polynomials are dimensionless and are 
defined as
\begin{equation}
\xi_{\mu} = b_{\mu}x_{\mu},\ \ \ \ \mu=x,y,z\ \text{and}\ x_{\mu} = x,y,z,
\end{equation}
with the oscillator scales $\gras{b} \equiv (b_{x},b_{y}, b_{z}) $ given by
\begin{equation}
b_{\mu} = \sqrt{\frac{m\omega_{\mu}}{\hbar}},\ \ \ \mu = x,y,z.
\end{equation}
The oscillator scales have dimensions $[ b_{\mu} ] = [\text{fm}]^{-1}$.
\item The Hermite polynomials are normalized
\begin{equation}
H^{(0)}_{n_{\mu}}(\xi_{\mu}) =
\frac{1}{\left( \sqrt{\pi}2^{n_{\mu}}n_{\mu}! \right)^{1/2} }H_{n_{\mu}}(\xi_{\mu}),
\end{equation}
and the $H_{n_{\mu}}(\xi_{\mu})$ are the standard Hermite polynomials as can be 
found, e.g., in Abramovitz \& Stegun, {\em Handbook of mathematical functions}, 
Chapter 22, Eq. 22.2.14. 
\item The spatial quantum numbers are $\gras{n} = ( n_{x}, n_{y}, n_{z})$, and 
the energy of the HO is given by
\begin{equation}
E^{\text{HO}}_{\gras{n}} = 
\hbar\omega_{x}\left( n_{x} + \frac{1}{2} \right)
+
\hbar\omega_{y}\left( n_{y} + \frac{1}{2} \right)
+
\hbar\omega_{z}\left( n_{z} + \frac{1}{2} \right)
\end{equation}
\end{itemize}

{\bf Matrix Elements of the Potential - } The matrix elements of the Gaussian 
potential in the HO basis are
\begin{equation}
\langle \gras{n}'\gras{m}' | \hat{V} | \gras{n}\gras{m}\rangle
=
\langle \gras{n}'\gras{m}' | 
\sum_{w} \alpha_{w} e^{-\beta_{w}(\gras{r}_{1} - \gras{r}_{2})^{2}} 
| \gras{n}\gras{m}\rangle.
\end{equation}
Explicitly, they read
\begin{equation}
\langle \gras{n}'\gras{m}' | \hat{V} | \gras{n}\gras{m}\rangle
= \sum_{w} \prod_{\mu=x,y,z}
\iint dx_{\mu}dx'_{\mu} \; \varphi^{(b_{\mu})}_{n'_{\mu}}(x_{\mu})\varphi^{(b_{\mu})}_{m'_{\mu}}(x'_{\mu})
\alpha_{w}^{1/3} e^{-\beta_{w}(x_{\mu} - x'_{\mu})^{2}} 
\varphi^{(b_{\mu})}_{n_{\mu}}(x_{\mu})\varphi^{(b_{\mu})}_{m_{\mu}}(x'_{\mu})
\end{equation}
In the following, we will focus on the generic matrix elements
\begin{equation}
v_{n'm'nm}^{w}
= \iint dxdx' \; \varphi^{(b)}_{n'}(x)\varphi^{(b)}_{m'}(x)
\alpha_{w}^{1/3} e^{-\beta_{w}(x - x')^{2}} 
\varphi^{(b)}_{n}(x)\varphi^{(b)}_{m}(x')
\label{eq:1DME}
\end{equation}

{\bf Expansion of Hermite Polynomials - } We search for an expansion of the 
product of two Hermite functions in the form
\begin{equation}
\varphi^{(b)}_{n'}(x)\varphi^{(b)}_{n}(x) = \sum_{A=0}^{n'+n} C_{n'n}^{A} \varphi^{(b')}_{A}(x).
\label{eq:2Hermites}
\end{equation}
After some not-so-lengthy algebra, we find
\begin{equation}
C_{n'n}^{A} = 
\sum_{i=1}^{N} w_{i}
\varphi_{n'}^{(b/\sqrt{\alpha})}(x_{i}) 
\varphi_{n}^{(b/\sqrt{\alpha})} (x_{i}) 
\varphi_{A}^{(b'/\sqrt{\alpha})}(x_{i}) 
\alpha^{1/4}
e^{x_{i}^{2}}.
\end{equation}
At this point, we choose $b' = b\sqrt{2}$, hence $\alpha = 2b^{2}$ (and 
$b' = \sqrt{\alpha}$). Our coefficients read
\begin{equation}
C_{n'n}^{A} = 
\sum_{i=1}^{N} \left( b\sqrt{2} \right)^{1/2} w_{i} e^{x_{i}^{2}}\;
\varphi_{n'}^{(1/\sqrt{2})}(x_{i}) 
\varphi_{n}^{(1/\sqrt{2})} (x_{i}) 
\varphi_{A}^{(1)}(x_{i}) .
\end{equation}
Let us then rescale the Gauss-Hermite nodes and weights in the following way:
\begin{equation}
w_{i} \rightarrow \omega_{i} = \frac{w_{i}}{b\sqrt{2}}e^{x_{i}^{2}}, \ \ \ \ 
x_{i} \rightarrow X_{i} = \frac{x_{i}}{b\sqrt{2}}.
\end{equation}
We have
\begin{equation}
\left\{
\begin{array}{l}
\varphi_{n}^{(1/\sqrt{2})}(x_{i}) 
= \displaystyle 
\frac{1}{2^{1/4}} H_{n}^{(0)}\left(\frac{x_{i}}{\sqrt{2}}\right) e^{-\frac{1}{2}\left(\frac{x_{i}}{\sqrt{2}} \right)^{2}}
= \frac{1}{\sqrt{b\sqrt{2}}} \varphi_{n}^{(b)}(X_{i}) \medskip\\
\varphi_{B}^{(1)}(x_{i}) 
= \displaystyle 
H_{B}^{(0)}(x_{i}) e^{-\frac{1}{2}x_{i}^{2}}
= \frac{1}{\sqrt{b\sqrt{2}}} \varphi_{B}^{(b\sqrt{2})}(X_{i}) \medskip\\
\end{array}
\right.
\end{equation}
After these simplifications, we find
\begin{equation}
C_{n'n}^{A} = 
\sum_{i=1}^{N} \omega_{i}
\varphi_{n'}^{(b)}(X_{i}) 
\varphi_{n}^{(b)} (X_{i}) 
\varphi_{A}^{(b\sqrt{2})}(X_{i}).
\end{equation}

{\bf Moshinsky Transformation - } After introducing the expansion 
(\ref{eq:2Hermites}) of the two product of two Hermite functions in the matrix 
element (\ref{eq:1DME}), we arrive at
\begin{equation}
v_{n'm'nm}^{w}
= \alpha_{w}^{1/3} \sum_{AB} C_{n'n}^{A}C_{m'm'}^{B} \iint dxdx' \; 
\varphi^{(b\sqrt{2})}_{A}(x)e^{-\beta_{w}(x - x')^{2}} 
\varphi^{(b\sqrt{2})}_{B}(x').
\end{equation}
The Moshinsky transformation consists in introducing the variables
\begin{equation}
\left\{
\begin{array}{l}
U = \displaystyle\frac{1}{\sqrt{2}} (x + x') \medskip\\
u = \displaystyle\frac{1}{\sqrt{2}} (x - x')
\end{array}
\right.
\end{equation}
The Jacobian of this transformation is 1. Note that, in our case, since $x$ and 
$x'$ are in fermis, so are the $U$ and $u$ variables. In fact, the Moshinsky 
transformation below can be applied either on the normalized Hermite 
polynomials, $H_{n}^{(0)}$, or on the Hermite functions, $\varphi_{n}$, 
irrespective of whether the latter are properly normalized by the $\sqrt{b}$ 
scale, in exactly the same way. The new variables $(U,u)$ have always the same 
units as the old one $(x,x')$. In our case, we apply the transformation on the 
fully normalized Hermite functions to find
\begin{equation}
\varphi^{(b\sqrt{2})}_{A}(x)\varphi^{(b\sqrt{2})}_{B}(x')
=
\sum_{N,n} D_{AB}^{Nn}\; \varphi^{(b\sqrt{2})}_{N}(U)\varphi^{(b\sqrt{2})}_{n}(u).
\end{equation}
Hence, our matrix element becomes
\begin{equation}
v_{n'm'nm}^{w}
= \alpha_{w}^{1/3} \sum_{AB} C_{n'n}^{A}C_{m'm'}^{B} \sum_{N,n} D_{AB}^{Nn}
I_{N} J_{n}
\end{equation}
with
\begin{equation}
\begin{array}{l}
I_{N} = \displaystyle\int dU\; \varphi^{(b\sqrt{2})}_{N}(U) \medskip\\
J_{n} = \displaystyle\int du\; e^{-2\beta_{w}u^{2}} \varphi^{(b\sqrt{2})}_{n}(u).
\end{array}
\end{equation}

{\bf Calculation of the integrals - } We give below the analytical results for 
the integrals [You can calculate them yourselves as a very simple and very 
boring little exercise]. We have
\begin{equation}
I_{N} 
= (-1)^{N}\left( \frac{\sqrt{2}}{b} \right)^{1/2}
\frac{\sqrt{N!}}{2^{N/2}(N/2)!}.
\end{equation}
and
\begin{equation}
J_{n} 
= \frac{1}{\sqrt{\alpha}}\sum_{n'} D_{nn'}\left( \frac{1}{\alpha} \right)  I_{n'}
\end{equation}
where the $D_{nn'}$ matrices are defined by
\begin{equation}
H_{n}^{(0)}\left(\frac{x}{b}\right) =
\sqrt{\frac{b}{b'}} \sum_{n'} D_{nn'}\left( \frac{b'}{b} \right) 
H_{n}^{(0)}\left(\frac{x}{b'}\right)
\end{equation}

{\bf Summary - } We define the auxiliary integral $G_{AB}$ as
\begin{equation}
G_{AB}^{w}
= \alpha_{w}^{1/3}\sum_{Nn} D_{AB}^{Nn} I_{N} J_{n}
= \alpha_{w}^{1/3}\frac{1}{\sqrt{\alpha}} \sum_{Nn} D_{AB}^{Nn} I_{N} 
\sum_{n'} D_{nn'}\left( \frac{1}{\alpha} \right)  I_{n'}
\label{eq:G}
\end{equation}
The matrix element of a single Gaussian $w$ in one-dimension reads
\begin{equation}
v^{w}_{n'm'nm} = \sum_{AB} C_{n'n}^{A}C_{m'm'}^{B} G_{AB}^{w}
\end{equation}
In three dimensions, we have
\begin{equation}
V^{w}_{\gras{n'}\gras{m'}\gras{n}\gras{m}} = 
v^{w}_{n'_{x}m'_{x}n_{x}m_{x}}
v^{w}_{n'_{y}m'_{y}n_{y}m_{y}}
v^{w}_{n'_{z}m'_{z}n_{z}m_{z}},
\end{equation}
which expands into
\begin{equation}
V^{w}_{\gras{n'}\gras{m'}\gras{n}\gras{m}} 
=
\sum_{A_{x}B_{x}} C_{n'_{x}n_{x}}^{A_{x}}C_{m'_{x}m_{x}}^{B_{x}} G_{A_{x}B_{x}}^{w}
\sum_{A_{y}B_{y}} C_{n'_{y}n_{y}}^{A_{y}}C_{m'_{y}m_{y}}^{B_{y}} G_{A_{y}B_{y}}^{w}
\sum_{A_{z}B_{z}} C_{n'_{z}n_{z}}^{A_{z}}C_{m'_{z}m_{z}}^{B_{z}} G_{A_{z}B_{z}}^{w}
\label{eq:TBME}
\end{equation}

{\bf Practical implementation}
\begin{enumerate}
\item Set up a new module/class/set of routines to compute Gauss-Hermite 
quadratures and the HO functions in Cartesian coordinates. Some unit tests 
to check the accuracy of GH quadratures would be welcome.
\item In the next step, you should compute the various coefficients appearing 
in Eqs.(\ref{eq:G})-(\ref{eq:TBME}). The coefficients $D_{AB}^{Nn}$ and 
$C_{n'n}^{A}$, and the integrals $I_{N}$ and $J_{n}$ can be precalculated once 
and for all.
\item In Cartesian coordinates, pre-computing all matrix elements and storing 
them on disk (or in memory) is {\em not a viable option}. There are 12 indices 
involved, the $n_{\mu}, \mu = x, y,z $ for each of the 4 s.p. states $a$, $b$, 
$c$ and $d$, and both the CPU time and the disk space needed to store all 
matrix elements would be prohibitive. In current DFT solvers implementing a 
finite-range force (such as Gogny) in the Cartesian HO basis, the {\em HF 
potential and two-body matrix elements are computed on the fly} at each 
iteration as 
\begin{equation}
\Gamma_{ac} = \sum_{bd} \bar{v}_{abcd}\rho_{db}
\end{equation}
Even then, you need to carefully think of how you will set up your loops in 
order to obtain a reasonable compute time.
\item The rest of the HF loop is no different from the spherical case. Note 
that each HF state is not characterized by any particular quantum number.
\end{enumerate}


