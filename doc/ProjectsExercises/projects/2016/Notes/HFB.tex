{\bf HFB equations - } Remember that the most general form of the HFB equations 
is
\begin{equation}
\left( \begin{array}{cc}
h - \lambda & \Delta \\
-\Delta^{*} & -h^{*} + \lambda
\end{array} \right)
\left( \begin{array}{cc} 
U & V^{*} \\ 
V & U^{*} 
\end{array} \right)
=
\left( \begin{array}{cc} 
U & V^{*} \\ 
V & U^{*} 
\end{array} \right)
\left( \begin{array}{cc}
E & 0 \\
0 & -E
\end{array} \right)
\end{equation}
In this expression, $h$ and $\Delta$ are matrices of size $n\times n$, with $n$ 
the size of the single-particle basis (i.e. the total number of states in the 
HO basis in our example). $\lambda$ is a shorthand notation for $\lambda I$, 
with $I$ the $n\times n$ identity matrix; $\lambda$ is the Fermi level 
introduced to constrain the average value of the particle number to its actual 
value. For each of the $n$ eigenvectors of energy $E_{\mu}$, there is one with 
eigenvector $-E_{\mu}$. 

{\bf Bogoliubov transformation - } In spherical symmetry, the conjugate 
single-particle states $a$ and $\bar{a}$ are characterized by 
$|a\rangle \equiv |n_{a} l_{a} j_{a} m_{a}\rangle$ and 
$|\bar{a}\rangle \equiv |n_{a} l_{a} j_{a} -m_{a}\rangle$. As Peter discussed 
in this lecture on pairing, $a$ and $\bar{a}$ are related by the time-reversal 
operator. Just as in the HF case, the $U$ and $V$ matrices are block diagonal, 
i.e., they take the generic form
\begin{equation}
U_{ab} \equiv \delta_{l_{a}l_{b}} \delta_{j_{a}j_{b}} U^{(l_{a}j_{a})}, 
\ \ \ 
V_{ab} \equiv \delta_{l_{a}l_{b}} \delta_{j_{a}j_{b}} V^{(l_{a}j_{a})}. 
\end{equation}
The difference with the HF case is that things are not entirely independent of 
the projection $m$. Suppose we reorder the labelling of the $2j_{a}+1$ 
$m$-projections in each $(l_{a},j_{a})$ block according to
\begin{equation}
m = -j, -j+1, \dots, +j \rightarrow m = +j,-j, +j-1, -j+1, \dots, +1/2, -1/2.
\end{equation}
In other words, we form pairs of states $(+m,-m)$; there are $j+1/2$ such 
pairs in each block $j$. Spherical symmetry imposes that the $U$ and $V$ 
matrices are block diagonal in each of these $j+1/2$ blocks. Denoting 
generically $U^{(lj)}_{|m|}$ and $V^{(lj)}_{|m|}$ such blocks, we find
\begin{equation}
U^{(lj)}_{|m|} =
\left(
\begin{array}{cc}
u & 0 \\
0 & u
\end{array}
\right),
\ \ \ \ \ 
V^{(lj)}_{|m|} =
\left(
\begin{array}{cc}
0 & v\\
\bar{v} & 0 \\
\end{array}
\right).
\end{equation}
For each s.p. state $a$, we will note
\begin{equation}
u \equiv u^{(a)}, 
\ \ \ 
v \equiv (-1)^{j_{a}-m_{a}} v^{(a)},
\ \ \ 
\bar{v} \equiv (-1)^{j_{a}+m_{a}} v^{(a)},
\end{equation}
The dimension of the block matrices $u^{(a)}$ and $v^{(a)}$ is 
$(N_{0} - l_{a})/2$, where $N_{0}$ is the 
number of oscillator shells.

{\bf Densities - } From these relations, it is straightforward to compute the 
density matrix $\rho$ and pairing tensor $\kappa$. Remember the general 
definition,
\begin{equation}
\rho_{ab} = (V^{*}V^{T})_{ab}, \ \ \ \ \kappa_{ab} = (V^{*}U^{T})_{ab}
\end{equation}
The density matrix and pairing tensor have a similar block structure as the 
$U$ and $V$ in the (reordered) s.p. basis. We find
\begin{equation}
\rho^{(lj)}_{|m|} = 
\left(
\begin{array}{cc}
\rho & 0 \\
0 & \rho
\end{array}
\right),\ \ \ \ 
\kappa^{(lj)}_{|m|} = 
\left(
\begin{array}{cc}
0 & \kappa \\
\bar{\kappa} & 0
\end{array}
\right),
\end{equation}
with 
\begin{equation}
\rho \equiv v^{(a)}v^{(a)T}, 
\ \ \ 
\kappa \equiv (-1)^{j_{a}-m_{a}} v^{(a)}u^{(a)T},
\ \ \ 
\bar{\kappa} \equiv (-1)^{j_{a}+m_{a}} v^{(a)}u^{(a)T},
\end{equation}

{\bf HFB equations - } Based on the previous remarks, one can show (i) first 
that the HFB equations can also be reduced to a block diagonal form in each of 
the $j+1/2$ blocks characterized by $l$, $j$ and $|m|$, (ii) then that these 
new equations can be further reduced so that they also become block-diagonal in 
the subspace of $(+m,-m)$. The end result is that the HFB equations take the 
following form
\begin{equation}
\left( \begin{array}{cc}
h^{(a)} - \lambda & -\Delta^{(a)} \\
-\Delta^{(a)} & -h^{(a)} + \lambda
\end{array} \right)
\left( \begin{array}{cc} 
u^{(a)} \\ 
v^{(a)} 
\end{array} \right)
= E^{(a)}
\left( \begin{array}{c} 
u^{(a)} \\ 
v^{(a)} 
\end{array} \right)
\end{equation}
where the $u^{(a)}$ and $v^{(a)}$ have been introduced before. Denoting
\begin{equation}
\rho^{(a)} = v^{(a)}v^{(a)T}, \ \ \ \kappa^{(a)} = v^{(a)}u^{(a)T}
\end{equation}
we find the mean-field,
\begin{equation}
h_{n_{a}n_{c}}^{(a)} = t_{n_{a}n_{c}}^{(a)} + \Gamma_{n_{a}n_{c}}^{(a)} 
\end{equation}
the HF potential,
\begin{equation}
\Gamma_{n_{a}n_{c}}^{(a)} =
\sum_{n_{b}n_{d}} 
\langle n_{a}m_{a} n_{b}m_{a} | \bar{v} | n_{c}m_{a} n_{d}m_{a}\rangle
\rho^{(a)}_{n_{d}n_{a}}
\end{equation}
and the pairing field
\begin{equation}
\Delta_{n_{a}n_{c}}^{(a)} =
\sum_{n_{b}n_{d}} 
\langle n_{a}m_{a} n_{b}-m_{a} | \bar{v} | n_{c}m_{a} n_{d}-m_{a}\rangle
\kappa^{(a)}_{n_{d}n_{a}}
\end{equation}


{\bf Practical implementation}
\begin{enumerate}
\item In your HF code, you were already dealing with the blocks of the density 
matrix, which I denoted by $\rho^{(a)}$; you must now introduce another such 
object that will contain the pairing tensor $\kappa^{(a)}$. This object will 
also have to be initialized before to start the HFB iterations [See what 
happens if you initialize it with zeros only].
\item The number of particles is not conserved in HFB. You must, therefore, not 
forget to readjust $\lambda$ at each iteration. This is done by requesting that 
\begin{equation}
\text{Tr} \rho = A
\end{equation}
where $A$ is your input particle number. The simplest way to do this is to use  
a BCS-like expression for quasiparticle occupations. At a given iteration, we 
thus define $\bar{\varepsilon}_{n}$ and $\bar{\Delta}_{n}$ 
according to
\begin{eqnarray}
\displaystyle E_{n} &=& \displaystyle \sqrt{ (\bar{\varepsilon}_{n} - \lambda)^{2} + \bar{\Delta}_{n}^{2} } \medskip \\
\displaystyle N_{n} &=& \displaystyle \frac{1}{2} \left[ 1 - \frac{\bar{\varepsilon}_{n} - \lambda}{\sqrt{ (\bar{\varepsilon}_{n} - \lambda)^{2} + \bar{\Delta}_{n}^{2} }} \right]
\end{eqnarray}
where $E_{n}$ is the energy of quasiparticle number $n$, 
$E_{n} \equiv E^{(l_{a}j_{a})}$, and $N_{n}$ is the norm of the $V$ matrix for 
this q.p.
\begin{equation}
N_{n} \equiv N^{(l_{a}j_{a})}= \sum_{n'} |v^{(a)}_{n'n}|^{2}
\end{equation}
[recall that you should get 
$\sum_{n'} |u^{(a)}_{n'n}|^{2} + |v^{(a)}_{n'n}|^{2} = 1$ from the 
diagonalization of the HFB matrix]. By computing the particle number as
\begin{equation}
A(\lambda) = \sum_{l_{a}j_{a}} (2j_{a}+1) N^{(l_{a}j_{a})}
\end{equation}
we can set up a Newton-like method to obtain $\lambda$.
\item The corollary of the previous steps is that you do not need to impose 
that only the first $n_{occ}$ occupied states are included when defining the 
density matrix.
\end{enumerate}

