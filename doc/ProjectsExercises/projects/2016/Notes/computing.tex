\documentclass[mode=present,paper=screen,size=12pt,style=paintings]{powerdot}

% Useful packages
\usepackage{amsmath,array,theorem,amssymb,mathrsfs}
\usepackage{empheq}
\usepackage{graphicx}
\usepackage{cprotect}
\usepackage{listings}
\lstset{
    language=C,
    basicstyle=\scriptsize,
    frame=single
}
% Useful commands
\newcommand{\gras}[1]{\boldsymbol{#1}}
\newcommand{\Tr}{\mathrm{Tr}}
\newcommand{\boite}[1]{\fbox{\parbox{\textwidth}{{\tt\color{blue} #1}}}}
\newcommand{\cc}[1]{{\tt\color{blue} #1}}

\renewcommand{\arraystretch}{0.5}



\pdsetup{method=direct,palette=MayThird,lf={Computing Environment},rf=\thepage}

\title{Programming Environment for the TALENT School}
\author{N. Schunck}
\date{July 15, 2014}

\begin{document}
\maketitle

%%%%%%%%%%%%%%%%%%%%%%%%%%%%%%%%%%%%%%%%%%%%%%%%%%%%%%%%%%%%%%%%%%%%%%%%%%%%%%%
%%%%%%%%%%%%%%%%%%%%%%%%%%%%%%%%%%%%%%%%%%%%%%%%%%%%%%%%%%%%%%%%%%%%%%%%%%%%%%%
%%%%%%%%%%%%%%%%%%%%%%%%%%%%%%%%%%%%%%%%%%%%%%%%%%%%%%%%%%%%%%%%%%%%%%%%%%%%%%%
%%%%%%%%%%%%%%%%%%%%%%%%%%%%%%%%%%%%%%%%%%%%%%%%%%%%%%%%%%%%%%%%%%%%%%%%%%%%%%%


\begin{slide}{General Requirements for Computational Projects}

\begin{itemize}
\item All programs must be controlled with the git control version system;
\item All programs must be compiled with a Makefile;
\item Your programs must be {\em modular} as much and as reasonably as possible;
\item We ask that you try to use concepts of {\em object-oriented programming (OOP)}. If you use languages like C++ or Python, this is easy; in Fortran or C, it is less obvious but still doable;
\item Fortran 77 is prohibited;
\item As much as possible, try to use {\em libraries} such as BLAS, LAPACK (Fortran), Armadillo (C/C++), etc.;
\item As much as possible, try to use {\em dynamic memory allocation};
\item All codes must be fully {\em documented}, both in the source code and in the form of a short manual -- or a long README, explaining how to build and run the code.
\end{itemize}

\end{slide}

%%%%%%%%%%%%%%%%%%%%%%%%%%%%%%%%%%%%%%%%%%%%%%%%%%%%%%%%%%%%%%%%%%%%%%%%%%%%%%%
%%%%%%%%%%%%%%%%%%%%%%%%%%%%%%%%%%%%%%%%%%%%%%%%%%%%%%%%%%%%%%%%%%%%%%%%%%%%%%%
%%%%%%%%%%%%%%%%%%%%%%%%%%%%%%%%%%%%%%%%%%%%%%%%%%%%%%%%%%%%%%%%%%%%%%%%%%%%%%%
%%%%%%%%%%%%%%%%%%%%%%%%%%%%%%%%%%%%%%%%%%%%%%%%%%%%%%%%%%%%%%%%%%%%%%%%%%%%%%%


\section{Makefiles}


\begin{slide}{Introducing Makefiles}

Our (sensible) requirement that your project must be modular implies that you will have several files to compile, some possibly depending on one or several of the others. The \cc{make} utility is an excellent way to automatize the compilation of such complex projects. \medskip

First things first: \cc{make} may not be installed by default in your system, so start by installing it. \medskip

We will assume a simple C program composed of 3 files
\begin{itemize}
\item the file \cc{main.c} containing the \cc{main()} function
\item the file \cc{hello.c} containing a function displaying {\em Hello, World!}
\item the file \cc{hello.h} containing a header
\end{itemize}

Without any Makefile, the program could be compiled by 

\boite{gcc -o main main.c hello.c -I.}

\end{slide}


\begin{slide}{C Programs Illustrating the Makefile}

\begin{tabular}[t]{p{0.45\textwidth}p{0.45\textwidth}}
\cc{main.c} & \cc{hello.c} \\
\begin{lstlisting}
#include 
int main() 
{
  // call the function
  myPrintHelloMake();

  return(0);
}
\end{lstlisting}
&
\begin{lstlisting}
#include 
#include 

void myPrintHelloMake(void) {

  printf("Hello makefiles!\n");

  return;
}
\end{lstlisting}
\\
\cc{hello.h} & \\
\begin{lstlisting}
/*
example include file
*/

void myPrintHelloMake(void);
\end{lstlisting}
&
\end{tabular}

\end{slide}


\begin{slide}{General Structure of a Makefile}

The general command of a \cc{Makefile} is 

\boite{Target : Dependency\\ \lbrack Tabulation\rbrack Command} \medskip

The \cc{Target} is a {\em label} that you choose to define a target, i.e., something that you want to obtain. In our example, the executable \cc{main} would be our target. \medskip

The \cc{Dependency} is anything that is required to ``reach'' the target. In our example, the executable \cc{main} depends on some object files: these are our dependencies. \medskip

The target is then produced by executing the \cc{Command}. In our example, the command would be the C compiler on our system. \medskip

Dependencies are often also targets. Again, in our example, the object file \cc{hello.o} needed to produce the target \cc{main} can be thought of a target, which will be produced by the C compiler.

\end{slide}


\begin{slide}{Simple Makefile}

Below is the simplest Makefile we could make out of our example:

\begin{lstlisting}
hello: hello.o main.o
	gcc -o hello hello.o main.o

hello.o: hello.c
	gcc -o hello.o -c hello.c -W -Wall -ansi -pedantic

main.o: main.c hello.h
	gcc -o main.o -c main.c -W -Wall -ansi -pedantic
\end{lstlisting}

If you save this code into a file \cc{Makefile}, you can compile the project by simply typing 

\boite{make}\medskip

This will produce the executable \cc{hello} (the primary target). This primary target has two dependencies, \cc{main.o} and \cc{hello.o}. Each of them is the target of another command, with \cc{main.o} depending on \cc{hello.h}.

\end{slide}


\begin{slide}{More Advanced Makefile}

We add three more rules to: (i) list of the executables to produce (\cc{all}), (ii) delete intermediate object files (\cc{clean}), and (iii) remove the executable to force a complete rebuild (\cc{mrproper}).

\begin{lstlisting}
all: hello

hello: hello.o main.o
	gcc -o hello hello.o main.o

hello.o: hello.c
	gcc -o hello.o -c hello.c -W -Wall -ansi -pedantic

main.o: main.c hello.h
	gcc -o main.o -c main.c -W -Wall -ansi -pedantic

clean:
	rm -rf *.o

mrproper: clean
	rm -rf hello
\end{lstlisting}

\end{slide}


\begin{slide}{User Variables}

Instead of repeating the same command for compilation, we define our own variables.

\begin{lstlisting}
CC      = gcc
CFLAGS  = -W -Wall -ansi -pedantic
LDFLAGS =
EXEC    = hello

all: $(EXEC)

hello: hello.o main.o
	$(CC) -o hello hello.o main.o $(LDFLAGS)

hello.o: hello.c
	$(CC) -o hello.o -c hello.c $(CFLAGS)

main.o: main.c hello.h
	$(CC) -o main.o -c main.c $(CFLAGS)

clean:
	rm -rf *.o

mrproper: clean
	rm -rf $(EXEC)
\end{lstlisting}

\end{slide}


\begin{slide}{Combining Internal and User Variables}

There are pre-defined variables that come handy:
\begin{tabular}{ll}
\$@  & Name of the target \\
\$\textless & The name of the first dependency \\
\$\^ & The list of all dependencies \\
\$?  & The list of all dependencies that are more recent than the target \\
\$*  & The name of the file without the extension
\end{tabular}

\lstset{
    language=C,
    basicstyle=\tiny,
    frame=single
}
\begin{lstlisting}
CC      = gcc
CFLAGS  = -W -Wall -ansi -pedantic
LDFLAGS =
EXEC    = hello

all: $(EXEC)

hello: hello.o main.o
	$(CC) -o $@ $^ $(LDFLAGS)

hello.o: hello.c
	$(CC) -o $@ -c $< $(CFLAGS)

main.o: main.c hello.h
	$(CC) -o $@ -c $< $(CFLAGS)

clean:
	rm -rf *.o

mrproper: clean
	rm -rf $(EXEC)
\end{lstlisting}

\end{slide}


\begin{slide}{Inference Rule}

We now define generic rules to build all object files (extension .o) from source files (extension .c in our C programs). This is done with the inference rule

\boite{\%.o : \%.c}\medskip

We also add the requirement that object files depend on the header files (only one in our case, but they could be more).

\lstset{
    language=C,
    basicstyle=\tiny,
    frame=single
}
\begin{lstlisting}
CC      = gcc
CFLAGS  = -W -Wall -ansi -pedantic
LDFLAGS =
EXEC    = hello
DEPS    = hello.h

all: $(EXEC)

hello: hello.o main.o
	$(CC) -o $@ $^ $(LDFLAGS)

%.o: %.c $(DEPS)
	$(CC) -o $@ -c $< $(CFLAGS)

clean:
	rm -rf *.o

mrproper: clean
	rm -rf $(EXEC)
\end{lstlisting}

\end{slide}


\begin{slide}{Automatic Generation of Object Files}

Finally, we define the list of all sources files that need be compiled, put it in a variable \cc{SRC}, and automatically define the corresponding list of object files through

\boite{OBJ= \$(SRC:.c=.o)}\medskip

The final Makefile looks like

\lstset{
    language=C,
    basicstyle=\tiny,
    frame=single
}
\begin{lstlisting}
CC      = gcc
CFLAGS  = -W -Wall -ansi -pedantic
LDFLAGS =
EXEC    = hello
DEPS    = hello.h
SRC     = hello.c main.c
OBJ     = $(SRC:.c=.o)

all: $(EXEC)

hello: $(OBJ)
	$(CC) -o $@ $^ $(LDFLAGS)

%.o: %.c $(DEPS)
	$(CC) -o $@ -c $< $(CFLAGS)

clean:
	rm -rf *.o

mrproper: clean
	rm -rf $(EXEC)
\end{lstlisting}

\end{slide}

%%%%%%%%%%%%%%%%%%%%%%%%%%%%%%%%%%%%%%%%%%%%%%%%%%%%%%%%%%%%%%%%%%%%%%%%%%%%%%%
%%%%%%%%%%%%%%%%%%%%%%%%%%%%%%%%%%%%%%%%%%%%%%%%%%%%%%%%%%%%%%%%%%%%%%%%%%%%%%%
%%%%%%%%%%%%%%%%%%%%%%%%%%%%%%%%%%%%%%%%%%%%%%%%%%%%%%%%%%%%%%%%%%%%%%%%%%%%%%%
%%%%%%%%%%%%%%%%%%%%%%%%%%%%%%%%%%%%%%%%%%%%%%%%%%%%%%%%%%%%%%%%%%%%%%%%%%%%%%%


\section{The Git Version Control System}


\begin{slide}{Introducing Version Control System}

Git is an open source version control software (VCS). \medskip

Advantages of VCS:
\begin{itemize}
\item Keep a record of the entire development history of a project (can be a code or anything else, for example the versions of your Ph.D. thesis...)
\item Each snapshot of the project (list and content of all files and directories) is recorded with a mandatory comment describing it
\item Considerably improves the maintenance of projects involving several people
\end{itemize}

Disadvantages: None... (OK, there is a learning curve)

\end{slide}


\begin{slide}{Installing Git}

Start with downloading the software:
\begin{itemize}
\item Linux: use your distribution package manager to install it
\item Windows: go to http://git-scm.com/downloads and download the client
\item MacOs: it is a good idea to install it via Homebrew or Macports. Ask if you are not familiar with these softwares
\end{itemize}

You may want to use a GUI client. One possible choice is git-cola, available at https://git-cola.github.io/. Its main advantage is that it is multi-platform, i.e., it will work on Linux, Windows and MacOS equally well.\medskip

In a terminal window, you can access the documentation with

\boite{git --help}\medskip

and the documentation on a specific topic with, e.g., 

\boite{git branch --help}

\end{slide}


\begin{slide}{First Git Project (1/2)}

In the following, we will illustrate most basic features of git using the command line. We assume that you want to create a folder \cc{dir/} that will contain a bunch of files, and you want the entire \cc{dir/} folder to be under git control. \medskip

In the folder that contains dir/, type: 

\boite{git init dir}\medskip

While still in the directory \cc{dir/} create the file \cc{hello.f90} and write the Fortran 90 code needed to display the usual {\em Hello World!} message. The content of the directory has been changed, since you have added a new file. \medskip

Take a snapshot of the directory with

\boite{git add .}\medskip

and record this snapshot with

\boite{git commit} 

\end{slide}


\begin{slide}{First Git Project (2/2)}

Note that when you commit your project, you have to type in a commit message. Be serious about it: in complex projects, commit messages are invaluable sources of information! \medskip

Under Linux, it is very likely that the commit message has to be entered with the \cc{vi} editor: first press \cc{i} to enter the edit mode, then type your comments, then type \cc{ZZ} to record and close. \medskip

The initial commit creates a new {\em branch} in the git repository, named {\em master} (all git repositories have a master branch). \medskip

If you add a file to the directory, or if you remove a file, or if you modify an existing file (by editing it for instance, or by changing its attribute), you are modifying the active branch -- in our simple example, \cc{master}.

\end{slide}


\begin{slide}{Useful Git Commands}

To see if there is anything that has been modified since your last commit

\boite{git status}\medskip

To see the history of the versions of your project

\boite{git log}\medskip

To list all branches of your project

\boite{git branch}\medskip

The asterisk lists the branch that is currently active, i.e., the branch you are ``sitting on'' (think ot the branches of the tree, really).\medskip

Commits are referenced with some crazy labels. You can view the difference between the current branch and a former commit with commands such as 

\boite{git diff 655987702ea15b9daa12dc8bc37023b35fc4eb50}

\end{slide}


\begin{slide}{Branches}

So far, the \cc{master} branch of our git project is made of a directory \cc{dir} with a file called \cc{hello.f90}. \medskip

Create a new branch with

\boite{git branch C++}\medskip

List the branches...

\boite{git branch}\medskip

Checkout the new branch, i.e., if your git project is a tree, you are now moving to a new branch, the one named \cc{C++}

\boite{git checkout C++}\medskip

Now, delete the file \cc{hello.f90}, and create a new one called \cc{hello.cc} with the code to display {\em Hello, World!} in C++.\medskip 

After checking the status of the git repository, add and commit the new file; check the history of the repository. What happens if you now checkout the \cc{master} branch?

\end{slide}


\begin{slide}{To go Further on Git}

This short introduction should allow you to get started with Git. The topics that have not been covered:
\begin{itemize}
\item How to import, export and backup a Git repository on some online servers such as Bitbicket or Github, where you can create accounts for free.
\item How to merge branches and resolve conflicts: this is especially important if you use Git as part of a team.
\end{itemize}

Documentation:
\begin{itemize}
\item http://git-scm.com/docs/gittutorial: This is a good starting point if you are totally unfamiliar with Git and version control system. In fact, I used the material in these slides as inspiration.
\item http://nvie.com/posts/a-successful-git-branching-model/: For the more advanced users, this blog entry discusses a possible Git strategy for a project development.
\end{itemize}

\end{slide}


%%%%%%%%%%%%%%%%%%%%%%%%%%%%%%%%%%%%%%%%%%%%%%%%%%%%%%%%%%%%%%%%%%%%%%%%%%%%%%%
%%%%%%%%%%%%%%%%%%%%%%%%%%%%%%%%%%%%%%%%%%%%%%%%%%%%%%%%%%%%%%%%%%%%%%%%%%%%%%%
%%%%%%%%%%%%%%%%%%%%%%%%%%%%%%%%%%%%%%%%%%%%%%%%%%%%%%%%%%%%%%%%%%%%%%%%%%%%%%%
%%%%%%%%%%%%%%%%%%%%%%%%%%%%%%%%%%%%%%%%%%%%%%%%%%%%%%%%%%%%%%%%%%%%%%%%%%%%%%%


\section{Workplan}


\begin{slide}{Today}

The goal of the first session is to write a short program (in your favorite language) that achieves the following
\begin{itemize}
\item Read as input an integer $N$ giving the dimension of a matrix
\item Set up a matrix of real numbers and dimension $N\times N$
\item Diagonalize the matrix using a call to a library
\item Display the first 10 eigenvalues
\end{itemize}

If you think this is so easy that it makes you ashamed of even contemplating doing this, we suggest you start developing a module (in your favorite language) implementing Gauss-Laguerre quadrature. In particular, you could try to very numerically the orthonormality of HO radial wave functions.

\end{slide}


\end{document}
