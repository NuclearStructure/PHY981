\documentclass[prc]{revtex4}
\usepackage[dvips]{graphicx}
\usepackage{mathrsfs}
\usepackage{amsfonts}
\usepackage{lscape}

\usepackage{epic,eepic}
\usepackage{amsmath}
\usepackage{amssymb}
\usepackage[dvips]{epsfig}
\usepackage[T1]{fontenc}
\usepackage{hyperref}
\usepackage{bezier}
\usepackage{pstricks}
\usepackage{dcolumn}% Align table columns on decimal point
\usepackage{bm}% bold math
%\usepackage{braket}
\usepackage[dvips]{graphicx}
\usepackage{pst-plot}

\newcommand{\One}{\hat{\mathbf{1}}}
\newcommand{\eff}{\text{eff}}
\newcommand{\Heff}{\hat{H}_\text{eff}}
\newcommand{\Veff}{\hat{V}_\text{eff}}
\newcommand{\braket}[1]{\langle#1\rangle}
\newcommand{\Span}{\operatorname{sp}}
\newcommand{\tr}{\operatorname{trace}}
\newcommand{\diag}{\operatorname{diag}}
\newcommand{\bra}[1]{\left\langle #1 \right|}
\newcommand{\ket}[1]{\left| #1 \right\rangle}
\newcommand{\element}[3]
    {\bra{#1}#2\ket{#3}}

\newcommand{\normord}[1]{
    \left\{#1\right\}
}

\usepackage{amsmath}
\begin{document}

\title{Exercises PHY981 Spring 2015}
%\author{}
\maketitle
The exercises are available at the beginning of the 
week and are  to be handed in 
the week thereafter on {\bf Wednesdays at noon}. This can be done electronically by sending your github link or by sending a scan of your notes, a pdf or postscript file, or ipython notebook or any format you prefer,  by email to 
hjensen@nscl.msu.edu. 



\subsection*{Exercise 5}
This exercise serves to convince you about the relation between
two different single-particle bases, where one could be our new Hartree-Fock basis and the other a harmonic oscillator basis.
\begin{enumerate}
\item[a)]
Consider a Slater determinant built up of single-particle orbitals $\psi_{\lambda}$, 
with $\lambda = 1,2,\dots,A$.

The unitary transformation
\[
\psi_a  = \sum_{\lambda} C_{a\lambda}\phi_{\lambda},
\]
brings us into the new basis.  
The new basis has quantum numbers $a=1,2,\dots,A$.
Show that the new basis is orthonormal.
Show that the new Slater determinant constructed from the new single-particle wave functions can be
written as the determinant based on the previous basis and the determinant of the matrix $C$.
Show that the old and the new Slater determinants are equal up to a complex constant with absolute value unity.
(Hint, $C$ is a unitary matrix). 
\item[b)]  Starting with the second quantization representation of the Slater determinant 
\[
\Phi_{0}=\prod_{i=1}^{n}a_{\alpha_{i}}^{\dagger}\ket{0},
\]
use Wick's theorem to compute the normalization integral
$<\Phi_{0}|\Phi_{0}>$.

\end{enumerate}

\subsection*{Exercise 6}

Calculate the matrix elements
\[
\bra{\alpha_{1}\alpha_{2}}\hat{F}\ket{\alpha_{1}\alpha_{2}}
\]
and
\[
\bra{\alpha_{1}\alpha_{2}}\hat{G}\ket{\alpha_{1}\alpha_{2}}
\]
with
\[
\ket{\alpha_{1}\alpha_{2}}=a_{\alpha_{1}}^{\dagger}
a_{\alpha_{2}}^{\dagger}\ket{0} ,
\]
\[
\hat{F}=\sum_{\alpha\beta}\bra{\alpha}f\ket{\beta}
a_{\alpha}^{\dagger}a_{\beta}  ,
\]
\[
\bra{\alpha}f\ket{\beta}=\int \psi_{\alpha}^{*}(x)f(x)\psi_{\beta}(x)dx ,
\]
\[
\hat{G} = \frac{1}{2}\sum_{\alpha\beta\gamma\delta}
\bra{\alpha\beta}g\ket{\gamma\delta}
a_{\alpha}^{\dagger}a_{\beta}^{\dagger}a_{\delta}a_{\gamma} ,
\]
and

\[
\bra{\alpha\beta}g\ket{\gamma\delta}=
\int\int \psi_{\alpha}^{*}(x_{1})\psi_{\beta}^{*}(x_{2})g(x_{1},
x_{2})\psi_{\gamma}(x_{1})\psi_{\delta}(x_{2})dx_{1}dx_{2}
\]

Compare these results with those from exercise 3c).


\subsection*{Exercise 7}

Show that the onebody part of the Hamiltonian
    \begin{equation*}
        \hat{H}_0 = \sum_{pq} \element{p}{\hat{h}_0}{q} a^\dagger_p a_q
    \end{equation*}
can be written, using standard annihilation and creation operators, in normal-ordered form as 
    \begin{align*}
        \hat{H}_0 &= \sum_{pq} \element{p}{\hat{h}_0}{q} a^\dagger_p a_q \nonumber \\
            &= \sum_{pq} \element{p}{\hat{h}_0}{q} \left\{a^\dagger_p a_q\right\} + 
                \delta_{pq\in i} \sum_{pq} \element{p}{\hat{h}_0}{q} \nonumber \\
            &= \sum_{pq} \element{p}{\hat{h}_0}{q} \left\{a^\dagger_p a_q\right\} +
                \sum_i \element{i}{\hat{h}_0}{i}
    \end{align*}
Explain the meaning of the various symbols. Which reference 
vacuum has been used?

\subsection*{Exercise 8}
Show that the twobody part of the Hamiltonian
    \begin{equation*}
        \hat{H}_I = \frac{1}{4} \sum_{pqrs} \element{pq}{\hat{v}}{rs} a^\dagger_p a^\dagger_q a_s  a_r
    \end{equation*}
can be written, using standard annihilation and creation operators, in normal-ordered form as 
    \begin{align*}
    \hat{H}_I &= \frac{1}{4} \sum_{pqrs} \element{pq}{\hat{v}}{rs} a^\dagger_p a^\dagger_q a_s  a_r \nonumber \\
        &= \frac{1}{4} \sum_{pqrs} \element{pq}{\hat{v}}{rs} \normord{a^\dagger_p a^\dagger_q a_s  a_r}
            + \sum_{pqi} \element{pi}{\hat{v}}{qi} \normord{a^\dagger_p a_q} 
            + \frac{1}{2} \sum_{ij} \element{ij}{\hat{v}}{ij}
    \end{align*}
Explain again the meaning of the various symbols.

This exercise is optional: Derive the normal-ordered form of the threebody part of the Hamiltonian.
    \begin{align*}
    \hat{H}_3 &= \frac{1}{36} \sum_{\substack{
                        pqr \\
                        stu}}
                 \element{pqr}{\hat{v}_3}{stu} a^\dagger_p a^\dagger_q a^\dagger_r a_u a_t a_s\\
    \end{align*}
and specify the contributions to the twobody, onebody and the scalar part.

\end{document}
