\documentclass[prc]{revtex4}
\usepackage[dvips]{graphicx}
\usepackage{mathrsfs}
\usepackage{amsfonts}
\usepackage{lscape}

\usepackage{epic,eepic}
\usepackage{amsmath}
\usepackage{amssymb}
\usepackage[dvips]{epsfig}
\usepackage[T1]{fontenc}
\usepackage{hyperref}
\usepackage{bezier}
\usepackage{pstricks}
\usepackage{dcolumn}% Align table columns on decimal point
\usepackage{bm}% bold math
%\usepackage{braket}
\usepackage[dvips]{graphicx}
\usepackage{pst-plot}

\newcommand{\One}{\hat{\mathbf{1}}}
\newcommand{\eff}{\text{eff}}
\newcommand{\Heff}{\hat{H}_\text{eff}}
\newcommand{\Veff}{\hat{V}_\text{eff}}
\newcommand{\braket}[1]{\langle#1\rangle}
\newcommand{\Span}{\operatorname{sp}}
\newcommand{\tr}{\operatorname{trace}}
\newcommand{\diag}{\operatorname{diag}}
\newcommand{\bra}[1]{\left\langle #1 \right|}
\newcommand{\ket}[1]{\left| #1 \right\rangle}
\newcommand{\element}[3]
    {\bra{#1}#2\ket{#3}}

\newcommand{\normord}[1]{
    \left\{#1\right\}
}

\usepackage{amsmath}
\begin{document}

\title{Exercises PHY981 Spring 2015}
%\author{}
\maketitle
The exercises are available at the beginning of the 
week and are  to be handed in 
the week thereafter on {\bf Wednesdays at noon}. This can be done electronically by sending your github link or by sending a scan of your notes, a pdf or postscript file, or ipython notebook or any format you prefer,  by email to 
hjensen@nscl.msu.edu. 

\subsection*{Exercise 2: This is a numerical exercise and is optional}


The program for finding the eigenvalues of the harmonic oscillator are in the github folder
\href{{https://github.com/NuclearStructure/PHY981/tree/master/doc/pub/spdata/programs}}{\nolinkurl{https://github.com/NuclearStructure/PHY981/tree/master/doc/pub/spdata/programs}}.

You can use this program to solve the exercises below, or write your own using your preferred programming language, be it python, fortran or c++ or other languages. 
\begin{enumerate}
\item[a)] Compute the eigenvalues of the five lowest states with a given orbital momentum and oscillator frequency $\omega$. Study these results as functions of the the maximum value of $r$ and the number of integration points $n$, starting with  $r_{\mathrm{max}}=10$. Compare the computed ones with the exact values and comment your results.

\item[b)]  Plot thereafter the eigenfunctions as functions of $r$ for the lowest-lying state with a given orbital momentum $l$.

\item[c)]  Replace thereafter the harmonic oscillator potential with a Woods-Saxon potential using the parameters discussed above. Compute the lowest five eigenvalues and plot the eigenfunction of the lowest-lying state. How does this compare with the harmonic oscillator? Comment your results and possible implications for nuclear physics studies.
\end{enumerate}
\subsection*{Exercise 3}
Consider the Slater determinant
\[
\Phi_{\lambda}^{AS}(x_{1}x_{2}\dots x_{N};\alpha_{1}\alpha_{2}\dots\alpha_{N})
=\frac{1}{\sqrt{N!}}\sum_{p}(-)^{p}P\prod_{i=1}^{N}\psi_{\alpha_{i}}(x_{i}).
\]
where $P$ is an operator which permutes the coordinates of two particles. We have assumed here that the 
number of particles is the same as the number of available single-particle states, represented by the
greek letters $\alpha_{1}\alpha_{2}\dots\alpha_{N}$.
\begin{enumerate}
\item[a)] Write  out $\Phi^{AS}$ for $N=3$.  
\item[b)] Show that
\[
\int dx_{1}dx_{2}\dots dx_{N}\left\vert
\Phi_{\lambda}^{AS}(x_{1}x_{2}\dots x_{N};\alpha_{1}\alpha_{2}\dots\alpha_{N})
\right\vert^{2} = 1.
\]
\item[c)] Define a general onebody operator $\hat{F} = \sum_{i}^N\hat{f}(x_{i})$ and a general 
twobody operator $\hat{G}=\sum_{i>j}^N\hat{g}(x_{i},x_{j})$
with $g$ being invariant under the interchange of the coordinates of particles $i$ and $j$.
Calculate the matrix elements for a two-particle Slater determinant
\[
\bra{\Phi_{\alpha_{1}\alpha_{2}}^{AS}}\hat{F}\ket{\Phi_{\alpha_{1}\alpha_{2}}^{AS}},
\]
and
\[
\bra{\Phi_{\alpha_{1}\alpha_{2}}^{AS}}\hat{G}\ket{\Phi_{\alpha_{1}\alpha_{2}}^{AS}}.
\]
Explain the short-hand notation for the Slater determinant.
Which properties do you expect these operators to have in addition to an eventual permutation
symmetry?
\end{enumerate}

\subsection*{Exercise 4}
We will now consider a simple three-level problem, depicted in the figure below. This is our first and very simple model of a possible many-nucleon (or just fermion) problem and the shell-model.
The single-particle states are labelled by the quantum number $p$ and can accomodate up to two single particles, 
viz., every single-particle state 
is doubly degenerate (you could think of this as one state having spin up and the other spin down). 
We let the spacing between the doubly degenerate single-particle states be constant, with value $d$.  The first state
has energy $d$. There are only three available single-particle states, $p=1$, $p=2$ and $p=3$, as illustrated
in the figure. 
\begin{enumerate}
\item[a)] How many two-particle Slater determinants can we construct in this space? 
\item[b)] We limit ourselves to a system with only the two lowest single-particle orbits and two particles, $p=1$ and $p=2$.
We assume that we can write the Hamiltonian as
\[
       \hat{H}=\hat{H}_0+\hat{H}_I,
\]
and that the onebody part of the Hamiltonian with single-particle operator $\hat{h}_0$ has the property
\[
\hat{h}_0\psi_{p\sigma} = p\times d \psi_{p\sigma},
\]
where we have added a spin quantum number $\sigma$. 
We assume also that the only two-particle states that can exist are those where two particles are in the 
same state $p$, as shown by the two possibilities to the left in the figure.
The two-particle matrix elements of $\hat{H}_I$ have all a constant value, $-g$.
\begin{figure}
\vspace{1.0cm}
 \setlength{\unitlength}{1cm}
 \begin{picture}(15,5)
 \thicklines
\put(-0.6,1){\makebox(0,0){$p=1$}}
\put(-0.6,2){\makebox(0,0){$p=2$}}
\put(-0.6,3){\makebox(0,0){$p=3$}}
% first 2-particle state
\put(0.8,1){\circle*{0.3}}
\put(1.7,1){\circle*{0.3}}
% second 2-particle state
\put(5.0,2){\circle*{0.3}}
\put(5.9,2){\circle*{0.3}}
% third 2-particle state
\put(9.2,1){\circle*{0.3}}
\put(10.1,3){\circle*{0.3}}
% third 2-particle state
\put(13.4,2){\circle*{0.3}}
\put(14.3,3){\circle*{0.3}}
\dashline[+1]{2.5}(0,1)(15,1)
\dashline[+1]{2.5}(0,2)(15,2)
\dashline[+1]{2.5}(0,3)(15,3)
 \end{picture}
\caption{Schematic plot of the possible single-particle levels with double degeneracy.
The filled circles indicate occupied particle states.
The spacing between each level $p$ is constant in this picture. We show some possible two-particle states.}
\end{figure}
Show then that the Hamiltonian matrix can be written as 
\[
\left(\begin{array}{cc}2d-g &-g \\
-g &4d-g \end{array}\right),
\]
and find the eigenvalues and eigenvectors.  What is mixing of the state with two particles in $p=2$ 
to the wave function with two-particles in $p=1$? Discuss your results in terms of a linear combination
of Slater determinants.  \\
\item[c)] Add the possibility that the two particles can be in the state with $p=3$ as well and find the Hamiltonian
matrix, the eigenvalues and the eigenvectors. We still insist that we only have two-particle states composed of two particles being in the same
level $p$. You can diagonalize numerically your $3\times 3$ matrix.\newline\newline
This simple model catches several birds with a stone. It demonstrates how we can build linear combinations
of Slater determinants and interpret these as different admixtures to a given state. It represents also the way we are going to interpret these contributions.  The two-particle states above $p=1$ will be interpreted as 
excitations from the ground state configuration, $p=1$ here.  The reliability of this ansatz for the ground state, 
with two particles in $p=1$,
depends on the strength of the interaction $g$ and the single-particle spacing $d$.
Finally, this model is a simple schematic ansatz for studies of pairing correlations and thereby superfluidity/superconductivity  
in fermionic systems. 
\end{enumerate}

\end{document}


