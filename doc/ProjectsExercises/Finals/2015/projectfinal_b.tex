\documentclass[prc]{revtex4}
\usepackage[dvips]{graphicx}
\usepackage{mathrsfs}
\usepackage{amsfonts}
\usepackage{lscape}

\usepackage{epic,eepic}
\usepackage{amsmath}
\usepackage{amssymb}
\usepackage[dvips]{epsfig}
\usepackage[T1]{fontenc}
\usepackage{hyperref}
\usepackage{bezier}
\usepackage{pstricks}
\usepackage{dcolumn}% Align table columns on decimal point
\usepackage{bm}% bold math
%\usepackage{braket}
\usepackage[dvips]{graphicx}
\usepackage{pst-plot}

\newcommand{\One}{\hat{\mathbf{1}}}
\newcommand{\eff}{\text{eff}}
\newcommand{\Heff}{\hat{H}_\text{eff}}
\newcommand{\Veff}{\hat{V}_\text{eff}}
\newcommand{\braket}[1]{\langle#1\rangle}
\newcommand{\Span}{\operatorname{sp}}
\newcommand{\tr}{\operatorname{trace}}
\newcommand{\diag}{\operatorname{diag}}
\newcommand{\bra}[1]{\left\langle #1 \right|}
\newcommand{\ket}[1]{\left| #1 \right\rangle}
\newcommand{\element}[3]
    {\bra{#1}#2\ket{#3}}

\newcommand{\normord}[1]{
    \left\{#1\right\}
}

\usepackage{amsmath}
\begin{document}


\title{Project for the final oral examination of PHY981 }
%\author{}
\maketitle
\section*{Studies $\beta$ decay of $^{16}$N}

The aim of this project is to perform shell-model calculations of the nuclei $^{16}$O and $^{16}$N, studying the low-lying excited states 
and perfoming $\beta$-decay studies using Alex Brown's shell-model program NushellX. 

The explicit task is thus to
\begin{enumerate}
\item Give a survey of $\beta$-decay experiments performed in this mass region. Motivate the importance of studies of $\beta$-decay in this mass region.
\item Perform shell-model studies of low-lying states of the above nuclei using both the $0p$ and the $1s0d$ shells. You may need to perform truncations
of the basis by leaving out specific single-particle states (for example the $0d_{3/2}$ state. 
\item Go through the details in chapter 7.4 of Suhonen and convince yourself about the correctness of equations (7.86)-(7.102). 
Present also other theoretical studies.
\item Calculate thereafter, based on your shell-model states, the relevant $\beta$-decays.
Give a critical analysis of your results and compare with existing theoretical studies and experiments.
\end{enumerate}


\end{document}


