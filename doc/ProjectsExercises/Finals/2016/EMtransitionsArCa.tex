\documentclass[prc]{revtex4}
\usepackage[dvips]{graphicx}
\usepackage{mathrsfs}
\usepackage{amsfonts}
\usepackage{lscape}

\usepackage{epic,eepic}
\usepackage{amsmath}
\usepackage{amssymb}
\usepackage[dvips]{epsfig}
\usepackage[T1]{fontenc}
\usepackage{hyperref}
\usepackage{bezier}
\usepackage{pstricks}
\usepackage{dcolumn}% Align table columns on decimal point
\usepackage{bm}% bold math
%\usepackage{braket}
\usepackage[dvips]{graphicx}
\usepackage{pst-plot}

\newcommand{\One}{\hat{\mathbf{1}}}
\newcommand{\eff}{\text{eff}}
\newcommand{\Heff}{\hat{H}_\text{eff}}
\newcommand{\Veff}{\hat{V}_\text{eff}}
\newcommand{\braket}[1]{\langle#1\rangle}
\newcommand{\Span}{\operatorname{sp}}
\newcommand{\tr}{\operatorname{trace}}
\newcommand{\diag}{\operatorname{diag}}
\newcommand{\bra}[1]{\left\langle #1 \right|}
\newcommand{\ket}[1]{\left| #1 \right\rangle}
\newcommand{\element}[3]
    {\bra{#1}#2\ket{#3}}

\newcommand{\normord}[1]{
    \left\{#1\right\}
}



\usepackage{amsmath}
\begin{document}


\title{Project for the final oral examination PHY981}
%\author{}
\maketitle
\section*{Project G, Studies of electromagnetic transitions in Ar isotopes and $^{48}$Ca}

The aim of this project is to study the structure of selected low-lying states of the Ar isotopes and selected 
$B(E2; 2^+_1\rightarrow 0^+_1)$  transitions using Alex Brown's Nushellx code. 
There is considerable interest in neutron-rich nuclei between $N = 14$ and $N = 28$,
motivated particularly by the development of collectivity in nuclei such as
$^{40,42}$S and by questions
concerning the strength of the N = 28 shell gap far from stability (see Refs.~[1-5] and references
therein). The Ar isotopes from $^{34}$Ar ($N = 16$) to $^{46}$Ar ($N = 28$) are important for studying this
evolution of collectivity and for testing shell model calculations for neutron-rich nuclei in the
region. Because they have two proton holes in the $sd$ shell, the neutron-rich Ar isotopes offer a
relatively ‘simple’ case to probe proton-neutron interactions and build up a
microscopic description of nuclear structure in neutron rich nuclei near $N = 28$. Being close to $^{48}$Ca, there 
is great interest in understanding the role of such interactions.   
The lightest Ar isotopes can be studied with effective interactions for $sd$-shell only as a first trial. 
For the heavier ones, one would need to test 
effective interactions that include the $0f_{7/2}$ orbit as well and possibly the $1p_{3/2}$ single-particle state. 
In Refs.~[2,3], shell-model calculations where performed
with effective interactions that included parts of the $pf$-shell as well. 

The task here is to compute low-lying states of the Ar isotopes from
$N=16$ to $N=28$ and compute $B(E2; 2^+_1\rightarrow 0^+_1)$
transition strengths and compare with available data. To achieve this
you will need to use an effective interaction designed for the $1s0d$
and $1s0d1p0f$ shells. Since a full calculation in two major shells
may be beyond reach, you will need to truncate the number of particles
which can leave/occupy selected single-particle states. In the file
which contains the single-particle data, you should block some of the
single-particle orbis in the $1p0f$ shells.
\begin{table}
\caption{Experimental excitation energies, B(E2) values and lifetimes for the Ar isotopes. Data
are from [1,6-11].}
\begin{tabular}{cccc} \hline\\
$A$ & $N$ & $E(2^+)$ [keV] & B(E2) [e2fm4] \\ \hline
38 &20 &2167& 130(10) \\
40 &22 &1461 &330(40) \\
42 &24 &1208 &430(100)\\
44 &26 &1158 &345(41)\\
46 &28 &1577 &196(39); 218(31)\\ \hline
\end{tabular}
\end{table}

The possible tasks are as follows
\begin{enumerate}
\item Compute the spectra  of the low-lying spectrum of the above even Ar isotopes using the $1s0d$ shell as degree of freedom first. Add thereafter the $0f_{7/2}$ single-particle orbit and study the behavior of your results. Compare with Refs.~[2,3]. 
\item Add thereafter the $1p_{3/2}$ single-particle state and commnet your results.
\item Compute thereafter the $B(E2; 2^+_1\rightarrow 0^+_1)$
transition strengths and compare with available data using the above model spaces. 
\item To study the similar transition in $^{48}$Ca, you will need to use the $1p0f$ shell. Compute the above transition strength in $^{48}$Ca and discuss the differences between $^{48}$Ca and $^{46}$Ar where two protons have been removed from the 
$1s0d$ shell.
\end{enumerate}



Here follows a list of possible references.

\begin{enumerate}
\item  H. Scheit et al., Phys Rev Lett 77, 3967 (1996) 
\item  J. Retamosa et al., Phys Rev C 55, 1266 (1997) 
\item  S. Nummela et al., Phys Rev C 63, 044316 (2001) 
\item A.D. Davies et al., Phys Rev Lett 96, 112503 (2006)
\item  A.E. Stuchbery et al., Phys Rev C 74, 054307 (2006)
\item  S. Raman et al., At Data Nucl Data Tables 78, 1 (2001)
\item  B. Fornal et al., Eur Phys J A 7, 147 (2000)
\item  A. Gade et al., Phys Rev C 68, 014302 (2003)
\item K.-H. Speidel et al., Phys Lett B632, 207 (2006)
\item J. Cub et al., Nucl Phys A 549, 304 (1992)
\item E. A. Stefanova et al., Phys Rev C 72, 014309 
\end{enumerate}
\end{document}










