%%
%% Automatically generated file from DocOnce source
%% (https://github.com/hplgit/doconce/)
%%
%%


%-------------------- begin preamble ----------------------

\documentclass[%
oneside,                 % oneside: electronic viewing, twoside: printing
final,                   % draft: marks overfull hboxes, figures with paths
10pt]{article}

\listfiles               % print all files needed to compile this document

\usepackage{relsize,makeidx,color,setspace,amsmath,amsfonts,amssymb}
\usepackage[table]{xcolor}
\usepackage{bm,ltablex,microtype}

\usepackage[pdftex]{graphicx}

\usepackage[T1]{fontenc}
%\usepackage[latin1]{inputenc}
\usepackage{ucs}
\usepackage[utf8x]{inputenc}

\usepackage{lmodern}         % Latin Modern fonts derived from Computer Modern

% Hyperlinks in PDF:
\definecolor{linkcolor}{rgb}{0,0,0.4}
\usepackage{hyperref}
\hypersetup{
    breaklinks=true,
    colorlinks=true,
    linkcolor=linkcolor,
    urlcolor=linkcolor,
    citecolor=black,
    filecolor=black,
    %filecolor=blue,
    pdfmenubar=true,
    pdftoolbar=true,
    bookmarksdepth=3   % Uncomment (and tweak) for PDF bookmarks with more levels than the TOC
    }
%\hyperbaseurl{}   % hyperlinks are relative to this root

\setcounter{tocdepth}{2}  % number chapter, section, subsection

% --- fancyhdr package for fancy headers ---
\usepackage{fancyhdr}
\fancyhf{} % sets both header and footer to nothing
\renewcommand{\headrulewidth}{0pt}
\fancyfoot[LE,RO]{\thepage}
\fancyfoot[C]{{\footnotesize \copyright\ 2013-2016, Morten Hjorth-Jensen. Released under CC Attribution-NonCommercial 4.0 license}}

% Ensure copyright on titlepage (article style) and chapter pages (book style)
\fancypagestyle{plain}{
  \fancyhf{}
  \fancyfoot[C]{{\footnotesize \copyright\ 2013-2016, Morten Hjorth-Jensen. Released under CC Attribution-NonCommercial 4.0 license}}
%  \renewcommand{\footrulewidth}{0mm}
  \renewcommand{\headrulewidth}{0mm}
}
% Ensure copyright on titlepages with \thispagestyle{empty}
\fancypagestyle{empty}{
  \fancyhf{}
  \fancyfoot[C]{{\footnotesize \copyright\ 2013-2016, Morten Hjorth-Jensen. Released under CC Attribution-NonCommercial 4.0 license}}
  \renewcommand{\footrulewidth}{0mm}
  \renewcommand{\headrulewidth}{0mm}
}

\pagestyle{fancy}


\usepackage[framemethod=TikZ]{mdframed}

% --- begin definitions of admonition environments ---

% --- end of definitions of admonition environments ---

% prevent orhpans and widows
\clubpenalty = 10000
\widowpenalty = 10000

% --- end of standard preamble for documents ---


% insert custom LaTeX commands...

\raggedbottom
\makeindex
\usepackage[totoc]{idxlayout}   % for index in the toc
\usepackage[nottoc]{tocbibind}  % for references/bibliography in the toc

%-------------------- end preamble ----------------------

\begin{document}

% matching end for #ifdef PREAMBLE


% ------------------- main content ----------------------

% Slides for PHY981 Nuclear Structure


% ----------------- title -------------------------

\thispagestyle{empty}

\begin{center}
{\LARGE\bf
\begin{spacing}{1.25}
PHY 981 Nuclear Structure
\end{spacing}
}
\end{center}

% ----------------- author(s) -------------------------

\begin{center}
{\bf Morten Hjorth-Jensen}
\end{center}

    \begin{center}
% List of all institutions:
\centerline{{\small \href{{http://www.nscl.msu.edu/}}{National Superconducting Cyclotron Laboratory} and \href{{https://www.pa.msu.edu/}}{Department of Physics and Astronomy}, \href{{http://www.msu.edu/}}{Michigan State University}, East Lansing, MI 48824, USA}}
\end{center}
    
% ----------------- end author(s) -------------------------

% --- begin date ---
\begin{center}
Spring 2016 
\end{center}
% --- end date ---

\vspace{1cm}


% !split
\subsection*{Overview of first week}

% --- begin paragraph admon ---
\paragraph{}

\begin{itemize}
\item First two weeks:
\begin{itemize}

 \item Presentation of topics to be covered and introduction to nuclear structure physics

 \item Discussion of quantities like binding energies, masses, radii, separation energies, see chapters 1-4 of Alex Brown's (AB) 2011 lectures

 \item Single-particle degrees of freedom, discussion of data

 \item Hamiltonians and single-particle fields, see  Suhonen (JS) chapter 3  and AB chapters 9 and 10.
\end{itemize}

\noindent
\end{itemize}

\noindent
% --- end paragraph admon ---




% !split
\subsection*{Overview of first week}

% --- begin paragraph admon ---
\paragraph{}
As a reading assignment, chapters 1-4 of Alex Brown's text are rather useful.
You can fetch Suhonen's book \href{{http://link.springer.com.proxy2.cl.msu.edu/book/10.1007/978-3-540-48861-3/page/1}}{online} using your MSU library \href{{https://www.lib.msu.edu/general/account/}}{access}.
You can find these texts at the webpage of the course as well.
The \href{{http://nuclearstructure.github.io/PHY981/doc/web/course.html}}{course} link contains 
all material from the lectures in various formats (html, ipython notebooks and PDF).
% --- end paragraph admon ---






% !split
\subsection*{Lectures, exercise sessions and syllabus}

% --- begin paragraph admon ---
\paragraph{}
\begin{itemize}
\item Lectures: Tuesdays (2pm-3.50pm, Theory trailer, conference room)

\item Detailed lecture notes, all exercises presented and projects can be found at the homepage of the course, \href{{https://github.com/NuclearStructure/PHY981}}{\nolinkurl{https://github.com/NuclearStructure/PHY981}}.

\item Exercises: No allocated time (but a given time can be determined)

\item Weekly plans and all other information are on the webpage of the course.
\end{itemize}

\noindent
% --- end paragraph admon ---




% !split
\subsection*{Lectures, exercise sessions and syllabus}

% --- begin paragraph admon ---
\paragraph{}
\begin{itemize}
\item Syllabus: Lecture notes, exercises and projects. Relevant chapters of Suhonen's text that cover parts of the material are chapters 3-9 and 11. Chapters 1-2 on angular momentum will be used as references for various derivations and only parts of these two chapters will be used. Alex Brown's lecture notes from 2011 can also be used, and the relevant chapters are 1-4, 6-29. Chapter 5 of Brown on angular momentum plays the same role as chapters 1-2 of Suhonen. 

\item A weekly mail will be sent to all participants. 
\end{itemize}

\noindent
% --- end paragraph admon ---




% !split
\subsection*{Links to all course material}

% --- begin paragraph admon ---
\paragraph{}
\begin{itemize}
\item PHY981 on github
\begin{itemize}

 \item All material is at  \href{{https://github.com/NuclearStructure/PHY981/}}{\nolinkurl{https://github.com/NuclearStructure/PHY981/}}

 \item The link \href{{http://nuclearstructure.github.io/PHY981/doc/web/course.html}}{\nolinkurl{http://nuclearstructure.github.io/PHY981/doc/web/course.html}} gives direct access to html, ipython notebooks and pdf files of the lectures.
\end{itemize}

\noindent
\end{itemize}

\noindent
% --- end paragraph admon ---





% !split
\subsection*{Main themes and reading suggestions}

% --- begin paragraph admon ---
\paragraph{}
The various observables we will discuss in the course are thought to be understood via the following five 
major topics
\begin{itemize}
\item Single  particle properties and mean-field models, lecture notes, JS chapters 3-5 and AB chapters 7-10 and 14

\item Nuclear forces, covered by lecture notes

\item The nuclear shell-model, lecture notes, JS chapter 8 and AB chapters 11-22.

\item Particle-hole excitations, random-phase approximation and pairing (and perhaps something on collective models), lecture notes and JS chapters 9 and 11.

\item Decays and one and two-body transition probabilities, lecture notes, JS chapters 6 and 7, AB chapters 23-29.
\end{itemize}

\noindent
To understand how these topics are linked will provide us with fundamental insights about the laws of motion that govern nuclear physics.
% --- end paragraph admon ---




% !split
\subsection*{Plan for the semester}

% --- begin paragraph admon ---
\paragraph{Projects, weekly exercises, deadlines and final oral exam.}
\begin{itemize}
\item Two  projects with a possible numerical content that count 25\% each of the final mark, weekly exercises  that count 20\% and a final oral exam which counts 30\% of the final grade. 

\item Project 1 will be available in week 6 (begins with February 8) and to be handed in before spring break (March 4)

\item Project 2 will be available in week 13 (begins with March 28) with deadline April 18.

\item For the final oral exam you have to prepare a 25-30 minutes talk based on either a topic chosen by you or a topic defined towards the end of the semester.
\end{itemize}

\noindent
Duration of the examination is 45 minutes.
% --- end paragraph admon ---




% !split
\subsection*{Selected Texts on Nuclear Structure and Many-body theory}

% --- begin paragraph admon ---
\paragraph{}
\begin{itemize}
\item Nuclear structure
\begin{itemize}

 \item Heyde, \emph{The Nuclear Shell Model}, Springer 1990

 \item Lawson, \emph{Theory of the Nuclear Shell Model}, Oxford 1980

 \item Ring and Schuck, \emph{Nuclear Many-Body Theory}, Springer 1980

 \item Talmi, \emph{Simple Models of Complex Nuclei: The Shell Model and Interacting Boson Model}, Harwood Academic Publishers 1993.

\end{itemize}

\noindent
\item Many-body theories
\begin{itemize}

 \item Blaizot and Ripka, \emph{Quantum Theory of Finite systems}, MIT press 1986

 \item Fetter and Walecka, \emph{Quantum Theory of Many-Particle Systems}, McGraw-Hill, 1971.

 \item Dickhoff and Van Neck, \emph{Many-Body Theory Exposed}, World Scientific, 2006.
\end{itemize}

\noindent
\end{itemize}

\noindent
% --- end paragraph admon ---








% !split
\subsection*{Background enquiry, first exercise}

% --- begin paragraph admon ---
\paragraph{Write a small summary of what you do.}
You can send the answer as an email to hjensen@msu.edu
\begin{itemize}
\item What is your background in nuclear physics, courses taken or attending this semester? 

\item If you have defined a thesis topic, please send me some details of your thesis project, your interests etc.

\item What is your background in computing? And, if you have programmed, which programming language(s) and environments  are you  most familiar with?  

\item Also, if you have specific wishes with respect to this course, expectations, topics you'd like me to cover or other things, please feel free to write them down, or swing by my office for a chat. 
\end{itemize}

\noindent
One of my aims is to be able to tailor this course as close as possible to your specific scientific interests (as far as possible obviously).
% --- end paragraph admon ---



% !split
\subsection*{First (real) exercise: Exercise 1}

% --- begin paragraph admon ---
\paragraph{Masses and binding energies.}
The data on binding energies can be found in the file bedata.dat at the github address of the \href{{https://github.com/NuclearStructure/PHY981/tree/master/doc/pub/spdata/programs}}{course}

\begin{itemize}
  \item Write a small program which reads in the proton and neutron numbers and the binding energies 
\end{itemize}

\noindent
and make a plot of all neutron separation energies for the chain of oxygen (O), calcium (Ca), nickel (Ni), tin (Sn) and lead (Pb) isotopes, that is you need to plot
\[
S_n= BE(N,Z)-BE(N-1,Z).
\]
Comment your results. 
\begin{itemize}
 \item In the same figures, you should also include the liquid drop model results of Eq.~(2.17) of Alex Brown's text, namely
\end{itemize}

\noindent
\[
BE(N,Z)= \alpha_1A-\alpha_2A^{2/3}-\alpha_3\frac{Z^2}{A^{1/3}}-\alpha_4\frac{(N-Z)^2}{A},
\]
with $\alpha_1=15.49$ MeV, $\alpha_2=17.23$ MeV, $\alpha_3=0.697$ MeV and $\alpha_4=22.6$ MeV.
% --- end paragraph admon ---





% !split
\subsection*{First (real) exercise: Exercise 1, continues}

% --- begin paragraph admon ---
\paragraph{Masses and binding energies.}
\begin{itemize}
 \item Make a plot of the binding energies as function of the number of nucleons $A$ using the data in the file on bindingenergies and the above liquid drop model.  Make a figure similar to figure 2.5 of Alex Brown where you set the various parameters $\alpha_i=0$. Comment your results. 

 \item Use the liquid drop model to find the neutron drip lines   for Z values up to 120.
\end{itemize}

\noindent
Analyze then the fluorine isotopes and find, where available the corresponding experimental data, and compare the liquid drop model predicition with experiment. 
Comment your results.

A program example in C++ and the input data file \emph{bedata.dat} can be found found at the github repository for the \href{{https://github.com/NuclearStructure/PHY981/tree/master/doc/pub/spdata/programs}}{course}

Deadline for this exercise is \textbf{January 22, 5pm}.  You can hand in electronically by just sending me your github link, or just the file. I digest most formats, from scans to ipython notebooks. The choice is yours.
% --- end paragraph admon ---






% !split
\subsection*{Links and useful software}

% --- begin paragraph admon ---
\paragraph{}
\begin{itemize}
\item Useful links
\begin{itemize}

 \item When I write code in C++ I tend to recommend to use the linear algebra library \href{{http://arma.sourceforge.net/}}{armadillo}. 

 \item For OS X users I recommend using \href{{http://brew.sh/}}{brew}.

 \item To set up github, use the \href{{https://guides.github.com/}}{guides} and/or the github \href{{https://github.com/}}{link}. 

 \item An excellent IDE for c++ programmers is \href{{http://www.qt.io/ide/}}{Qt Creator}
\end{itemize}

\noindent
\end{itemize}

\noindent
% --- end paragraph admon ---



% ------------------- end of main content ---------------

\end{document}

