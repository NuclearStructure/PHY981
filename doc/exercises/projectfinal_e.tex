\documentclass[prc]{revtex4}
\usepackage[dvips]{graphicx}
\usepackage{mathrsfs}
\usepackage{amsfonts}
\usepackage{lscape}

\usepackage{epic,eepic}
\usepackage{amsmath}
\usepackage{amssymb}
\usepackage[dvips]{epsfig}
\usepackage[T1]{fontenc}
\usepackage{hyperref}
\usepackage{bezier}
\usepackage{pstricks}
\usepackage{dcolumn}% Align table columns on decimal point
\usepackage{bm}% bold math
%\usepackage{braket}
\usepackage[dvips]{graphicx}
\usepackage{pst-plot}

\newcommand{\One}{\hat{\mathbf{1}}}
\newcommand{\eff}{\text{eff}}
\newcommand{\Heff}{\hat{H}_\text{eff}}
\newcommand{\Veff}{\hat{V}_\text{eff}}
\newcommand{\braket}[1]{\langle#1\rangle}
\newcommand{\Span}{\operatorname{sp}}
\newcommand{\tr}{\operatorname{trace}}
\newcommand{\diag}{\operatorname{diag}}
\newcommand{\bra}[1]{\left\langle #1 \right|}
\newcommand{\ket}[1]{\left| #1 \right\rangle}
\newcommand{\element}[3]
    {\bra{#1}#2\ket{#3}}

\newcommand{\normord}[1]{
    \left\{#1\right\}
}


\usepackage{amsmath}
\begin{document}


\title{Project for the final oral examination of PHY981}
%\author{}
\maketitle
\section*{Nucleon-nucleon scattering}

The aim of this project is to solve the Lippman-Schwinger equation for two interacting nucleons and relate the obtained phase shifts with those extracted from the experimental cross sections.

We are going to solve the Schr\"odinger equation (SE) 
for the neutron-proton system 
in momentum space for positive energies $E$ in order to obtain
the phase shifts. 
We can rewrite the SE 
in momentum space as
\begin{equation}
     \frac{k^2}{m}\psi_l(k)+\frac{2}{\pi}\int_0^{\infty}dqq^2V_l(k,q)\psi_l(q)=E\psi_l(k).
     \label{eq:sem}
\end{equation}
Here we have used units $\hbar=c=1$. This means that $k$ has dimension energy. 
$k$ is the relative momentum between the two particles. A partial
wave expansion has been used in order to reduce the problem to an integral
over the magnitude of momentum only. The subscript $l$ refers therefore to a partial wave with a given orbital momentum $l$.
To obtain the potential in momentum space we used 
the Fourier-Bessel transform (Hankel transform)
\begin{equation}
      V_l(k,k')= \int j_l(kr)V(r)j_l(k'r)r^2dr,
      \label{eq:vtrans}
\end{equation}
where $j_l$ is the spherical Bessel function. 
We will just study the case
$l=0$, which means that $j_0(kr)=sin(kr)/kr$. 

For scattering states, $E>0$, the corresponding equation to solve is 
the so-called Lippman-Schwinger equation. This is an integral equation
where we have to deal with the amplitude $R(k,k')$ (reaction matrix) 
defined through the integral equation 
\begin{equation}
    R_l(k,k') = V_l(k,k') +\frac{2}{\pi}{\cal P}
                \int_0^{\infty}dqq^2V_l(k,q)\frac{1}{E-q^2/m}R_l(q,k'),
   \label{eq:ls1}
\end{equation}
where the total kinetic energy of the two 
incoming particles in the center-of-mass system
is 
\begin{equation}
    E=\frac{k_0^2}{m}.
\end{equation}
The symbol ${\cal P}$ indicates that Cauchy's principal-value prescription
is used in order to avoid the singularity arising from the zero of the denominator.
We will discuss below how to solve this problem. Eq.\ (\ref{eq:ls1}) represents
then the problem you will have to solve numerically.

The matrix $R_l(k,k')$ relates to the 
the  phase shifts through its diagonal elements as
\begin{equation}
     R_l(k_0,k_0)=-\frac{tan\delta_l}{mk_0}.
     \label{eq:shifts}
\end{equation}

From now on we will drop the subscript $l$ in all equations.

In order to solve the Lippman-Schwinger equation 
in momentum space, we need first to write 
a function which sets up the mesh points. 
We need to do that since we are going to approximate an integral
through 
\[
   \int_a^bf(x)dx\approx\sum_{i=1}^Nw_if(x_i),
\]
where we have fixed $N$ lattice points through the corresponding weights
$w_i$ and points $x_i$. 
Start writing your main program by setting up the mesh points
and the corresponding weights.
Fix first the number of mesh points $N$.
Use the included function {\em gauleg} in the file lib.cpp (see also lib.h) to set up the 
weights $\omega_i$ and the points $k_i$. Before you go on 
you need to keep in mind that {\em gauleg} uses 
the Legendre polynomials to fix
the mesh points and weights. This means that the integral is for the 
interval [-1,1]. Your integral is for the interval [0,$\infty$]. 
You will need to map the weights  from {\em gauleg} to your interval.
To do this, call first {\em gauleg(a,b,x[],w[],N)}, 
with $a=-1$, $b=1$.
It returns the mesh points $x_i$ and weights $w_i$. 
You map these points over to the limits in your integral. You can then
use the following mapping
        \[
          k_i=const\times tan\left\{\frac{\pi}{4}(1+x_i)\right\},
        \]
and 
         \[
            \omega_i= const\frac{\pi}{4}\frac{w_i}{cos^2\left(\frac{\pi}{4}(1+x_i)\right)}.
         \]
If you choose dimension fm$^{-1}$ for $k$, set $const=1$. If you choose to work
with MeV, set $const\sim 200$ ($\hbar c=197$ MeVfm).
You must choose the units to use.

The next step is to write a function which calculates the potential
in momentum space. The potential we will use here is a parametrized  potential
between a proton and neutron for the partial wave  $^1S_0$, 
i.e., spin $S=0$ and
orbital momentum $l=0$, a singlet S-state. 
This state does not have a bound state
for the deuteron (only the triplet S-state has). 
The parametrized version of this potential fits the experimental
phase-shifts. It is given by
\begin{equation}
  V(r)=V_a \frac{e^{-ax}}{x}+V_b \frac{e^{-bx}}{x}+V_c \frac{e^{-cx}}{x}
  \label{eq:realp}
\end{equation}
with $x=\mu r$, $\mu=0.7$ fm$^{-1}$ (the inverse of the pion mass),
$V_a=-10.463$ MeV and $a=1$, $V_b=-1650.6$ MeV and $b=4$ and
$V_c=6484.3$ MeV and $c=7$. 
Find the potential in momentum space using Eq.\ (\ref{eq:vtrans})
with $j_0(kr)=sin(kr)/kr$. 
The transform of a potential on the form 
$V(r)=V_0\exp{(-\mu r)}/r$ is
\begin{equation}
     V(k',k)= \frac{V_0}{4k'k}ln\left(\frac{(k'+k)^2+\mu^2}{(k'-k)^2+\mu^2}\right).
\end{equation}

Write a function which calculates the expressions for the
potential in momentum space.

The principal value in Eq.\ (\ref{eq:ls1}) is rather tricky
to evaluate numerically, mainly since computers have limited
precision. We will here use a subtraction trick often used
when dealing with singular integrals in numerical calculations.
We introduce first the calculus relation
\begin{equation}
  \int_{-\infty}^{\infty} \frac{dk}{k-k_0} =0.
\end{equation}
It means that the curve $1/(k-k_0)$ has equal and opposite
areas on both sides of the singular point $k_0$. If we break
the integral into one over positive $k$ and one over 
negative $k$, a change of variable $k\rightarrow -k$ 
allows us to rewrite the last equation as
\begin{equation}
  \int_{0}^{\infty} \frac{dk}{k^2-k_0^2} =0.
\end{equation}
We can use this to express a principal values integral
as
\begin{equation}
  {\cal P}\int_{0}^{\infty} \frac{f(k)dk}{k^2-k_0^2} =
  \int_{0}^{\infty} \frac{(f(k)-f(k_0))dk}{k^2-k_0^2},
   \label{eq:trick}
\end{equation}
where the right-hand side is no longer singular at 
$k=k_0$, it is proportional to the derivative $df/dk$,
and can be evaluated numerically as any other integral.

We can then use the trick in Eq.\ (\ref{eq:trick}) to rewrite
Eq.\ (\ref{eq:ls1}) as
\begin{equation}
    R(k,k') = V(k,k') +\frac{2}{\pi}
                \int_0^{\infty}dq
                \frac{q^2V(k,q)R(q,k')-k_0^2V(k,k_0)R(k_0,k')  }
                     {(k_0^2-q^2)/m}.
   \label{eq:ls2}
\end{equation}
{\bf This is the equation you are going to solve numerically in order
to calculate the phase shifts of 
Eq.\ (\ref{eq:shifts}). } We are interested in obtaining
$R(k_0,k_0)$.

How do we proceed in order to solve Eq.\ (\ref{eq:ls2})?
\begin{enumerate}
  \item  Using the mesh points $k_j$ and the weights $\omega_j$,
         we can rewrite Eq.\ (\ref{eq:ls2}) as
\begin{equation}
          R(k,k') = V(k,k') +\frac{2}{\pi}
          \sum_{j=1}^N\frac{\omega_jk_j^2V(k,k_j)R(k_j,k')}
                           {(k_0^2-k_j^2)/m}
           -\frac{2}{\pi}k_0^2V(k,k_0)R(k_0,k')
          \sum_{n=1}^N\frac{\omega_n}
                           {(k_0^2-k_n^2)/m}.                
          \label{eq:ls3}
\end{equation}
This equation contains now the unknowns $R(k_i,k_j)$
(with dimension $N\times N$) and $R(k_0,k_0)$.
\item 
We can turn Eq.\ (\ref{eq:ls3}) into an equation
with dimension $(N+1)\times (N+1)$ with  a mesh
which contains the original mesh points $k_j$ for $j=1,N$
and the point which corresponds to the energy $k_0$.
Consider the latter as the 'observable' point.
The mesh points become then $k_j$ for $j=1,n$ and
$k_{N+1}=k_0$. 
\item With these new mesh points we define the matrix
\begin{equation}
      A_{i,j}=\delta_{i,j}-V(k_i,k_j)u_j,
      \label{eq:aeq}
\end{equation}
where $\delta$ is the Kronecker $\delta$
and
\begin{equation}
     u_j=\frac{2}{\pi}
         \frac{\omega_jk_j^2}{(k_0^2-k_j^2)/m}\hspace{1cm}
         j=1,N
\end{equation}
and
\begin{equation}
     u_{N+1}=-\frac{2}{\pi}
          \sum_{j=1}^N\frac{k_0^2\omega_j}{(k_0^2-k_j^2)/m}.
\end{equation}
The first task is then to 
set up the matrix $A$ for a given $k_0$. This is an
$(N+1)\times (N+1)$ matrix. It can be convenient
to have an outer loop which runs over the chosen
observable values for the energy $k_0^2/m$.
{\em Note that all mesh points $k_j$ for $j=1,N$ must be
different from $k_0$. Note also that
$V(k_i,k_j)$ is an
$(N+1)\times (N+1)$ matrix}. Write a small function
which sets up $A$.
\item
  With the matrix $A$ we can rewrite Eq.\ (\ref{eq:ls3}) 
  as a matrix problem of dimension $(N+1)\times (N+1)$.
  All matrices $R$, $A$ and $V$ have this dimension
  and we get
\begin{equation}
    A_{i,l}R_{l,j}=V_{i,j},
\end{equation} 
or just
\begin{equation}
    AR=V.
    \label{eq:final1}
\end{equation} 
\item Since you already have defined $A$ and $V$
(these are stored as $(N+1)\times (N+1)$ matrices) 
Eq.\ (\ref{eq:final1}) involves only the unknown
$R$. We obtain it by matrix inversion, i.e.,
\begin{equation}
    R=A^{-1}V.
    \label{eq:final2}
\end{equation} 
Thus, to obtain $R$, you will need to set up the matrices
$A$ and $V$ and invert the matrix $A$. To do that
you must call the function
{\em matinv} in the program library lib.cpp (see also lib.h).
With the inverse $A^{-1}$, performing
a matrix multiplication with $V$ results in $R$.
\end{enumerate}

With $R$ you can then evaluate the phase shifts
by noting that 
\begin{equation}
      R(k_{N+1},k_{N+1})=R(k_0,k_0),
\end{equation}
and you are done.

You can choose to read $k_0$ from file or screen, or set up
a loop over chosen values of $k_0$ and for each
$k_0$ solve Eq.\ (\ref{eq:final2}). 



\begin{enumerate}
\item When you have $R(k,k')$ for the given potential, evaluate now the phase-shifts 
using
\[
     R(k_0,k_0)=-\frac{tan\delta}{mk_0}.
\]
\item Compare the phase shifts for the potential of Eq.\ (\ref{eq:realp}) with the 
experimental phase shifts that can be found 
in the article 
of the Nijmegen group in Physical Review C {\bf 48}, 792 (1993).  Alternatively look up their webpage at
\url{http://nn-online.org/}
\end{enumerate}

The code can easily be extended to include realistic nucleon-nucleon interactions.
Let me know if this is of interest, I will then provide a function (fortran) for the nucleon-nucleon force.
\end{document}
